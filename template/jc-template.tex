% ============================================================
% IQB Journal Club Template
% 空白模板 - 快速启动你的 JC 演示
% ============================================================

\documentclass[aspectratio=169,11pt]{beamer}

% Load IQB theme
\usepackage{../theme/beamerthemeiqb}
\usepackage{../theme/iqb-layouts}

% Set header image path (adjust if needed based on your directory structure)
\renewcommand{\iqbheaderimage}{../theme/images/header.png}

% Additional useful packages
\usepackage{graphicx}      % 图片支持
\usepackage{amsmath}       % 数学公式
\usepackage{amssymb}       % 数学符号
\usepackage{booktabs}      % 表格美化

% Chinese support (if needed, requires XeLaTeX)
% \usepackage{xeCJK}
% \setCJKmainfont{SimSun}  % Windows: SimSun, Linux: Noto Sans CJK SC

% ============================================================
% TODO: 填写你的演示信息
% ============================================================
\title{你的文献标题}
\subtitle{副标题(可选)}
\author{你的名字}
\institute{IQB Lab}
\date{\today}  % 或指定日期:\date{2025-10-20}

% ============================================================
\begin{document}

% ============================================================
% 封面页 (Cover Page) - 无 header/footer
% ============================================================
\begin{frame}[plain,noframenumbering]
  \titlepage
  % 或自定义封面:
  % \vfill
  % \centering
  % {\LARGE\textcolor{iqbblue}{\textbf{你的标题}}}
  % \vfill
\end{frame}

% ============================================================
% 内容页开始
% ============================================================

% TODO: 设置当前 section(会显示在 footer 中间)
\setsection{Background}

% ------------------------------------------------------------
% 示例页 1:纯文本
% ------------------------------------------------------------
\begin{frame}{背景}
  \begin{itemize}
    \item 要点 1
    \item 要点 2
    \item 要点 3
  \end{itemize}

  \vspace{0.5cm}

  \begin{block}{重要结论}
    这里是一个 block 环境,用于强调关键信息。
  \end{block}
\end{frame}

% ------------------------------------------------------------
% 示例页 2:双列布局
% ------------------------------------------------------------
\begin{frame}{双列布局示例}
  \iqblayouttwo{
    % 左列
    \textbf{左侧内容}:
    \begin{itemize}
      \item 观点 A
      \item 观点 B
    \end{itemize}

    % 可以插入图片
    % \includegraphics[width=\textwidth]{path/to/image.png}
  }{
    % 右列
    \textbf{右侧内容}:
    \begin{itemize}
      \item 观点 C
      \item 观点 D
    \end{itemize}
  }
\end{frame}

% ------------------------------------------------------------
% 示例页 3:图片 + 文字
% ------------------------------------------------------------
\setsection{Methods}  % 更新 footer section

\begin{frame}{方法部分}
  \iqbimagetext[width=0.4\textwidth]{example-image-a}{
    % 右侧文字说明
    \textbf{方法描述}:

    \begin{enumerate}
      \item 步骤 1
      \item 步骤 2
      \item 步骤 3
    \end{enumerate}

    \vspace{0.3cm}

    这里可以添加更多解释文字。
  }

  % 注意:example-image-a 是 LaTeX 内置的示例图片
  % 替换为你的图片路径,如:images/your-figure.png
\end{frame}

% ------------------------------------------------------------
% 示例页 4:三列布局
% ------------------------------------------------------------
\setsection{Results}

\begin{frame}{结果对比}
  \iqblayoutthree{
    % 列 1
    \centering
    \textbf{方法 A}

    % \includegraphics[width=\textwidth]{images/result-a.png}

    准确率: 85\%
  }{
    % 列 2
    \centering
    \textbf{方法 B}

    % \includegraphics[width=\textwidth]{images/result-b.png}

    准确率: 90\%
  }{
    % 列 3
    \centering
    \textbf{我们的方法}

    % \includegraphics[width=\textwidth]{images/result-ours.png}

    准确率: 95\%
  }
\end{frame}

% ------------------------------------------------------------
% 示例页 5:2×2 网格
% ------------------------------------------------------------
\begin{frame}{四图网格}
  \iqbgridtwobytwo{
    \centering
    % \includegraphics[width=\textwidth]{images/fig1.png}
    \small (A) 图 1 说明
  }{
    \centering
    % \includegraphics[width=\textwidth]{images/fig2.png}
    \small (B) 图 2 说明
  }{
    \centering
    % \includegraphics[width=\textwidth]{images/fig3.png}
    \small (C) 图 3 说明
  }{
    \centering
    % \includegraphics[width=\textwidth]{images/fig4.png}
    \small (D) 图 4 说明
  }
\end{frame}

% ------------------------------------------------------------
% 示例页 6:数学公式
% ------------------------------------------------------------
\setsection{Discussion}

\begin{frame}{公式示例}
  \textbf{自由能计算}:

  \vspace{0.3cm}

  成核阶段:
  $$\Delta G(\text{CV}) = k \cdot \text{CV}^2 + c$$

  \vspace{0.3cm}

  扩展阶段:
  $$G(r) = 2\pi r\gamma$$

  \vspace{0.3cm}

  其中 $\gamma$ 为线张力,$r$ 为孔半径。
\end{frame}

% ============================================================
% 致谢页 (Thank You) - 无 header/footer
% ============================================================
\begin{frame}[plain,noframenumbering]
  \vfill
  \centering

  {\Huge \textcolor{iqbblue}{\textbf{Thank You!}}}

  \vspace{1.5cm}

  {\Large Questions?}

  \vspace{2cm}

  % 可选:添加联系信息
  % {\normalsize
  % Contact: yourname@iqblab.edu\\
  % IQB Lab, 2025
  % }

  \vfill
\end{frame}

% ============================================================
% 常用布局命令速查
% ============================================================

% 双列(50-50):
% \iqblayouttwo{左列内容}{右列内容}

% 双列(1/3-2/3):
% \iqblayoutonethird{左列(窄)}{右列(宽)}

% 双列(2/3-1/3):
% \iqblayouttwothirds{左列(宽)}{右列(窄)}

% 三列(均分):
% \iqblayoutthree{列1}{列2}{列3}

% 2×2 网格:
% \iqbgridtwobytwo{图1}{图2}{图3}{图4}

% 图片 + 文字(图在左):
% \iqbimagetext[width=0.4\textwidth]{image.png}{右侧文字}

% 文字 + 图片(图在右):
% \iqbtextimage[width=0.4\textwidth]{左侧文字}{image.png}

% 设置 footer section:
% \setsection{Section Name}

% 无 header/footer 的页面:
% \begin{frame}[plain,noframenumbering]
%   内容
% \end{frame}

% ============================================================

\end{document}
