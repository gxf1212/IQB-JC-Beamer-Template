% ============================================================
% IQB Journal Club Beamer Template
%
% 快速启动指南 (Quick Start Guide)
% 版本: 1.0 | 最后更新: 2025-10-20
%
% 这是一个完整的空白模板,用于创建 IQB 课题组文献分享演示。
% 该模板已配置好 IQB 主题、布局工具、字体和样式。
% 你只需要修改内容并编译即可。
%
% 编译命令 (Compilation):
%   Linux/Mac:   xelatex -interaction=nonstopmode jc-template.tex
%   Windows WSL: /mnt/d/texlive/2022/bin/win32/xelatex.exe jc-template.tex
%
% ============================================================

\documentclass[aspectratio=169,11pt]{beamer}

% ============================================================
% 主题加载 (Load Theme Packages)
% ============================================================
\usepackage{../theme/beamerthemeiqb}
\usepackage{../theme/iqb-layouts}

% ============================================================
% 页面配置 (Header/Footer Configuration)
% ============================================================

% 设置页眉图片路径(如果你的目录结构不同,需要调整)
% Set header image path (adjust based on your directory structure)
\renewcommand{\iqbheaderimage}{../theme/images/header.png}

% 自定义机构名称 - 显示在页脚左侧
% Customize institute name shown in footer (left side)
% 默认值: Institute of Quantitative Biology
% 修改示例: \setiqbinstitute{Wang Lab @ IQB}
\setiqbinstitute{Institute of Quantitative Biology}

% 自定义机构logo(如果有的话)
% Customize logo path if needed (optional)
% \setiqblogopath{../theme/images/your-logo.png}

% ============================================================
% 必需包 (Required Packages)
% ============================================================
\usepackage{graphicx}      % 图片支持 (Graphics support)
\usepackage{amsmath}       % 数学公式 (Math formulas)
\usepackage{amssymb}       % 数学符号 (Math symbols)
\usepackage{booktabs}      % 表格美化 (Table formatting)
\usepackage{mhchem}        % 化学公式 (Chemical formulas: \ce{H2O})

% ============================================================
% 跨平台字体支持 (Cross-Platform Font Support)
% ============================================================
%
% 该模板支持 Windows, macOS, Linux 上的中文渲染。
% 需要使用 XeLaTeX 编译才能支持中文。
%
% 自动字体选择 (Automatic):
%   - Windows:  SimSun (内置)
%   - macOS:    Songti SC (内置)
%   - Linux:    Noto Serif CJK SC (需要安装)
%
% 手动指定字体 (Manual override, uncomment if needed):
%   \usepackage{xeCJK}
%   \setCJKmainfont{SimSun}  % 或其他字体名称
%
% 如果编译时出现中文乱码或缺失,请检查:
%   1. 确认使用 XeLaTeX (不是 pdflatex)
%   2. Linux 用户:安装 fonts-noto-cjk 包
%   3. 手动指定字体(参见上面的注释)

% ============================================================
% 颜色定义 (Color Definitions - from beamerthemeiqb.sty)
% ============================================================
%
% 主要颜色:
%   iqbblue      - IQB 主题色 #003366 (RGB: 0, 51, 102)
%   iqbdarkblue  - 深蓝 #001A33 (RGB: 0, 26, 51)
%   iqblightblue - 浅蓝 #4A90E2 (RGB: 74, 144, 226)
%   iqbgray      - 深灰 (RGB: 85, 85, 85)
%
% 强调颜色:
%   iqbgreen     - 绿色 #2ECC71
%   iqborange    - 橙色 #FF6B35
%   iqbred       - 红色 #E74C3C
%
% 使用示例: \textcolor{iqbblue}{蓝色文字}

% ============================================================
% 间距命令 (Spacing Commands - Unified System)
% ============================================================
%
% 使用统一的间距命令而不是 \vspace{}:
%   \iqbtinysep   - 很小的间距 (~0.15cm),用于紧凑布局
%   \iqbsep       - 标准间距 (~0.3cm),最常用
%   \iqbbigsep    - 大间距 (~0.5cm),用于分隔内容块
%
% 示例:
%   \iqbsep
%   \begin{block}{标题}
%     内容
%   \end{block}

% ============================================================
% 演示信息填写 (Fill In Your Presentation Information)
% ============================================================
\title{你的文献标题 (Your Literature Title)}
\subtitle{副标题 - 可选 (Subtitle - Optional)}
\author{你的名字 (Your Name)}

% 机构信息 - 会显示在封面和 footer 中
% Institute - shown on cover page and footer
\institute{IQB Lab}

% 日期 - 默认为当前日期,也可以手动指定
% Date - defaults to today, or specify manually
\date{\today}  % 或指定: \date{2025-10-20}

% ============================================================
\begin{document}

% ============================================================
% 封面页 (Title Page) - 无 header/footer
% ============================================================
% plain 选项: 不显示 header/footer/navigation
% noframenumbering 选项: 不计入页码
%
\begin{frame}[plain,noframenumbering]
  \titlepage

  % 自定义封面的另一种方式(取消注释可使用):
  % \vfill
  % \centering
  % {\LARGE\textcolor{iqbblue}{\textbf{你的标题}}}
  % \vspace{1cm}
  % {\large 你的名字}
  % \vfill
\end{frame}

% ============================================================
% 内容页开始 (Content Pages)
% ============================================================

% 设置 footer 中间显示的 section 名称
% Update this periodically as you move through different sections
\setsection{Background}

% ============================================================
% 示例页 1: 纯文本页面 (Text-Only Page)
% ============================================================
% 适用场景: 背景介绍、关键概念、主要问题等
% 最佳实践:
%   - 使用 bullet points 而不是长段落
%   - 每行不超过 50 个字符
%   - 使用 block 环境突出重要信息
%
\begin{frame}{背景:研究问题的提出}
  % 使用 itemize 列出关键点(不要超过 5 个)
  \begin{itemize}
    \item 第一个关键问题或背景
    \item 第二个相关的现象
    \item 第三个需要解决的问题
  \end{itemize}

  % 使用统一间距命令
  \iqbsep

  % 用 block 环境突出核心观点
  \begin{block}{核心问题}
    这里用来强调该页面的主要洞察或问题陈述。
    可以是一句话或两句话,字数不要太多。
  \end{block}

  % 另一个 block 示例(可选)
  % \iqbsep
  % \begin{block}{已知限制}
  %   描述现有方法的不足之处
  % \end{block}
\end{frame}

% ============================================================
% 示例页 2: 双列布局 (Two-Column Layout)
% ============================================================
% 适用场景: 对比方法、左右并行、对偶分析
% 命令: \iqblayouttwo{左侧}{右侧}
% 布局: 50% - 50%
% 最佳实践:
%   - 左右两侧内容保持平衡
%   - 可在任一侧放图片或文字
%   - 使用相同字体大小以保持视觉一致性
%
\begin{frame}{对比分析:方法 A 与方法 B}
  \iqblayouttwo{
    % 左列 (Left Column)
    \textbf{方法 A 的特点:}
    \begin{itemize}
      \item 优点 1:快速
      \item 优点 2:简单
      \item 缺点 1:精度低
    \end{itemize}
    \iqbtinysep
    \textcolor{iqborange}{适用于粗筛}

  }{
    % 右列 (Right Column)
    \textbf{方法 B 的特点:}
    \begin{itemize}
      \item 优点 1:精确
      \item 优点 2:可靠
      \item 缺点 1:耗时
    \end{itemize}
    \iqbtinysep
    \textcolor{iqbgreen}{适用于精细分析}

  }
\end{frame}

% ============================================================
% 示例页 3: 图片 + 文字布局 (Image + Text)
% ============================================================
% 适用场景: 方法示意图、装置设置、步骤说明
% 命令: \iqbimagetext[width=0.4\textwidth]{image.png}{右侧文字}
% 布局: 左图 (40%) + 右文 (60%)
% 最佳实践:
%   - 图片高度: 0.4-0.5 \textheight
%   - 右侧文字用列表或枚举
%   - 若文字过多,拆分为两页
%
\setsection{Methods}

\begin{frame}{关键方法:切换函数的设计}
  \iqbimagetext[width=0.42\textwidth]{example-image-a}{
    % 右侧文字说明
    \textbf{创新设计步骤:}

    \begin{enumerate}
      \item 定义开关条件
      \item 设计过渡函数
      \item 参数优化
    \end{enumerate}

    \iqbsep

    \textbf{核心优势:}
    \begin{itemize}
      \item 光滑过渡
      \item 物理意义清晰
      \item 计算效率高
    \end{itemize}

    \iqbtinysep
    更详细的解释可以写成具体数值。
  }

  % 说明: example-image-a 是 LaTeX 内置的示例图片
  % 替换为你的图片路径:images/your-figure.png
\end{frame}

% ============================================================
% 示例页 4: 三列布局 (Three-Column Layout)
% ============================================================
% 适用场景: 三方对比、多个结果、方法并列
% 命令: \iqblayoutthree{列1}{列2}{列3}
% 布局: 33% - 33% - 33%
% 最佳实践:
%   - 三列内容保持对称和平衡
%   - 每列可放一张图 (height=0.35-0.4 \textheight)
%   - 或文字说明 + 指标
%   - 字号较小,避免溢出
%
\setsection{Results}

\begin{frame}{方法对比:三种方案的性能}
  \iqblayoutthree{
    % 列 1
    \centering
    \textbf{传统方法}

    % \includegraphics[height=0.35\textheight]{images/result-a.png}

    准确率: $85\%$

    运行时间: $10$ 秒
  }{
    % 列 2
    \centering
    \textbf{改进方法}

    % \includegraphics[height=0.35\textheight]{images/result-b.png}

    准确率: $90\%$

    运行时间: $8$ 秒
  }{
    % 列 3
    \centering
    \textbf{我们的方法}

    % \includegraphics[height=0.35\textheight]{images/result-ours.png}

    准确率: $95\%$

    运行时间: $12$ 秒
  }
\end{frame}

% ============================================================
% 示例页 5: 2×2 网格布局 (2×2 Grid)
% ============================================================
% 适用场景: 展示 4 个相关的图表、不同条件下的结果
% 命令: \iqbgridtwobytwo{图1}{图2}{图3}{图4}
% 布局: 2 行 × 2 列的网格
% 最佳实践:
%   - 每个图都要有编号和说明 ((A), (B), (C), (D))
%   - 使用统一的图片大小
%   - 可选: 在图下方添加数值或结论
%
\begin{frame}{多图对比:4 个实验条件}
  \iqbgridtwobytwo{
    % 图 1 (左上)
    \centering
    % \includegraphics[width=0.45\textwidth]{images/fig1.png}
    \textbf{(A)} 条件 1

    参数值: $10$ 倍
  }{
    % 图 2 (右上)
    \centering
    % \includegraphics[width=0.45\textwidth]{images/fig2.png}
    \textbf{(B)} 条件 2

    参数值: $20$ 倍
  }{
    % 图 3 (左下)
    \centering
    % \includegraphics[width=0.45\textwidth]{images/fig3.png}
    \textbf{(C)} 条件 3

    参数值: $30$ 倍
  }{
    % 图 4 (右下)
    \centering
    % \includegraphics[width=0.45\textwidth]{images/fig4.png}
    \textbf{(D)} 条件 4

    参数值: $40$ 倍
  }
\end{frame}

% ============================================================
% 示例页 6: 数学公式 (Mathematical Equations)
% ============================================================
% 适用场景: 理论推导、模型方程、数学定义
% 最佳实践:
%   - 公式前后用 \iqbsep 分隔
%   - 重要公式用 $$ $$ (displayed) 而不是 $ $ (inline)
%   - 在公式下方用文字解释变量含义
%   - 使用 mhchem 处理化学式: \ce{H2O}, \ce{Ca2+}
%
\setsection{Discussion}

\begin{frame}{理论基础:自由能函数}
  % 第一个公式块
  \textbf{成核阶段的自由能:}
  \iqbtinysep

  $$\Delta G_{\text{nuc}}(\xi) = k \cdot \xi^2 + c$$

  其中 $\xi$ 是反应坐标,$k$ 和 $c$ 是拟合系数。

  % 间距
  \iqbsep

  % 第二个公式块
  \textbf{扩展阶段的张力贡献:}
  \iqbtinysep

  $$\gamma(r) = 2\pi r \sigma$$

  其中 $\sigma$ 是线张力 (line tension),$r$ 是孔半径。

  % 关键物理量
  \iqbsep

  \begin{block}{关键物理量}
    线张力 $\sigma \approx 0.1$ 到 $1.0 \, \text{pN}$,
    决定了孔稳定性和形成自由能。
  \end{block}
\end{frame}

% ============================================================
% 致谢页 (Thank You) - 无 header/footer
% ============================================================
\begin{frame}[plain,noframenumbering]
  \vfill
  \centering

  {\Huge \textcolor{iqbblue}{\textbf{Thank You!}}}

  \vspace{1.5cm}

  {\Large Questions?}

  \vspace{2cm}

  % 可选:添加联系信息
  % {\normalsize
  % Contact: yourname@iqblab.edu\\
  % IQB Lab, 2025
  % }

  \vfill
\end{frame}

% ============================================================
% 完整命令参考手册 (Complete Command Reference)
% ============================================================
%
% 本模板提供了丰富的布局命令,以下是详细的使用说明。
%
% ============================================================
% 1. 布局命令 (Layout Commands)
% ============================================================
%
% 【双列布局 - 50% 分割】
% \iqblayouttwo{左列内容}{右列内容}
% 用途: 对比、平行、左右对称
% 示例:
%   \begin{frame}{方法对比}
%     \iqblayouttwo{
%       方法A描述
%     }{
%       方法B描述
%     }
%   \end{frame}
%
% 【双列布局 - 1/3:2/3 分割(图在左)】
% \iqblayoutonethird{左侧(窄)}{右侧(宽)}
% 用途: 左侧是图或装置,右侧是详细说明
% 建议: 左侧图宽度约 width=0.25-0.30\textwidth
%
% 【双列布局 - 2/3:1/3 分割(图在右)】
% \iqblayouttwothirds{左侧(宽)}{右侧(窄)}
% 用途: 左侧是详细说明,右侧是图
%
% 【三列均分布局】
% \iqblayoutthree{列1}{列2}{列3}
% 用途: 三方对比、多个方法、结果展示
% 注意: 字号会自动缩小,内容不要太多
% 建议每列图片高度: height=0.35-0.40\textheight
%
% 【2×2 网格】
% \iqbgridtwobytwo{图1}{图2}{图3}{图4}
% 用途: 展示 4 个相关的图或数据
% 排列: 上左 | 上右
%       下左 | 下右
% 建议: 每个图加标号 (A), (B), (C), (D)
%
% ============================================================
% 2. 图片文字混合布局 (Image-Text Combinations)
% ============================================================
%
% 【图片左,文字右】
% \iqbimagetext[width=0.4\textwidth]{image.png}{右侧文字}
% 参数说明:
%   width: 图片宽度,通常 0.35-0.45\textwidth
%   image.png: 图片文件路径
%   右侧文字: 可包含列表、段落、公式等
%
% 【文字左,图片右】
% \iqbtextimage[width=0.4\textwidth]{左侧文字}{image.png}
% 参数说明: 同上,但左右位置互换
%
% ============================================================
% 3. 间距控制 (Spacing Control)
% ============================================================
%
% 为保持一致性,请使用以下统一的间距命令,而不是手动 \vspace{}
%
% \iqbtinysep       - 很小间距 (~0.15cm)
%   用途: 列表项之间、紧凑的 block 间距
%
% \iqbsep           - 标准间距 (~0.3cm)  [最常用]
%   用途: 段落间、block 间、内容块分隔
%
% \iqbbigsep        - 大间距 (~0.5cm)
%   用途: 页面主要内容块分隔、大的逻辑段落间
%
% 示例:
%   \begin{itemize}
%     \item 要点 1
%     \item 要点 2
%   \end{itemize}
%   \iqbsep
%   \begin{block}{重要信息}
%     ...
%   \end{block}
%
% ============================================================
% 4. 页面选项 (Frame Options)
% ============================================================
%
% 【无页眉页脚的页面】
% \begin{frame}[plain,noframenumbering]
%   内容(通常用于封面、致谢页)
% \end{frame}
%
% 选项说明:
%   plain: 不显示 header, footer, navigation symbols
%   noframenumbering: 不计入总页数
%
% 【设置 footer 中间显示的 section 名称】
% \setsection{Section Name}
% 用途: 在 footer 中间显示当前 section,帮助观众理解进度
% 建议: 每进入新的主要内容区域时调用一次
%
% ============================================================
% 5. 颜色使用 (Color Usage)
% ============================================================
%
% 【文字着色】
% \textcolor{iqbblue}{蓝色文字}
% \textcolor{iqborange}{橙色强调}
% \textcolor{iqbgreen}{绿色标签}
% \textcolor{iqbred}{红色警告}
%
% 【背景着色 - 用于 block 或强调】
% \begin{block}{标题}
%   内容会自动使用 IQB 主题色
% \end{block}
%
% ============================================================
% 6. 图片尺寸参考 (Image Sizing Reference)
% ============================================================
%
% 纸张可用高度约 0.65 \textheight(去掉 header/footer)
%
% 单列图片:
%   - 单独占一行: height=0.5-0.6\textheight
%   - 配合文字: height=0.4-0.5\textheight
%
% 双列图片 (50-50):
%   - 每列: height=0.45-0.55\textheight
%
% 三列图片:
%   - 每列: height=0.35-0.40\textheight
%
% 2×2 网格:
%   - 通常固定: width=0.45\textwidth (自动高度)
%
% ============================================================
% 7. 最佳实践总结 (Best Practices)
% ============================================================
%
% 【页面设计】
% - 避免文字过多,每页最多 8-10 行有效内容
% - 每页尽量包含图表,图文结合效果最佳
% - 竖版高图必须与文字并排,不要堆在下方
%
% 【文字书写】
% - 使用短句而非长段落
% - 列表项不超过 50 个汉字
% - 标题要体现该页的核心洞察,而不仅是主题
%
% 【图片使用】
% - 每个图都要有图注(caption)
% - 标号使用 (A), (B), (C)... 或 ①②③...
% - 避免图片过小,要清晰可见
%
% 【公式排版】
% - 重要公式用 $$ $$ (displayed)
% - 公式前后用 \iqbsep 分隔
% - 化学式用 \ce{} 格式(需要 mhchem 包)
%
% 【编译排查】
% - 如出现 "Overfull \hbox", 说明某行超宽
% - 如出现 "Overfull \vbox", 说明内容超高,需要减少或拆页
% - 增加 --file-line-error 选项获得详细错误位置
%
% ============================================================
% 更多帮助和示例请参考:
% - software-copyright/3-usage.tex (详细命令文档)
% - examples/membrane-pore-jc.tex (完整工作示例)
% ============================================================

\end{document}
