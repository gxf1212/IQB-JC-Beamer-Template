% ============================================================
% IQB Journal Club: 快捷命令演示 (Phase 1 Advanced Shortcuts Demo)
% 展示如何用新增的高级宏命令简化编写
% ============================================================

\documentclass[aspectratio=169,11pt]{beamer}

% Load IQB theme
\usepackage{../theme/beamerthemeiqb}
\usepackage{../theme/iqb-layouts}

% Set header image path
\renewcommand{\iqbheaderimage}{../theme/images/header.png}

% Additional packages
\usepackage{graphicx}
\usepackage{amsmath}
\usepackage{amssymb}
\usepackage[version=4]{mhchem}
\usepackage{booktabs}

% Chinese support
\usepackage{xeCJK}
\setCJKmainfont{SimSun}

% Metadata
\title{快捷命令演示}
\subtitle{IQB-JC Advanced Shortcuts Demo}
\author{高旭帆}
\institute{IQB Lab}
\date{\today}

% ============================================================
\begin{document}

% ============================================================
% Page 1: Cover Page
% ============================================================
\begin{frame}[plain,noframenumbering]
  \vfill
  \centering

  {\LARGE\textcolor{iqbblue}{\textbf{IQB Journal Club}}}\\[0.3cm]
  {\LARGE\textcolor{iqbblue}{\textbf{快捷命令使用演示}}}

  \vspace{0.8cm}

  {\large Advanced Shortcuts for Efficient Presentations}

  \vspace{1.2cm}

  {\normalsize Phase 1: 高级宏命令简化编写}

  \vfill
\end{frame}

% ============================================================
% 演示1: Section Management (自动section管理)
% ============================================================
\iqbsectionframe{Shortcuts}{快捷命令}

% ============================================================
% Page 3: 演示 - \iqbframe (标准单栏frame)
% ============================================================
\iqbframe{演示1: 标准Frame快捷命令}{
  \textbf{核心思想}:用简洁的一行命令替代复杂的环境嵌套

  \medskip

  \textbf{优势}:
  \iqbitemize{
    \item 代码从3行减少到1行(减少67\%)
    \item 无需手动写 \texttt{begin/end} 环境
    \item 自动继承主题设置
  }

  \medskip

  \textbf{代码对比}:

  改进前:3行代码 + 括号嵌套

  \vspace{0.1cm}

  改进后:\textcolor{iqbblue}{\textbf{1行代码}}

  \medskip

  \small
  \textit{用法}:参见文档或使用 VSCode snippet \texttt{iqbf}
}

% ============================================================
% Page 4: 演示 - \iqbframetextfig (文字+图片frame)
% ============================================================
\iqbframetextfig{演示2: 文字+图片Frame快捷}{
  \textbf{命令}: \texttt{\textbackslash iqbframetextfig\{...\}\{...\}\{...\}}

  \medskip

  \textbf{参数}:
  \iqbenumerate{
    \item 可选: 图片高度 [0.55\textbackslash textheight]
    \item 必需: 标题
    \item 必需: 左侧文字内容
    \item 必需: 右侧图片路径
  }

  \medskip

  \textbf{优势}:
  \iqbitemize{
    \item 自动应用1/3-2/3布局
    \item 无需手动嵌套\texttt{\textbackslash iqblayoutonethird}
    \item 图片自动居中
  }
}{images/membrane-pore-jc/fig1.png}

% ============================================================
% Page 5: 演示 - 公式Frame快捷 (改进版本)
% ============================================================
\iqbformulaframe{演示3: 公式+解释Frame快捷}{
  \iqbformblock{新命令 $\text{CV}_{\text{new}}$}{
    $$\text{CV}_{\text{new}} = \text{CV}_{\text{cyl}} \times s_1 + \text{CV}_{\text{radius}} \times s_2$$
  }{自动进行成核与扩展的切换}

  \iqbformblock{传统方法 $\text{CV}_{\text{old}}$}{
    $$\text{CV}_{\text{old}} = \text{CV}_1 \text{ or } \text{CV}_2$$
  }{需要分别计算,无法统一}
}{
  \textbf{优点对比}:
  \iqbitemize{
    \item \textcolor{green!60!black}{无滞后现象}
    \item \textcolor{green!60!black}{可逆性好}
    \item \textcolor{green!60!black}{平滑过渡}
  }
}

% ============================================================
% Page 6: 演示 - 三列对比
% ============================================================
\begin{frame}{演示4: 三列对比快捷命令}
  \iqbthreecolcompare
    {方法1}{images/membrane-pore-jc/fig1u.png}{\item 优点A \item 优点B}
    {方法2}{images/membrane-pore-jc/fig1d.png}{\item 优点C \item 优点D}
    {方法3}{images/membrane-pore-jc/plumed.png}{PLUMED库\\兼容性强}
\end{frame}

% ============================================================
% Page 7: 演示 - 结果Frame
% ============================================================
\iqbresultframe{演示5: 结果展示Frame快捷}{
  \textbf{关键发现}:
  \iqbitemize{
    \item 脂质尾部密度与孔寿命正相关 (R²=0.82)
    \item Full-Path与Rapid方法高度一致
    \item CHARMM36精准描述离子效应
  }

  \medskip

  \small
  \textbf{物理机制}:
  Na$^+$与阴离子脂质头部相互作用,导致膜线张力改变
}{images/membrane-pore-jc/fig7.png}

% ============================================================
% Page 8: 演示 - 对比Frame
% ============================================================
\iqbcomparisonframe{演示6: 对比结果Frame快捷}{
  images/membrane-pore-jc/fig6.png}{Full-Path CV:完整过程}{
  images/membrane-pore-jc/fig5c.png}{Rapid CV:快速评估}

% ============================================================
% Page 9: 代码行数对比统计
% ============================================================
\begin{frame}{代码改进统计}
  \begin{table}[htbp]
    \small
    \centering
    \begin{tabular}{lcccc}
      \toprule
      \textbf{Frame类型} & \textbf{改进前} & \textbf{改进后} & \textbf{减少} \\
      \midrule
      标准Frame & 3行 & 1行 & -67\% \\
      文字+图片 & 6行 & 1行 & -83\% \\
      公式+解释 & 40行 & 12行 & -70\% \\
      三列对比 & 48行 & 8行 & -83\% \\
      结果展示 & 15行 & 3行 & -80\% \\
      对比展示 & 10行 & 1行 & -90\% \\
      \midrule
      \textbf{平均} & - & - & \textbf{-79\%} \\
      \bottomrule
    \end{tabular}
  \end{table}

  \medskip

  \textbf{总体效果}:典型14页幻灯片从270行代码减少至\textcolor{iqbblue}{\textbf{80行}}
\end{frame}

% ============================================================
% Page 10: VSCode Snippets使用
% ============================================================
\iqbframe{VSCode Snippets快捷方式}{
  \textbf{配置位置}: \texttt{.vscode/latex.code-snippets}

  \medskip

  \textbf{快捷方式列表}:
  \begin{table}[htbp]
    \tiny
    \begin{tabular}{ll}
      \texttt{iqbf} & 标准frame \\
      \texttt{iqbtf} & 文字+图片 \\
      \texttt{iqbfit} & 图片+文字 \\
      \texttt{iqbff} & 公式+解释 \\
      \texttt{iqbsec} & Section分隔 \\
      \texttt{iqb3c} & 三列对比 \\
      \texttt{iqb2c} & 双列对比 \\
      \texttt{iqbrf} & 结果展示 \\
      \texttt{iqbcf} & 对比结果 \\
    \end{tabular}
  \end{table}

  \medskip

  \textbf{使用方法}: 在 .tex 文件中输入快捷方式 (如 \texttt{iqbtf}) 后按 Tab 键自动展开
}

% ============================================================
% Page 11: 使用建议
% ============================================================
\iqbblock{使用建议}{
  \small
  \iqbitemize{
    \item \textbf{日常编写}: 优先使用快捷命令,保持代码简洁
    \item \textbf{特殊需求}: 支持在快捷命令基础上手动调整参数
    \item \textbf{可选参数}: 大多命令支持 \texttt{[height=...]} 自定义高度
    \item \textbf{兼容性}: 新命令与旧代码完全兼容,可逐步迁移
    \item \textbf{IDE配置}: 推荐使用 VSCode + Snippets 实现最佳体验
  }
}

% ============================================================
% Page 12: Thank You
% ============================================================
\begin{frame}[plain,noframenumbering]
  \vfill
  \centering

  {\Huge \textcolor{iqbblue}{\textbf{感谢!}}}

  \vspace{1.5cm}

  {\large 快捷命令已集成到 iqb-layouts.sty}

  \vspace{1cm}

  {\normalsize \texttt{.vscode/latex.code-snippets} 已配置}

  \vfill
\end{frame}

\end{document}
