% ============================================================
% IQB Journal Club: Membrane Pore Free Energy
% 膜孔自由能与稳定性的分子动力学模拟研究
% Date: 2025-10-20
% Total Pages: 14
% ============================================================

\documentclass[aspectratio=169,11pt]{beamer}

% Load IQB theme
\usepackage{../theme/beamerthemeiqb}
\usepackage{../theme/iqb-layouts}

% Set header image path (relative to this .tex file)
\renewcommand{\iqbheaderimage}{../theme/images/header.png}

% Additional packages
\usepackage{graphicx}
\usepackage{amsmath}
\usepackage{amssymb}
\usepackage[version=4]{mhchem}  % For chemical formulas (\ce command)
\usepackage{booktabs}
\usepackage{hyperref}
\hypersetup{
  colorlinks=true,
  linkcolor=blue,
  urlcolor=blue,
  citecolor=blue
}

% Chinese support (XeLaTeX required)
\usepackage{xeCJK}
\setCJKmainfont{SimSun}  % Windows: SimSun, Linux: Noto Sans CJK SC

% Metadata
\title{Free Energy of Membrane Pore Formation and Stability}
\subtitle{from Molecular Dynamics Simulations}
\author{高旭帆}
\institute{IQB Lab}
\date{\today}

% ============================================================
% Paper Information (论文配置 - 在此定义,全文自动使用)
% ============================================================
\papertitle{膜孔形成的自由能与稳定性的分子动力学模拟研究}
\paperengtitle{Free Energy of Membrane Pore Formation and Stability from Molecular Dynamics Simulations}
\papercitation{Rivel, T., Biriukov, D., Kabelka, I., \& Vácha, R. (2025). J. Chem. Inf. Model. \textbf{65}, 908–920.}
\paperdoi{10.1021/acs.jcim.4c01960}
\paperjournal{Journal of Chemical Information and Modeling}

% ============================================================
\begin{document}

% ============================================================
% Page 1: Cover Page (无 header/footer)
% ============================================================
\begin{frame}[plain,noframenumbering]
  \vfill
  \centering

  {\LARGE\textcolor{iqbblue}{\textbf{破解膜孔之谜:双CV联手揭示}}}\\[0.3cm]
  {\LARGE\textcolor{iqbblue}{\textbf{从成核到扩展的完整能量图景}}}

  \medskip

  {\large Free Energy of Membrane Pore Formation and Stability\\
  from Molecular Dynamics Simulations}

  \bigskip

  {\normalsize 汇报人:高旭帆}\\[0.2cm]
  {\normalsize IQB Lab}\\[0.2cm]
  {\normalsize \today}

  \vfill
\end{frame}

% ============================================================
% Page 2: Paper Information (通讯作者 + 一作 with photos)
% ============================================================
\iqbsectionframe{Overview}{论文信息}
\begin{frame}{发表论文}
  \setauthorfirstfield{计算生物物理、膜动力学、增强采样方法}
  \iqbauthorstwophoto{images/membrane-pore-jc/corresponding.png}{Robert Vácha}{%
    CEITEC and NCBR, Faculty of Science\\
    Masaryk University, Brno, Czech Republic
  }{%
    \href{https://vacha.ceitec.cz}{vacha.ceitec.cz} | \href{https://scholar.google.com/citations?user=NEt2O0MAAAAJ}{Google Scholar}
  }{%
    膜蛋白相互作用、抗菌肽、多尺度分子模拟
  }{images/membrane-pore-jc/first_author.png}{Timothée Rivel}{%
    CTO at InSiliBio\\
    Université de Franche-Comté, France
  }{%
    \href{https://www.insilibio.com}{www.insilibio.com}
  }

  \smallskip

  \footnotesize
  \textbf{发表}: \textit{J. Chem. Inf. Model.} \textbf{65}, 908–920 (2025) | \textbf{DOI}: 10.1021/acs.jcim.4c01960
\end{frame}

% ============================================================
% Page 3: Background
% ============================================================
\iqbsectionframe{Background}{背景和意义}
\iqbframetextfig[0.52\textheight]{膜孔形成:从抗菌肽到电穿孔的关键过程}{
  \textbf{膜孔形成的重要性}:
  \iqbitemize{
    \item 抗菌肽设计与抗菌机制
    \item 药物递送系统穿膜
    \item 细胞通透性调控
    \item 电穿孔技术应用
  }

  \medskip

  \textbf{实验方法的局限}:
  \iqbitemize{
    \item 缺乏原子尺度分辨率
    \item 难以捕捉纳秒瞬态
    \item 无法量化自由能学
  }

  \medskip

  {\footnotesize\textbf{图注}:孔闭合的四个阶段。(A)平衡孔,水线贯穿膜;(B)半径缩小,脂质重排;(C)临界瞬间,仅剩水线程;(D)孔闭合,膜恢复。}
}{images/membrane-pore-jc/fig3.png}

% ============================================================
% Page 4: Key Scientific Questions
% ============================================================
\begin{frame}{挑战:统一描述成核与扩展两个截然不同的阶段}
  \iqbblock{核心挑战}{
    如何通过MD模拟准确量化膜孔形成全过程(成核+扩展),\\并可靠预测不同条件下的膜线张力?
  }

  \medskip

  \iqblayouttwo{
    \textbf{现有方法的问题}:
    \iqbenumerate{
      \item 传统CV难以统一描述成核和扩展
      \item 常出现滞后现象和收敛问题
      \item 不同力场预测准确性差异大
      \item 缺乏快速评估膜稳定性的方法
    }
  }{
    \textbf{为什么这很难}:
    \iqbitemize{
      \item 成核:极性重原子数量变化
      \item 扩展:孔半径线性增长
      \item 两者物理机制不同
      \item 需要无缝切换的描述
    }

    \medskip

    \textbf{时间尺度}:\\
    成核 $\sim$ ns,扩展 $\sim$ $\mu$s,自发形成 $\sim$ ms
  }
\end{frame}

% ============================================================
% Page 5: Innovations (三列布局)
% ============================================================
\iqbsectionframe{Methods}{方法}
\begin{frame}{创新:双CV策略破解成核-扩展统一描述难题}
  \iqbthreecolcompare
    {Full-Path CV}{images/membrane-pore-jc/fig1u.png}{%
      \item 成核+扩展统一
      \item 尾部密度驱动
      \item 无滞后、可逆
      \item 平滑切换
    }
    {Rapid CV}{images/membrane-pore-jc/fig1d.png}{%
      \item 无限大孔模拟
      \item 脂质条带构型
      \item 线张力快速评估
      \item 效率提高10×
    }
    {开源实现}{images/membrane-pore-jc/plumed.png}{%
      PLUMED库实现\\
      GROMACS/LAMMPS\\
      全原子/粗粒化\\
      参数可定制
    }
\end{frame}

% ============================================================
% Page 6: Method - Full-Path CV Principle
% ============================================================
\iqbformulaframe{Full-Path:切换函数巧妙结合成核与扩展}{
  \iqbformblock{成核部分 $\text{CV}_{\text{cyl}}$}{
    $$\text{CV}_{\text{cyl}} = 1 - d/\text{CV}_{\text{eq}}$$
  }{$d$: 圆柱内脂质尾部原子数}

  \iqbformblock{扩展部分 $\text{CV}_{\text{radius}}$}{
    $$\text{CV}_{\text{radius}} = r_{\text{min}}/r_{\text{unit}}$$
  }{$r_{\text{min}}$: 孔中心到最近脂质距离}

  \iqbformblock{联合 CV}{
    $$\text{CV} = \text{CV}_{\text{cyl}} s_1 + \text{CV}_{\text{r}} s_2$$
  }{权重函数平滑切换}
}{
  \centering
  \includegraphics[height=0.45\textheight]{images/membrane-pore-jc/fig1.png}

  \tinysep
  {\footnotesize\raggedright\setlength{\rightskip}{0pt plus 1cm}
  \textbf{图1}:两种创新的集体变量设计。\textbf{上半部}:Full-Path CV同时追踪成核(圆柱内尾部密度CV$_{\text{cyl}}$)和扩展(孔半径CV$_{\text{r}}$);通过sigmoid切换函数平滑过渡。\textbf{下半部}:Rapid CV使用脂质条带模拟无限孔,通过调节盒子尺寸快速提取线张力(因子2考虑两孔边缘)。
  }
}

% ============================================================
% Page 7: Method - Switching Functions
% ============================================================
\begin{frame}{平滑过渡:sigmoid函数消除成核-扩展边界}
  \iqblayouttwo{
    \iqbformblock{切换函数 $s_1, s_2$}{
      $$s_1 = \frac{1}{1 + e^{\alpha(\text{CV}_r - \text{CV}_0)}}$$
      $$s_2 = \frac{1}{1 + e^{-\alpha(\text{CV}_r - \text{CV}_0)}}$$
    }{}

    \iqbformblock{优化参数}{
      $\alpha = 20$ (陡峭)\\
      $\text{CV}_0 = 0.95$ (切换点)
    }{CV < 0.95: 主导 $\text{CV}_{\text{cyl}}$ | CV > 0.95: 主导 $\text{CV}_{\text{radius}}$}

    \textbf{关键优势}:
    \iqbitemize{
      \item 平滑过渡,无不连续性
      \item 避免数值问题
      \item 单一CV占主导
    }
  }{
    \centering
    \includegraphics[width=0.9\textwidth]{images/membrane-pore-jc/switching-functions.png}

    \medskip
    {\footnotesize\raggedright
    \textbf{图2}:Sigmoid切换函数实现平滑过渡。两条曲线在 $\text{CV}_0 = 0.95$ 处相交,权重各为 0.5。\textbf{蓝线}($s_1$):CV < 0.95时主导成核,权重从1递减。\textbf{红线}($s_2$):CV > 0.95时主导扩展,权重从0递增。参数 $\alpha=20$确保陡峭过渡,避免不连续性。物理效应:单一CV占主导,无数值问题,与实验孔寿命趋势一致。
    }
  }
\end{frame}

% ============================================================
% Page 8: Validation - Pore State Definition
% ============================================================
\iqbsectionframe{Results}{结果}
\begin{frame}{孔状态定义:17层切片追踪跨膜水通道形成}
  \iqblayouttwothirds{
    {\footnotesize
    \textbf{切片划分方法}:
    \iqbitemize{
      \item 膜沿z轴分17个切片
      \item 每片厚0.25 nm
      \item 范围:[-2.125, 2.125] nm
      \item 高斯权重平滑计数
    }

    \medskip

    \textbf{孔状态$s(t)$计算}:
    \iqbenumerate{
      \item $s_i(t)$:高斯平滑水原子计数
      \item $\mathcal{H}(s_i - 1)$:Heaviside判定
      \item $s(t) = \frac{1}{17}\sum \mathcal{H}(s_i - 1)$
    }

    \medskip

    \textbf{物理意义}:
    \iqbitemize{
      \item $s=1$:完全开放跨膜孔
      \item $s=0$:完整无孔膜
      \item $0<s<1$:孔形成/闭合过程
    }

    \medskip

    \textbf{孔寿命趋势}:DMPC $>$ DPPC $>$ POPC $>$ DOPC
    }
  }{
    \centering
    \includegraphics[height=0.65\textheight,keepaspectratio]{images/membrane-pore-jc/fig2.png}

    \tinysep
    {\tiny\raggedright\setlength{\rightskip}{0pt plus 1cm}
    \textbf{图3}:孔状态追踪及闭合动力学。(A)蓝色=含水通道。(B)4种磷脂闭合过程,饱和度高→寿命长。
    }
  }
\end{frame}

% ============================================================
% Page 9: Validation - Pore Closure Snapshots
% ============================================================
\begin{frame}{孔闭合过程:尾部碳密度决定膜缺陷稳定性}
  \iqblayouttwothirds{
    {\footnotesize
    \textbf{四阶段闭合动力学}:
    \iqbenumerate{
      \item \textbf{(A) 平衡孔}:初始稳定孔,连续水柱贯穿膜厚度
      \item \textbf{(B) 半径缩小}:孔边缘脂质重排,整体结构保持不变
      \item \textbf{(C) 水线程}:关键瞬间,孔只有纤细连续水线相连
      \item \textbf{(D) 膜恢复}:孔完全闭合,局部膜厚度恢复
    }

    \medskip

    \textbf{临界发现}:
    {\color{iqbblue}\textbf{脂质尾部密度}} $\leftrightarrow$ {\color{iqbblue}\textbf{孔寿命 $\tau$}} (R²=0.82)

    \textit{相关性极强!}

    \tinysep

    \textbf{设计启示}:
    这一强相关性启发了\textbf{尾部密度CV}($\text{CV}_{\text{cyl}}$)的创新设计,追踪圆柱体内脂质尾部原子数。

    \medskip

    \textbf{力场表现} (孔寿命/ns):

    DMPC/DPPC/POPC/DOPC

    \textit{C36}: 122/94/34/15

    \textit{Slipids}: 110/32/27/18
    }
  }{
    \centering
    \includegraphics[height=0.68\textheight,keepaspectratio]{images/membrane-pore-jc/fig3.png}

    \tinysep
    {\footnotesize\raggedright\setlength{\rightskip}{0pt plus 1cm}
    \textbf{图4}:孔闭合快照。左侧四个阶段清晰展示孔半径缩小、水线程形成、最终闭合的完整过程。右侧四种膜的孔闭合动力学曲线,展示饱和度差异导致的孔寿命差异。}
  }
\end{frame}

% ============================================================
% Page 10: Results - Full-Path Free Energy Profile
% ============================================================
\iqbformulaframe{Full-Path结果:正反向拉伸完全重合,CV设计可逆无滞后}{
  \iqbformblock{成核阶段 ($\text{CV} < 0.5$)}{
    $$\Delta G = k \cdot \text{CV}^2 + c$$
  }{二次增长,$k$ 表征成核能垒}

  \iqbformblock{扩展阶段 ($\text{CV} > 1.2$)}{
    $$G(r) = 2\pi r\gamma$$
  }{线性增长,斜率给出线张力}

  \iqbformblock{关键结果}{
    \textit{CHARMM36}: $k=72.9$ kJ/mol, $\gamma=32.5$ pN\\
    \textit{Martini 2.2p}: $k=73.7$ kJ/mol, $\gamma=49.2$ pN
  }{{\color{iqbblue}\textbf{无滞后}}:正反向完全重合}
}{
  \centering
  \includegraphics[height=0.58\textheight]{images/membrane-pore-jc/fig4.png}

  \tinysep
  {\footnotesize\raggedright\setlength{\rightskip}{0pt plus 1cm}
  \textbf{图5}:Full-Path自由能分析。\textbf{(A)}典型孔结构的侧视图和俯视图。\textbf{(B)}自由能剖面展示两阶段:成核(CV<0.5)二次增长$\Delta G=k \cdot \text{CV}^2$($k$=72.9 kJ/mol);扩展(CV>1.2)线性增长$G=2\pi r\gamma$($\gamma$=32.5 pN)。正反向完全重合证明无滞后。\textbf{(C)}不同力场的线张力比较,CHARMM36和prosECCo75最准确。
  }
}

% ============================================================
% Page 11: Results - Rapid Method
% ============================================================
\begin{frame}{Rapid方法:脂质条带模拟"无限孔"快速提取线张力}
  \iqblayouttwothirds{
    {\footnotesize
    \textbf{创新思路}:
    \iqbitemize{
      \item ① 沿膜平面拉伸模拟盒子
      \item ② 形成脂质条带结构
      \item ③ 周期边界条件连接
      \item ④ 形成环形"无限孔"
    }

    \medskip

    \textbf{线张力原理}:

    大孔自由能:$\Delta G = 2L\gamma$

    其中$L$为孔边缘长度,因子2考虑两个孔边缘。通过线性拟合斜率直接提取$\gamma$。

    \medskip

    \textbf{计算效率对比}:
    \iqbitemize{
      \item Full-Path:65窗 $\times$ 200ns(全原子)
      \item Rapid:仅21窗 $\times$ 150ns,效率提升$\sim$6倍
      \item 准确性:与Full-Path结果高度一致
    }

    \medskip

    \textbf{关键数据}:

    POPC  \textit{C36}: $\gamma=34.1$ pN

    POPC  \textit{M2.2p}: $\gamma=49.2$ pN
    }
  }{
    \centering
    \includegraphics[height=0.65\textheight,keepaspectratio]{images/membrane-pore-jc/fig5ab.png}

    \tinysep
    {\footnotesize\raggedright\setlength{\rightskip}{0pt plus 1cm}
    \textbf{图6}:Rapid方法示意。(A)脂质条带侧视图,两个孔边缘(膜外界面)中间夹水。(B)俯视图,通过PBC形成环形无限孔。(C)自由能随盒子尺寸线性增长,斜率为$2\pi\gamma$。
    }
  }
\end{frame}

% ============================================================
% Page 12: Results - Force Field Comparison
% ============================================================
\iqbframetextfig[0.53\textheight]{力场筛选:仅CHARMM36和prosECCo75准确复现实验趋势}{
  \footnotesize
  \textbf{实验趋势}:\\
  POPE $>$ POPS $>$ POPC $>$ POPG

  \medskip

  \textbf{力场表现}:
  \iqbitemize{
    \item {\color{green!60!black}\textbf{C36/prosECCo75}}:准确(NMR调参精细)
    \item {\color{orange}\textbf{Slipids/M2.2p}}:部分正确
    \item {\color{red}\textbf{M2.2/M3}}:失败(粗粒化失效)
  }

  \medskip

  \textbf{核心差异}:阴离子脂质头部与 $\ce{Na+}$/$\ce{Ca^{2+}}$ 结合的描述精度

  \medskip

  \small
  结论:选择头部-离子作用精确的力场至关重要

  \medskip

  {\footnotesize
  \textbf{图注}:力场性能对标。实验趋势POPE>POPS>POPC>POPG。绿色(C36/p75)准确,橙色(Slipids/M2.2p)部分正确,红色(M2.2/M3)完全失败。差异根源:阴离子脂质与Na$^+$/Ca$^{2+}$结合的参数化精度。}
}{images/membrane-pore-jc/fig5c.png}

% ============================================================
% Page 13: Results - Cross-Validation
% ============================================================
\begin{frame}{双重验证:Full-Path与Rapid高度一致,与实验定性吻合}
  \iqblayouttwothirds{
    {\footnotesize
    \textbf{(A) 与实验定性对标}:

    {\color{iqbblue}\textbf{✓ 定性吻合}}:
    \iqbitemize{
      \item POPC:POPG膜
      \item PG含量↑ $\Rightarrow$ $\gamma$↓
      \item 添加Ca$^{2+}$ $\Rightarrow$ $\gamma$恢复
    }

    {\color{orange}\textbf{⚠ 定量偏差}}:
    \iqbitemize{
      \item 实验Ca$^{2+}$效应:~40 pN
      \item 模拟Ca$^{2+}$效应:~6 pN
      \item 原因:离子参数化精度、体系差异
    }

    \medskip

    \textbf{(B) 两种方法互补验证}:
    \iqbitemize{
      \item Full-Path vs Rapid相关性极高
      \item C36 Rapid偏低~2 pN
      \item M2.2p Rapid偏高~3-5 pN
      \item 原因:孔几何差异(圆柱vs环形)
    }

    \medskip

    {\color{iqbblue}\textbf{★ 结论}}:双CV互补,提高可信度
    }
  }{
    \centering
    \includegraphics[height=0.68\textheight,keepaspectratio]{images/membrane-pore-jc/fig6.png}

    \tinysep
    {\footnotesize\raggedright\setlength{\rightskip}{0pt plus 1cm}
    \textbf{图7}:交叉验证。(A)与Lira等文献实验对比,模拟准确捕捉PG含量和离子效应的定性趋势。(B)两种CV的线张力对比,R²极高,证明方法一致性。
    }
  }
\end{frame}

% ============================================================
% Page 14: Results - Ion Effects
% ============================================================
\begin{frame}{离子效应:Ca$^{2+}$恢复线张力,Na$^+$增强阴离子膜效应}
  \iqblayouttwothirds{
    {\footnotesize
    \textbf{(A) Ca$^{2+}$恢复膜稳定性}:

    POPC:POPG体系:
    \iqbitemize{
      \item 仅加NaCl:$\gamma$ ↓ 到29 pN
      \item 再加CaCl$_2$:$\gamma$ ↑ 恢复至35.5 pN
      \item 机制:Ca$^{2+}$与PG头部强结合,静电屏蔽
    }

    \medskip

    \textbf{(B) Na$^+$增强阴离子膜效应}:

    对比有/无0.15M NaCl:
    \iqbitemize{
      \item POPG/POPS:$\Delta\gamma >$ 0(↑30-50\%)
      \item POPC/POPE:$\Delta\gamma \approx$ 0(无显著变化)
      \item 机制:Na$^+$屏蔽阴离子头部静电斥力
    }

    疏水代价增加 $\Rightarrow$ 线张力升高

    \medskip

    \textbf{力场性能评分}:

    {\color{green!60!black}✓ C36}:精确

    {\color{orange}△ prosECCo75}:偏小

    {\color{red}✗ M2.2p}:低估
    }
  }{
    \centering
    \includegraphics[height=0.68\textheight,keepaspectratio]{images/membrane-pore-jc/fig7.png}

    \tinysep
    {\footnotesize\raggedright\setlength{\rightskip}{0pt plus 1cm}
    \textbf{图8}:离子效应定量分析。(A)Ca$^{2+}$恢复POPC:POPG膜的线张力。(B)Na$^+$对不同脂质的$\Delta\gamma$影响,清晰展示阴离子脂质特异性增强效应。力场准确性直接取决于离子-脂质头部相互作用的参数化。
    }
  }
\end{frame}

% ============================================================
% Page 14: Inspirations (重点页)
% ============================================================
\setsection{Discussion}
\begin{frame}{Inspirations: 借鉴意义}
  \begin{block}{对 HA/OP 膜相互作用研究的启发}
    \centering
    \footnotesize
    \textbf{核心洞察}:OP 降低膜线张力 $\to$ 促进膜孔形成/稳定 $\to$ 可能的透皮机制
  \end{block}

  \vspace{0.25cm}

  \iqblayouttwo{
    \textbf{可借鉴的方法}:

    \tinysep

    \footnotesize
    \textbf{1. 线张力计算}\\
    $\bullet$ Full-Path CV: 成核+扩展全过程\\
    $\bullet$ Rapid CV: 快速评估大孔极限\\
    $\bullet$ 伞形采样 + WHAM 自由能

    \medskip

    \textbf{2. 膜性质表征}\\
    $\bullet$ Lindemann 指数:脂质流动性\\
    $\bullet$ Bond-orientational order:相态\\
    $\bullet$ Area per lipid:膜面积变化

    \medskip

    \textbf{3. 力场选择}\\
    $\bullet$ CHARMM36: 准确离子-脂质作用\\
    $\bullet$ Martini 3: 快速筛选\\
    $\bullet$ 全原子验证关键发现
  }{
    \textbf{与我们研究的关联}:

    \tinysep

    \footnotesize
    \textbf{实验观察}:\\
    {\color{iqbblue}$\bullet$} OP仅与FFA结合,不与CER/CHOL结合\\
    {\color{iqbblue}$\bullet$} pH<5时OP质子化带正电\\
    {\color{iqbblue}$\bullet$} 200nm纳米凝胶穿过20nm通道

    \medskip

    \textbf{假说}:\\
    {\color{orange}$\bullet$} OP形成"局部阳离子补丁"\\
    {\color{orange}$\bullet$} 特异性结合FFA羧基(-COO$^-$)\\
    {\color{orange}$\bullet$} 降低膜线张力,诱导瞬时孔道\\
    {\color{orange}$\bullet$} 纳米凝胶通过形变挤压穿过

    \medskip

    \textbf{下一步}:\\
    $\Rightarrow$ 定量计算OP-FFA膜的线张力\\
    $\Rightarrow$ 对比纯SC膜 vs OP结合后的$\gamma$\\
    $\Rightarrow$ 验证线张力降低假说
  }
\end{frame}

% ============================================================
% Page 15: Thank You (致谢页 - 使用固定样式)
% ============================================================
\iqbthankyouframe

\end{document}
