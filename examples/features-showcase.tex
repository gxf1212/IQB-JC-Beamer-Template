% ============================================================
% IQB Beamer Template - Features Showcase
% 功能全展示模板 - 演示所有可用的布局命令和视觉增强功能
%
% 版本: 1.1 | 创建日期: 2025-10-23
%
% 用途:
%   1. 作为模板功能的参考手册
%   2. 测试新功能的兼容性
%   3. 为用户展示所有可用的排版选项
%
% 编译命令:
%   Linux/Mac:   xelatex -interaction=nonstopmode features-showcase.tex
%   Windows WSL: /mnt/d/texlive/2022/bin/win32/xelatex.exe features-showcase.tex
% ============================================================

\documentclass[aspectratio=169,11pt]{beamer}

% Load IQB theme
\usepackage{../theme/beamerthemeiqb}
\usepackage{../theme/iqb-layouts}

% Set header image path
\renewcommand{\iqbheaderimage}{../theme/images/header.png}

% Additional packages
\usepackage{graphicx}
\usepackage{amsmath}
\usepackage{amssymb}
\usepackage{booktabs}
\usepackage{tikz}
\usetikzlibrary{shapes.geometric, arrows.meta, positioning, backgrounds}
\usepackage{listings}  % 代码高亮支持

% Bibliography support (用于演示footer引用系统)
\usepackage[style=authoryear,maxbibnames=3,maxcitenames=2]{biblatex}
\addbibresource{references.bib}

% Chinese support (XeLaTeX required)
\usepackage{xeCJK}
\usepackage{ifplatform}
\ifwindows
  \setCJKmainfont{SimSun}
\else
  \ifmacosx
    \setCJKmainfont{Songti SC}
  \else
    \setCJKmainfont{Noto Serif CJK SC}
  \fi
\fi

% Metadata
\title{IQB Beamer Template Features Showcase}
\subtitle{功能展示与最佳实践}
\author{Xufan Gao}
\institute{Institute of Quantitative Biology}
\date{\today}

% Footer customization
\setiqbinstitute{Xufan Gao}

% Paper metadata (示例)
\papertitle{IQB Beamer 模板功能展示}
\paperengtitle{IQB Beamer Template Features Showcase}
\papertitlechn{学术演示利器}
\papertitlechnsub{灵活布局,清晰呈现}
\paperdoi{10.5281/zenodo.0000000}

% ============================================================
\begin{document}

% ============================================================
% Cover Page
% ============================================================
\iqbcoverframe

% ============================================================
% Table of Contents
% ============================================================
\begin{frame}{目录}
  \tableofcontents
\end{frame}

% ============================================================
% Section 1: Layout System
% ============================================================
\iqbsectionframe{Layouts}{布局系统}

\begin{frame}{布局概览:五种核心布局模式}
  本模板提供灵活的布局系统,覆盖学术演示的所有场景:


  \iqbitemize{
    \item \textbf{双列布局}:50-50、1/3-2/3、2/3-1/3、自定义比例
    \item \textbf{三列布局}:均分33-33-33,适合三方对比
    \item \textbf{网格布局}:2×2、3×2,适合多图展示
    \item \textbf{单列布局}:标准文字页、纯图页
    \item \textbf{公式布局}:公式+解释,左右分栏
  }


  \iqbbluebox{设计理念}{%
    所有布局命令遵循"简洁优先"原则,一行代码即可创建复杂布局
  }
\end{frame}

\begin{frame}{双列布局 (50-50):对比分析}
  \iqblayouttwo{%
    % 左列
    \textbf{传统方法}
    \iqbitemize

    运行时间: $\sim$10秒
  }{%
    % 右列
    \textbf{我们的方法}
    \iqbitemize

    运行时间: $\sim$12秒
  }


  命令:\texttt{\\iqblayouttwo\{左列\}\{右列\}}
\end{frame}

\begin{frame}{双列布局 (1/3 + 2/3):竖版图展示}
  \iqblayoutonethird{%
    % 左侧窄列(竖版图)
    \centering

    % 这里放竖版图
    \colorbox{iqblightgray}{\parbox{0.8\textwidth}{
      \centering
      \vspace{2cm}
      \textcolor{iqbgray}{竖版图\\height=0.6\textbackslash textheight}
      \vspace{2cm}
    }}
  }{%
    % 右侧宽列(文字说明)
    \textbf{竖版高图的正确用法}:
    \iqbitemize{%
      \item 竖版图(高$>$宽$\times$1.2)必须采用横向column布局
      \item 左侧1列放图(占31\%宽度)
      \item 右侧2列放文字说明(占65\%宽度)
    }


    \textbf{禁止做法}:
    \iqbitemize{%
      \item 禁止:竖版图堆在文字下方(图会过小)
      \item 禁止:强行缩小图片到横版比例(变形)
    }


    命令:\texttt{\\iqblayoutonethird\{图\}\{文字\}}
  }
\end{frame}

\begin{frame}{三列布局:方法对比}
  \iqbthreecolcompare
    {Full-Path CV}{example-image-a}{
      \item 成核+扩展统一
      \item 尾部密度驱动
      \item 无滞后、可逆
      \item 平滑切换
    }
    {Rapid CV}{example-image-b}{
      \item 无限大孔模拟
      \item 脂质条带构型
      \item 线张力快速评估
      \item 效率提高6倍
    }
    {开源实现}{example-image-c}{
      \item PLUMED库实现
      \item GROMACS/LAMMPS
      \item 全原子/粗粒化
      \item 参数可定制
    }
\end{frame}

\begin{frame}{2×2 网格布局:多条件对比}
  \iqbgridtwobytwo{
    % 左上
    \centering
    \textbf{(A) 条件1}

    \colorbox{iqblightgray}{\parbox{0.9\textwidth}{
      \centering
      \vspace{1cm}
      图片A
      \vspace{1cm}
    }}

    参数: $\alpha = 10$
  }{
    % 右上
    \centering
    \textbf{(B) 条件2}

    \colorbox{iqblightgray}{\parbox{0.9\textwidth}{
      \centering
      \vspace{1cm}
      图片B
      \vspace{1cm}
    }}

    参数: $\alpha = 20$
  }{
    % 左下
    \centering
    \textbf{(C) 条件3}

    \colorbox{iqblightgray}{\parbox{0.9\textwidth}{
      \centering
      \vspace{0.3cm}
      图片C
      \vspace{0.3cm}
    }}

    参数: $\alpha = 30$
  }{
    % 右下
    \centering
    \textbf{(D) 条件4}

    \colorbox{iqblightgray}{\parbox{0.9\textwidth}{
      \centering
      \vspace{0.3cm}
      图片D
      \vspace{0.3cm}
    }}

    参数: $\alpha = 40$
  }
\end{frame}

% ============================================================
% Section 2: Visual Enhancement
% ============================================================
\section{Visual Enhancement}
\iqbsectionframe{Visual}{视觉增强}

\begin{frame}{彩色 Punchline 框:引导注意力}
  \iqbbluebox{核心观点(蓝色框)}{%
    用于突出页面的主要发现、核心问题或关键方法。\\
    这是最常用的punchline框,适用于80\%的情况。
  }


  \iqborangebox{挑战/警示(橙色框)}{%
    用于标识需要注意的问题、方法的局限性或尚未解决的挑战。
  }

  \iqbgreenbox{成功验证(绿色框)}{%
    用于展示验证通过、实验成功或正面结果的内容。
  }

  \iqbtinysep

  \footnotesize 命令:\texttt{\\iqbbluebox}, \texttt{\\iqborangebox}, \texttt{\\iqbgreenbox}
\end{frame}

\begin{frame}{数据强调:突出关键数值}
  \textbf{传统写法}(平淡):

  我们的方法线张力为 34.1 pN,相比传统方法的 49.2 pN 降低了 30.7\%。

  \iqbbigsep

  \textbf{使用emphdata(醒目)}:

  我们的方法线张力为 \emphdata{34.1 pN},相比传统方法的 \emphdata{49.2 pN} 降低了 \emphdata{30.7\%}。

  \iqbbigsep

  \textbf{关键发现高亮}:

  \keyfinding{双CV方法预测高度一致,与实验定性吻合 (R²=0.95)}


  命令:\texttt{\\emphdata\{数据\}}、\texttt{\\keyfinding\{结论\}}
\end{frame}

\begin{frame}{字号层级示例}
  {\large \textbf{大号标题 (12pt, large)}}

  用于frametitle或页面内的二级标题


  {\normalsize \textbf{重要正文 (11pt, normalsize)}}

  用于需要强调的段落或关键数据说明


  {\footnotesize 标准正文 (9pt, footnotesize) - 默认}

  这是模板的默认字号,适用于大部分内容、列表和图注


  {\scriptsize 备注与参考 (8pt, scriptsize)}

  用于参考文献、表格注释、技术细节等次要信息

  \iqbbigsep

  \iqborangebox{最佳实践}{%
    避免在同一页混用超过3种字号,保持视觉一致性
  }
\end{frame}

% ============================================================
% Section 3: Advanced Features
% ============================================================
\section{Advanced Features}
\iqbsectionframe{Advanced}{高级功能}

\begin{frame}[fragile]{代码块展示:Python示例}
  \begin{iqbcode}[Python]
import numpy as np
import matplotlib.pyplot as plt

x = np.linspace(0, 2*np.pi, 100)
y = np.sin(x)
plt.plot(x, y, 'b-', label='sin(x)')
plt.legend()
  \end{iqbcode}

  \iqbtinysep

  环境:\texttt{\\begin\{iqbcode\}[Python] ... \\end\{iqbcode\}}
\end{frame}

\begin{frame}{行内代码与代码框}
  在正文中引用函数名:使用 \texttt{np.linspace()} 生成等间距数组。


  命令:使用 \texttt{iqbbluebox} 或 \texttt{iqborangebox} 展示代码片段
\end{frame}

\begin{frame}{TikZ 流程图示例}
  \centering
  \begin{tikzpicture}[
    node distance=0.8cm,
    box/.style={rectangle, draw=iqbblue, fill=iqblightgray, thick, minimum height=1.3cm, minimum width=2.5cm, align=center, font=\small},
    arrow/.style={->, >=stealth, thick, color=iqbblue}
  ]
    \node[box] (step1) {\textbf{数据采集}\\Data Collection};
    \node[box, right=of step1] (step2) {\textbf{预处理}\\Preprocessing};
    \node[box, right=of step2] (step3) {\textbf{模型训练}\\Model Training};
    \node[box, right=of step3] (step4) {\textbf{结果验证}\\Validation};

    \draw[arrow] (step1) -- (step2);
    \draw[arrow] (step2) -- (step3);
    \draw[arrow] (step3) -- (step4);
  \end{tikzpicture}


  使用 TikZ 可创建流程图、决策树、网络图等复杂图形
\end{frame}

\begin{frame}{渐进显示 (\textbackslash pause)}
  第一句话总是显示在屏幕上

  \pause

  点击后第二句出现

  \pause

  再点击第三句出现

  \pause

  \iqbbluebox{适用场景}{%
    逐步揭示推导过程、分步骤展示实验结果、制造悬念引导思考
  }

  \pause

  注意:\textbackslash pause 会增加PDF页数,谨慎使用以避免演示过长
\end{frame}

% ============================================================
% Section 4: Academic Essentials
% ============================================================
\section{Academic Modules}
\iqbsectionframe{Academics}{学术常用模块}

\iqbformulaframe{公式+解释布局(示例1)}{
  \iqbformblock{成核阶段}{
    $$\Delta G_{\text{nuc}}(\xi) = k \cdot \xi^2 + c$$
  }{$\xi$ 是反应坐标,$k$ 和 $c$ 是拟合系数}

  \iqbformblock{扩展阶段}{
    $$\Delta G_{\text{exp}}(r) = 2\pi r \gamma$$
  }{$r$ 是孔半径,$\gamma$ 是线张力}
}{
  % 右侧图片或补充说明
  \centering
  \colorbox{iqblightgray}{\parbox{0.9\textwidth}{
    \centering
    \vspace{0.8cm}
    自由能曲线图\\
    (这里放实际图片)
    \vspace{0.8cm}
  }}
}

\begin{frame}{公式+解释布局(示例2)}
  \iqblayouttwo{%
    \iqbformblock{总自由能}{
      $$\Delta G_{\text{total}} = \Delta G_{\text{nuc}} + \Delta G_{\text{exp}}$$
    }{平滑拼接两个阶段}

    适用于展示复杂公式推导过程
  }{%
    \iqborangebox{布局特点}{%
      左侧:公式+解释\\
      右侧:图片或补充说明\\
      命令:\texttt{iqbformulaframe}
    }
  }
\end{frame}

\begin{frame}{三线表:力场性能对比}
  \iqbthreelinetable{不同力场的线张力预测(单位:pN)}{
    力场 & POPC & POPG & POPS & POPE \\
  }{
    CHARMM36 & 34.1 & 29.3 & 31.8 & 38.2 \\
    Martini 2.2p & 49.2 & 42.5 & 45.1 & 51.3 \\
    prosECCo75 & 35.8 & 30.1 & 33.2 & 39.5 \\
  }

  \keyfinding{CHARMM36 和 prosECCo75 准确复现实验趋势}
\end{frame}

\begin{frame}{作者信息展示}
  \setauthorfirstfield{计算生物物理、膜动力学、增强采样方法}

  \iqbauthorstwophoto{example-image-a}{Robert Vácha}{%
    CEITEC and NCBR, Faculty of Science\\
    Masaryk University, Brno, Czech Republic
  }{%
    \href{https://vacha.ceitec.cz}{vacha.ceitec.cz}
  }{%
    膜蛋白相互作用、抗菌肽、多尺度分子模拟
  }{example-image-b}{Timothée Rivel}{%
    CTO at InSiliBio\\
    Université de Franche-Comté, France
  }{%
    \href{https://www.insilibio.com}{www.insilibio.com}
  }

  \iqbtinysep

  命令:\texttt{\\iqbauthorstwophoto\{...\}}(需9个参数)
\end{frame}

% ============================================================
% Section 5: Best Practices
% ============================================================
\section{Best Practices}
\iqbsectionframe{Guidelines}{最佳实践指南}

\begin{frame}{内容密度控制:6-6-6 规则}
  \iqbbluebox{6-6-6 规则(推荐但非强制)}{%
    \begin{enumerate}
      \item 每页最多 \textbf{6 个 bullet points}
      \item 每个 bullet 最多 \textbf{6 个词}(中文约12字)
      \item 连续文字页不超过 \textbf{6 页}
    \end{enumerate}
  }


  \textbf{拆分建议}:
  \iqbitemize{%
    \item 内容超过10行 → 拆分为2页
    \item 图片+文字溢出 → 调整图片高度或拆页
    \item 编译警告 "Overfull \textbackslash vbox" → 必须减少内容
  }


  \iqborangebox{记住}{%
    你是演讲,幻灯片是辅助。slides应该"无法自解释"。
  }
\end{frame}

\begin{frame}{图片布局原则}
  \textbf{竖版图(高$>$宽$\times$1.2)}:
  \iqbitemize{%
    \item 推荐:使用横向column布局(1列图 + 2列文字)
    \item 推荐:图片高度 \texttt{height=0.5-0.65\\textbackslash textheight}
    \item 禁止:将竖版图堆在文字下方(会过小)
  }


  \textbf{宽图(宽$\geq$高$\times$1.5)}:
  \iqbitemize{%
    \item 推荐:2:1布局(图占2,文字占1)
    \item 推荐:或纯图页(图注在下方)
  }


  \textbf{每张图都必须有图注}:
  \iqbitemize{%
    \item 详细 figure caption,或
    \item 整页文字都围绕图片解读
  }
\end{frame}

\begin{frame}{色彩无障碍建议}
  \iqborangebox{避免红绿配色}{%
    8\%男性 + 0.5\%女性有色盲,红绿是最常见的色盲类型
  }


  \textbf{推荐做法}:
  \iqbitemize{%
    \item 使用蓝-黄配色代替红-绿
    \item 图表中用不同\textbf{形状}辅助颜色(圆形、方形、三角形)
    \item 添加texture patterns(斜线、点阵)
    \item 确保文字与背景对比度 $\geq$ 4.5:1(WCAG标准)
  }


  \iqbgreenbox{本模板的色彩设计}{%
    所有预设颜色(蓝、橙、绿)都是色盲友好的高对比度配色
  }
\end{frame}

\begin{frame}{排版自查清单}
  编译完成后,检查以下事项:

  \iqbenumerate{
    \item 无编译警告(Overfull \textbackslash hbox/\textbackslash vbox)
    \item 行间距舒适(已自动设置为1.5倍)
    \item 字号层级清晰(最多3种字号)
    \item 每页都有图文结合
    \item 图片足够大且清晰
    \item Frametitle包含punchline(而非简单罗列)
    \item Header占满全宽且保持比例
    \item Footer三段式内容完整
  }


  使用 \texttt{pdf-layout-reviewer} agent 自动检查布局问题
\end{frame}

% ============================================================
% Section 6: References & Citations (第6节:引用与学术标识)
% ============================================================
\iqbsectionframe{Citations}{引用与学术标识}

\begin{frame}{Footer引用系统}
  \textbf{使用场景}:在页面底部显示参考文献


  \textbf{可用命令}:
  \begin{itemize}\small
    \item \texttt{\textbackslash iqbfootcite\{key\}} - 完整引用(需biblatex)
    \item \texttt{\textbackslash iqbfooterurl[title]\{url\}} - URL链接
    \item \texttt{\textbackslash iqbfooterdoi\{10.xxxx/xxx\}} - DOI链接
    \item \texttt{\textbackslash iqbfootcitewithurl\{key\}\{url\}} - 引用+URL
  \end{itemize}


  \iqbbluebox{最佳实践}{%
    引用文献优先用footcite;在线资源用footerurl;期刊文章同时显示DOI
  }
\end{frame}

\begin{frame}{学术徽章与图标}
  \textbf{可用图标命令}:

  \iqblayoutthree{%
    \textbf{学术身份}\\[0.2cm]
    \texttt{\textbackslash iqborcid\{ID\}}\\
    \texttt{\textbackslash iqbgscholar\{url\}}\\
    \texttt{\textbackslash iqbresearchgate\{name\}}
  }{%
    \textbf{社交与代码}\\[0.2cm]
    \texttt{\textbackslash iqbgithub\{user\}}\\
    \texttt{\textbackslash iqbtwitter\{user\}}\\
    \texttt{\textbackslash iqbwebsite\{url\}}
  }{%
    \textbf{联系方式}\\[0.2cm]
    \texttt{\textbackslash iqbemail\{email\}}\\
    \faLinkedin\ LinkedIn\\
    \faWeixin\ WeChat
  }


  \textbf{所需包}:\texttt{fontawesome5}(IQB主题已加载)
\end{frame}

% ============================================================
% Section 7: Real World Examples (真实案例)
% ============================================================
\iqbsectionframe{Real Cases}{真实案例展示}

\begin{frame}{真实案例1:挑战页面 - 突出核心问题}
  \footnotesize
  \textbf{来源}:membrane-pore-jc-original.tex

  \iqbtinysep

  \textbf{设计要点}:
  \iqbitemize{%
    \item 双列布局:左侧文字分析,右侧核心挑战+关键图
    \item 标题包含punchline:「统一描述成核与扩展两个截然不同的阶段」
    \item 使用 \texttt{\\iqbblock} 突出核心问题
    \item 图片高度0.5textheight,确保清晰可见
  }

  \iqbtinysep

  \iqbbluebox{排版技巧}{%
    问题分析页面:左侧列举具体问题(enumerated list),右侧用视觉元素(图+强调框)抓住注意力
  }

  \iqbtinysep

  {\scriptsize 参考:第143行,\texttt{\\iqblayouttwo}双列布局 + \texttt{\\iqbblock}强调框}
\end{frame}

\begin{frame}{真实案例2:方法页面 - 公式与图解并重}
  \footnotesize
  \textbf{来源}:membrane-pore-jc-original.tex (Line 260)

  \iqbtinysep

  \textbf{设计要点}:
  \iqbitemize{%
    \item 自定义1/3布局:左侧公式+要点,右侧大图
    \item Sigmoid函数用 \texttt{\$\$...\$\$} 独立成行
    \item 关键参数用itemize清晰列出
    \item 使用 \texttt{\\iqbfigcap} 添加详细图注
  }

  \iqbtinysep

  \iqborangebox{数学公式排版}{%
    方法页:左侧公式不要过多(2-3个核心公式),用bullet points解释参数物理意义
  }

  \iqbtinysep

  {\scriptsize 参考:第260行,\texttt{\\iqblayoutcustom[0.3]}自定义比例布局}
\end{frame}

\begin{frame}{真实案例3:结果页面 - 表格+图片+结论}
  \footnotesize
  \textbf{来源}:membrane-pore-jc-original.tex (Line 395)

  \iqbtinysep

  \textbf{设计要点}:
  \iqbitemize{%
    \item 2/3-1/3反向布局:左侧表格+图注,右侧高图
    \item 两个 \texttt{\\iqbthreelinetable} 展示不同阶段数据
    \item 图片占0.75textheight(竖版图需要足够高)
    \item 文字部分用 \texttt{\\iqbcaptiontext} 详细图注
  }

  \iqbtinysep

  \iqbgreenbox{结果呈现最佳实践}{%
    定量结果页:用表格精确展示数值,用图片展示趋势,用加粗文字强调核心发现
  }

  \iqbtinysep

  {\scriptsize 参考:第395行,\texttt{\\iqblayouttwothirds}布局 + 多表格组合}
\end{frame}

% ============================================================
% Thank You
% ============================================================
\iqbthankyouframe

% ============================================================
% Appendix: Backup Slides
% ============================================================
\iqbappendix

\begin{frame}{附录:核心命令速查(1/2)}
  \textbf{核心布局}:
  \begin{itemize}\tiny
    \item \texttt{\\iqblayouttwo}, \texttt{\\iqblayoutonethird}, \texttt{\\iqblayoutthree}
  \end{itemize}

  \textbf{视觉增强}:
  \begin{itemize}\tiny
    \item \texttt{\\emphdata}, \texttt{\\keyfinding}, \texttt{\\iqbbluebox}
  \end{itemize}

  \textbf{学术模块}:
  \begin{itemize}\tiny
    \item \texttt{\\iqbfig}, \texttt{\\iqbthreelinetable}, \texttt{\\iqbauthorstwophoto}
  \end{itemize}
\end{frame}

\begin{frame}{附录:核心命令速查(2/2)}
  \textbf{引用与学术标识}:
  \begin{itemize}\tiny
    \item \texttt{\\iqbfootcite}, \texttt{\\iqbfooterdoi}, \texttt{\\iqborcid}, \texttt{\\iqbemail}
  \end{itemize}

  详细命令列表请参考 \texttt{iqb-layouts.sty} 或 GitHub README
\end{frame}

\begin{frame}{附录:更多资源}
  \textbf{模板文件}:
  \begin{itemize}\small
    \item \texttt{jc-template.tex} - 空白起始模板
    \item \texttt{membrane-pore-jc-original.tex} - 真实案例
    \item \texttt{features-showcase.tex} - 本展示文件
  \end{itemize}


  \textbf{文档}:
  \begin{itemize}\small
    \item \texttt{iqb-layouts.sty} - 命令详细注释
    \item GitHub README - 快速入门指南
  \end{itemize}


  \iqbbluebox{贡献与反馈}{%
    发现问题或有建议?请在项目仓库提交 Issue
  }
\end{frame}

\end{document}
