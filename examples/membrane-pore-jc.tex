% ============================================================
% IQB Journal Club: Membrane Pore Free Energy
% 膜孔自由能与稳定性的分子动力学模拟研究
% Date: 2025-10-20
% Total Pages: 14
% ============================================================

\documentclass[aspectratio=169,11pt]{beamer}

% Load IQB theme
\usepackage{../theme/beamerthemeiqb}
\usepackage{../theme/iqb-layouts}

% Set header image path (relative to this .tex file)
\renewcommand{\iqbheaderimage}{../theme/images/header.png}

% Additional packages
\usepackage{graphicx}
\usepackage{amsmath}
\usepackage{amssymb}
\usepackage[version=4]{mhchem}  % For chemical formulas (\ce command)
\usepackage{booktabs}
\usepackage{hyperref}
\hypersetup{
  colorlinks=true,
  linkcolor=blue,
  urlcolor=blue,
  citecolor=blue
}

% Chinese support (XeLaTeX required)
\usepackage{xeCJK}
\setCJKmainfont{SimSun}  % Windows: SimSun, Linux: Noto Sans CJK SC

% Metadata
\title{Free Energy of Membrane Pore Formation and Stability}
\subtitle{from Molecular Dynamics Simulations}
\author{高旭帆}
\institute{IQB Lab}
\date{\today}

% ============================================================
\begin{document}

% ============================================================
% Page 1: Cover Page (无 header/footer)
% ============================================================
\begin{frame}[plain,noframenumbering]
  \vfill
  \centering

  {\LARGE\textcolor{iqbblue}{\textbf{破解膜孔之谜:双CV联手揭示}}}\\[0.3cm]
  {\LARGE\textcolor{iqbblue}{\textbf{从成核到扩展的完整能量图景}}}

  \vspace{0.8cm}

  {\large Free Energy of Membrane Pore Formation and Stability\\
  from Molecular Dynamics Simulations}

  \vspace{1.2cm}

  {\normalsize 汇报人:高旭帆}\\[0.2cm]
  {\normalsize IQB Lab}\\[0.2cm]
  {\normalsize \today}

  \vfill
\end{frame}

% ============================================================
% Page 2: Paper Information (通讯作者 + 一作 with photos)
% ============================================================
\iqbsectionframe{Overview}{概述}
\begin{frame}{论文信息}
  \setauthorfirstfield{计算生物物理、膜动力学、增强采样方法}
  \iqbauthorstwophoto{images/membrane-pore-jc/corresponding.png}{Robert Vácha}{%
    CEITEC and NCBR, Faculty of Science\\
    Masaryk University, Brno, Czech Republic
  }{%
    \href{https://vacha.ceitec.cz}{vacha.ceitec.cz} | \href{https://scholar.google.com/citations?user=NEt2O0MAAAAJ}{Google Scholar}
  }{%
    膜蛋白相互作用、抗菌肽、多尺度分子模拟
  }{images/membrane-pore-jc/first_author.png}{Timothée Rivel}{%
    CTO at InSiliBio\\
    Université de Franche-Comté, France
  }{%
    \href{https://www.insilibio.com}{www.insilibio.com}
  }

  \vspace{0.25cm}

  \footnotesize
  \textbf{发表}: \textit{J. Chem. Inf. Model.} \textbf{65}, 908–920 (2025) | \textbf{DOI}: 10.1021/acs.jcim.4c01960
\end{frame}

% ============================================================
% Page 3: Background
% ============================================================
\iqbsectionframe{Background}{背景}
\iqbframetextfig[0.50\textheight]{膜孔形成:从抗菌肽到电穿孔的关键过程}{
  \textbf{膜孔形成的重要性}:
  \iqbitemize{
    \item 抗菌肽设计
    \item 药物递送系统
    \item 细胞通透性调控
    \item 电穿孔技术
  }

  \medskip

  \textbf{实验方法的局限}:
  \iqbitemize{
    \item 缺乏时空分辨率
    \item 难以捕捉瞬态结构
    \item 无法量化能量学
  }
}{images/membrane-pore-jc/fig3.png}

% ============================================================
% Page 4: Key Scientific Questions
% ============================================================
\begin{frame}{挑战:统一描述成核与扩展两个截然不同的阶段}
  {\small\iqbblock{核心挑战}{
    如何通过MD模拟准确量化膜孔形成全过程(成核+扩展),\\并可靠预测不同条件下的膜线张力?
  }}

  \medskip

  {\small
  \iqblayouttwo{
    \textbf{现有方法的问题}:
    \iqbenumerate{
      \item 传统CV难以统一描述成核和扩展
      \item 常出现滞后现象和收敛问题
      \item 不同力场预测准确性差异大
      \item 缺乏快速评估膜稳定性的方法
    }
  }{
    \textbf{为什么这很难}:
    \iqbitemize{
      \item 成核:极性重原子数量变化
      \item 扩展:孔半径线性增长
      \item 两者物理机制不同
      \item 需要无缝切换的描述
    }

    \medskip

    \textbf{时间尺度}:\\
    成核 $\sim$ ns,扩展 $\sim$ $\mu$s,自发形成 $\sim$ ms
  }
  }
\end{frame}

% ============================================================
% Page 5: Innovations (三列布局)
% ============================================================
\iqbsectionframe{Methods}{方法}
\begin{frame}{创新:双CV策略破解成核-扩展统一描述难题}
  \iqbthreecolcompare
    {Full-Path CV}{images/membrane-pore-jc/fig1u.png}{%
      \item 成核+扩展统一
      \item 尾部密度驱动
      \item 无滞后、可逆
      \item 平滑切换
    }
    {Rapid CV}{images/membrane-pore-jc/fig1d.png}{%
      \item 无限大孔模拟
      \item 脂质条带构型
      \item 线张力快速评估
      \item 效率提高10×
    }
    {开源实现}{images/membrane-pore-jc/plumed.png}{%
      PLUMED库实现\\
      GROMACS/LAMMPS\\
      全原子/粗粒化\\
      参数可定制
    }
\end{frame}

% ============================================================
% Page 6: Method - Full-Path CV Principle
% ============================================================
\iqbformulaframe{Full-Path:切换函数巧妙结合成核与扩展}{
  {\small
  \iqbformblock{成核部分 $\text{CV}_{\text{cyl}}$}{
    $$\text{CV}_{\text{cyl}} = 1 - d/\text{CV}_{\text{eq}}$$
  }{$d$: 圆柱内脂质尾部原子数}

  \iqbformblock{扩展部分 $\text{CV}_{\text{radius}}$}{
    $$\text{CV}_{\text{radius}} = r_{\text{min}}/r_{\text{unit}}$$
  }{$r_{\text{min}}$: 孔中心到最近脂质距离}

  \iqbformblock{联合 CV}{
    $$\text{CV} = \text{CV}_{\text{cyl}} \times s_1 + \text{CV}_{\text{radius}} \times s_2$$
  }{}
  }
}{
  \centering
  \includegraphics[height=0.50\textheight]{images/membrane-pore-jc/fig1.png}

  \tinysep
  {\footnotesize
  \raggedright
  \textbf{图1}:本工作引入的集体变量示意图。上:Full-Path CV(成核追踪圆柱内尾部密度,扩展追踪孔半径);下:Rapid CV(脂质条带模拟无限孔)
  }
}

% ============================================================
% Page 7: Method - Switching Functions
% ============================================================
\begin{frame}{平滑过渡:sigmoid函数消除成核-扩展边界}
  {\small
  \iqblayouttwo{
    \iqbformblock{切换函数 $s_1, s_2$}{
      $$s_1 = \frac{1}{1 + e^{\alpha(\text{CV}_r - \text{CV}_0)}}$$
      $$s_2 = \frac{1}{1 + e^{-\alpha(\text{CV}_r - \text{CV}_0)}}$$
    }{}

    \iqbformblock{优化参数}{
      $\alpha = 20$ (陡峭)\\
      $\text{CV}_0 = 0.95$ (切换点)
    }{CV < 0.95: 主导 $\text{CV}_{\text{cyl}}$ | CV > 0.95: 主导 $\text{CV}_{\text{radius}}$}

    \textbf{关键优势}:
    \iqbitemize{
      \item 平滑过渡,无不连续性
      \item 避免数值问题
      \item 单一CV占主导
    }
  }{
    \centering
    \includegraphics[width=0.9\textwidth]{images/membrane-pore-jc/switching-functions.png}

    \medskip
    {\footnotesize
    \raggedright
    \textbf{图2}:两个切换函数在 $\text{CV}_0 = 0.95$ 处相交,权重各为 0.5。$s_1$(蓝色)在 CV < 0.95 时接近 1,$s_2$(红色)在 CV > 0.95 时接近 1。
    }
  }
  }
\end{frame}

% ============================================================
% Page 8: Validation - Pore State Definition
% ============================================================
\iqbsectionframe{Results}{结果}
\iqbresultframe{孔状态定义:17层切片追踪跨膜水通道形成}{
  \footnotesize
  \textbf{切片划分}:
  \iqbitemize{
    \item 膜沿z轴分17个切片
    \item 每片厚0.25 nm
    \item 范围:[-2.125, 2.125] nm
  }

  \medskip

  \textbf{孔状态 $s(t)$ 计算}:
  \iqbenumerate{
    \item 高斯平滑计数 $s_i(t)$
    \item Heaviside判定:$\mathcal{H}(s_i - 1)$
    \item 平均:$s(t) = \frac{1}{17}\sum \mathcal{H}(s_i - 1)$
  }

  \medskip

  \textbf{孔寿命}:DMPC $>$ DPPC $>$ POPC $>$ DOPC
}{images/membrane-pore-jc/fig2.png}

% ============================================================
% Page 9: Validation - Pore Closure Snapshots
% ============================================================
\iqbframetextfig[0.65\textheight]{孔闭合过程:尾部碳密度决定膜缺陷稳定性}{
  \footnotesize
  \textbf{四阶段闭合过程}:
  \iqbenumerate{
    \item \textbf{(A) 平衡孔}:初始稳定孔,连续水柱贯穿膜
    \item \textbf{(B) 半径缩小}:孔边缘脂质重排,结构保持
    \item \textbf{(C) 水线程}:闭合最后瞬间,仅剩连续水线
    \item \textbf{(D) 膜变薄}:孔完全闭合,局部膜缺陷
  }

  \medskip

  \textbf{关键发现}:
  {\color{iqbblue}\textbf{脂质尾部密度}} $\leftrightarrow$ 孔寿命 $\tau$ 强相关 (R²=0.82)

  \tinysep

  启发设计基于\textbf{尾部密度}的 $\text{CV}_{\text{cyl}}$

  \medskip

  \footnotesize
  \textbf{力场对比}:
  \textit{C36}: 122/94/34/15 ns
  \textit{Slipids}: 110/32/27/18 ns
}{images/membrane-pore-jc/fig3.png}

% ============================================================
% Page 10: Results - Full-Path Free Energy Profile
% ============================================================
\iqbformulaframe{Full-Path结果:正反向拉伸完全重合,CV设计可逆无滞后}{
  \iqbformblock{成核阶段 ($\text{CV} < 0.5$)}{
    $$\Delta G = k \cdot \text{CV}^2 + c$$
  }{二次增长,$k$ 表征成核能垒}

  \iqbformblock{扩展阶段 ($\text{CV} > 1.2$)}{
    $$G(r) = 2\pi r\gamma$$
  }{线性增长,斜率给出线张力}

  \iqbformblock{关键结果}{
    \textit{CHARMM36}: $k=72.9$ kJ/mol, $\gamma=32.5$ pN\\
    \textit{Martini 2.2p}: $k=73.7$ kJ/mol, $\gamma=49.2$ pN
  }{{\color{iqbblue}\textbf{无滞后}}:正反向完全重合}
}{
  \centering
  \includegraphics[height=0.65\textheight]{images/membrane-pore-jc/fig4.png}

  \tinysep
  {\footnotesize
  \raggedright
  \textbf{图4}:Full-Path自由能剖面。(A) 膜孔快照 (B) 自由能曲线及分段拟合 (C) 力场对比
  }
}

% ============================================================
% Page 11: Results - Rapid Method
% ============================================================
\iqbframefigttext[0.70\textheight]{Rapid方法:脂质条带模拟"无限孔"快速提取线张力}{images/membrane-pore-jc/fig5ab.png}{
  \footnotesize
  \textbf{核心思路}:
  \iqbitemize{
    \item 拉伸盒子 $\Rightarrow$ 脂质条带
    \item PBC连接 $\Rightarrow$ 环形无限孔
    \item 改变 $L_x$ $\Rightarrow$ 改变孔边缘长度
  }

  \medskip

  \iqbformblock{物理基础}{
    $$\Delta G = 2L\gamma$$
    $$\gamma = m/(2N_A)$$
  }{因子2:两个孔边缘}

  \iqbformblock{关键结果}{
    \textit{C36}: $\gamma=34.1$ pN\\
    \textit{M2.2p}: $\gamma=49.2$ pN
  }{计算效率高:21窗×150ns,可减半仍准确}
}

% ============================================================
% Page 12: Results - Force Field Comparison
% ============================================================
\iqbframetextfig[0.53\textheight]{力场筛选:仅CHARMM36和prosECCo75准确复现实验趋势}{
  \footnotesize
  \textbf{实验趋势}:\\
  POPE $>$ POPS $>$ POPC $>$ POPG

  \medskip

  \textbf{力场表现}:
  \iqbitemize{
    \item {\color{green!60!black}\textbf{C36/prosECCo75}}:准确(NMR调参精细)
    \item {\color{orange}\textbf{Slipids/M2.2p}}:部分正确
    \item {\color{red}\textbf{M2.2/M3}}:失败(粗粒化失效)
  }

  \medskip

  \textbf{核心差异}:阴离子脂质头部与 $\ce{Na+}$/$\ce{Ca^{2+}}$ 结合的描述精度

  \medskip

  \small
  结论:选择头部-离子作用精确的力场至关重要
}{images/membrane-pore-jc/fig5c.png}

% ============================================================
% Page 13: Results - Cross-Validation
% ============================================================
\iqbframefigttext[0.70\textheight]{双重验证:Full-Path与Rapid高度一致,与实验定性吻合}{images/membrane-pore-jc/fig6.png}{
  \footnotesize

  \textbf{(A) 与实验对比}:

  {\color{iqbblue}\textbf{定性吻合}}:POPC:POPG膜,PG↑ $\to$ $\gamma$↓,Ca$^{2+}$ 恢复

  {\color{orange}\textbf{定量偏差}}:Ca$^{2+}$ 幅度(实验~40 pN,模拟~6 pN)

  \medskip

  \textbf{(B) 两方法高度一致}:
  \iqbitemize{
    \item Full-Path vs Rapid相关性极高
    \item 系统偏差:C36 Rapid低~2 pN;M2.2p 高~3-5 pN
    \item 原因:孔几何(圆vs环)、离子浓度定义
  }

  \medskip

  {\color{iqbblue}\textbf{结论}}:两种CV互补验证,结果可信度高
}

% ============================================================
% Page 14: Results - Ion Effects
% ============================================================
\iqbframetextfig[0.70\textheight]{离子效应:Ca$^{2+}$恢复线张力,Na$^+$增强阴离子膜效应}{
  \footnotesize

  \textbf{(A) Ca$^{2+}$恢复效应}:
  \iqbitemize{
    \item POPC:POPG+NaCl:$\gamma$降低 (~29 pN)
    \item 添加 CaCl$_2$:$\gamma$恢复 (~35.5 pN)
    \item 机制:Ca$^{2+}$ 与PG强结合,屏蔽负电荷
  }

  \medskip

  \textbf{(B) Na$^+$对阴离子脂质效应}:
  \iqbitemize{
    \item POPG/POPS:$\Delta\gamma > 0$(增加30-50\%)
    \item POPC/POPE:$\Delta\gamma \approx 0$
    \item 机制:Na$^+$ 屏蔽静电 $\to$ 疏水代价增加
  }

  \medskip

  \textbf{力场表现}:
  {\color{green!60!black}\textit{C36}}准确 | {\color{orange}\textit{p75}}偏小 | {\color{red}\textit{M2.2p}}低估
}{images/membrane-pore-jc/fig7.png}

% ============================================================
% Page 14: Inspirations (重点页)
% ============================================================
\setsection{Discussion}
\begin{frame}{Inspirations: 借鉴意义}
  \begin{block}{对 HA/OP 膜相互作用研究的启发}
    \centering
    \footnotesize
    \textbf{核心洞察}:OP 降低膜线张力 $\to$ 促进膜孔形成/稳定 $\to$ 可能的透皮机制
  \end{block}

  \vspace{0.25cm}

  \iqblayouttwo{
    \textbf{可借鉴的方法}:

    \tinysep

    \footnotesize
    \textbf{1. 线张力计算}\\
    $\bullet$ Full-Path CV: 成核+扩展全过程\\
    $\bullet$ Rapid CV: 快速评估大孔极限\\
    $\bullet$ 伞形采样 + WHAM 自由能

    \medskip

    \textbf{2. 膜性质表征}\\
    $\bullet$ Lindemann 指数:脂质流动性\\
    $\bullet$ Bond-orientational order:相态\\
    $\bullet$ Area per lipid:膜面积变化

    \medskip

    \textbf{3. 力场选择}\\
    $\bullet$ CHARMM36: 准确离子-脂质作用\\
    $\bullet$ Martini 3: 快速筛选\\
    $\bullet$ 全原子验证关键发现
  }{
    \textbf{与我们研究的关联}:

    \tinysep

    \footnotesize
    \textbf{实验观察}:\\
    {\color{iqbblue}$\bullet$} OP仅与FFA结合,不与CER/CHOL结合\\
    {\color{iqbblue}$\bullet$} pH<5时OP质子化带正电\\
    {\color{iqbblue}$\bullet$} 200nm纳米凝胶穿过20nm通道

    \medskip

    \textbf{假说}:\\
    {\color{orange}$\bullet$} OP形成"局部阳离子补丁"\\
    {\color{orange}$\bullet$} 特异性结合FFA羧基(-COO$^-$)\\
    {\color{orange}$\bullet$} 降低膜线张力,诱导瞬时孔道\\
    {\color{orange}$\bullet$} 纳米凝胶通过形变挤压穿过

    \medskip

    \textbf{下一步}:\\
    $\Rightarrow$ 定量计算OP-FFA膜的线张力\\
    $\Rightarrow$ 对比纯SC膜 vs OP结合后的$\gamma$\\
    $\Rightarrow$ 验证线张力降低假说
  }
\end{frame}

% ============================================================
% Page 15: Thank You (致谢页)
% ============================================================
\begin{frame}[plain]
  \vfill
  \centering

  {\Huge \textcolor{iqbblue}{\textbf{Thanks for Listening!}}}

  \vspace{1.5cm}

  {\Large Questions?}

  \vspace{2cm}

  {\normalsize
  \textbf{参考文献}:\\[0.2cm]
  Rivel, T., Biriukov, D., Kabelka, I., \& Vácha, R. (2025).\\
  \textit{J. Chem. Inf. Model.} \textbf{65}, 908–920.\\
  DOI: 10.1021/acs.jcim.4c01960
  }

  \vfill
\end{frame}

\end{document}
