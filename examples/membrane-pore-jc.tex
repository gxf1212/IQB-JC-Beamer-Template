% ============================================================
% IQB Journal Club: Membrane Pore Free Energy
% 膜孔自由能与稳定性的分子动力学模拟研究
% Date: 2025-10-20
% Total Pages: 14
% ============================================================

\documentclass[aspectratio=169,11pt]{beamer}

% Load IQB theme
\usepackage{../theme/beamerthemeiqb}
\usepackage{../theme/iqb-layouts}

% Set header image path (relative to this .tex file)
\renewcommand{\iqbheaderimage}{../theme/images/header.png}
\setstretch{1.0}

% Metadata
\title{Free Energy of Membrane Pore Formation and Stability}
\subtitle{from Molecular Dynamics Simulations}
\author{高旭帆}
\institute{Institute of Quantitative Biology}
\date{\today}

% ============================================================
% Paper Information (论文配置 - 在此定义,全文自动使用)
% ============================================================
\papertitle{膜孔形成的自由能与稳定性的分子动力学模拟研究}
\paperengtitle{Free Energy of Membrane Pore Formation and Stability from Molecular Dynamics Simulations}
\papertitlechn{破解膜孔之谜:双CV联手揭示}
\papertitlechnsub{从成核到扩展的完整能量图景}
\papercitation{Rivel, T., Biriukov, D., Kabelka, I., \& Vácha, R. (2025). J. Chem. Inf. Model. \textbf{65}, 908–920.}
\paperdoi{10.1021/acs.jcim.4c01960}
\paperjournal{Journal of Chemical Information and Modeling}

% ============================================================
\begin{document}

% ============================================================
% Page 1: Cover Page (using template - 一行式命令)
% ============================================================
\iqbcoverframe

% ============================================================
% Page 2: Paper Information (通讯作者 + 一作 with photos)
% ============================================================
% \iqbsectionframe{Overview}{论文信息}
\begin{frame}{发表论文}
  \setauthorfirstfield{计算生物物理、膜动力学、增强采样方法}
  \iqbauthorstwophoto{images/membrane-pore-jc/corresponding.png}{Robert Vácha}{%
    CEITEC and NCBR, Faculty of Science\\
    Masaryk University, Brno, Czech Republic
  }{%
    \href{https://vacha.ceitec.cz}{vacha.ceitec.cz} | \href{https://scholar.google.com/citations?user=NEt2O0MAAAAJ}{Google Scholar}
  }{%
    膜蛋白相互作用、抗菌肽、多尺度分子模拟
  }{images/membrane-pore-jc/first_author.png}{Timothée Rivel}{%
    CTO at InSiliBio\\
    Université de Franche-Comté, France
  }{%
    \href{https://www.insilibio.com}{www.insilibio.com}
  }

  \bigskip

  \footnotesize
  \iqbpublicationinfo
\end{frame}

% ============================================================
% Page 3: Background
% ============================================================
\iqbsectionframe{Background}{背景和意义}
\begin{frame}{膜孔形成:从抗菌肽到电穿孔的关键过程}
  \iqblayouttwo{
    \textbf{生物学意义}:
    膜孔形成是理解细胞防御机制和设计治疗策略的关键。\\

    \iqbitemize{
      \item \textbf{抗菌肽}:诱导孔形成破坏细胞屏障
      \item \textbf{药物递送}:控制跨膜物质运输
      \item \textbf{电穿孔}:促进大分子受控递送
      \item \textbf{细胞生物学}:基本膜转运机制
    }

    \medskip

    \textbf{实验局限}:
    中子散射、固态NMR、AFM等虽能提供信息,但存在根本缺陷:
    \iqbitemize{
      \item 静态快照难以捕捉瞬态机制,追踪纳秒动态过程
      \item 原子尺度分辨率不足
      \item 缺乏定量自由能学数据
    }

    \medskip

    \textbf{MD模拟优势}:
    \iqbitemize{
      \item 获得完整结构动态信息,准确量化膜线张力和成核能垒,解决成核与扩展统一描述难题
    }
  }{
    \iqbfigcap[height=0.52\textheight]{images/membrane-pore-jc/fig3.png}{%
      \textbf{图3}:孔闭合的四个阶段。(A)平衡孔,水线贯穿膜;(B)半径缩小,脂质重排;(C)临界瞬间,仅剩水线;(D)孔闭合,膜恢复。
    }
  }
\end{frame}

% ============================================================
% Page 4: Key Scientific Questions
% ============================================================
\begin{frame}{挑战:统一描述成核与扩展两个截然不同的阶段}

  \iqblayouttwo{
    \textbf{现有方法的问题}:
    \iqbenumerate{
      \item 传统CV难以统一描述成核和扩展
      \item 常出现滞后现象和收敛问题
      \item 不同力场预测准确性差异大
      \item 缺乏快速评估膜稳定性的方法
    }

    \medskip

    \textbf{为什么这很难}:
    \iqbitemize{
      \item 成核:极性重原子数量变化
      \item 扩展:孔半径线性增长
      \item 两者物理机制,需要无缝切换的描述
    }

    \medskip

    \textbf{时间尺度}:\\
    成核 $\sim$ ns,扩展 $\sim$ $\mu$s,自发形成 $\sim$ ms
  }{
    \centering
    
      \iqbblock{核心挑战}{\scriptsize
    如何通过MD模拟准确量化膜孔形成全过程(成核+扩展),并可靠预测不同条件下的膜线张力?不同离子条件下的膜线张力如何变化?
  }
  
  \medskip

    \includegraphics[height=0.5\textheight,keepaspectratio]{images/membrane-pore-jc/fig4b.png}
  }
\end{frame}

% ============================================================
% Page 4.5: Research Overview (摘要图+流程图)
% ============================================================
\begin{frame}{研究概览:问题、方法与核心发现}
  % 上半部:左右两栏(方法创新 + 核心发现)
  \iqblayouttwo{
    % 左列:方法创新
    \textbf{方法创新}:
    \iqbitemize{
      \item 双CV策略:Full-Path CV(成核+扩展统一)+ Rapid CV(快速评估)
      \item Sigmoid切换函数实现平滑过渡
      \item 多力场对标验证(CHARMM36/M2.2p/prosECCo75)
    }
  }{
    % 右列:核心发现
    \textbf{核心发现}:
    \iqbitemize{
      \item 线张力直接反映膜孔稳定性
      \item PG脂质含量增加导致线张力降低
      \item $\ce{Ca^{2+}}$恢复膜稳定,$\ce{Na+}$增强阴离子效应
      \item 力场选择关键(C36精确,M2.2p低估)
    }
  }

  % 下半部:mermaid流程图
  \medskip
  \iqbimgcenter[height=0.32\textheight,keepaspectratio]{images/membrane-pore-jc/mermaid.png}
\end{frame}

% ============================================================
% Page 5: Innovations (三列布局)
% ============================================================
\iqbsectionframe{Methods}{方法}
\begin{frame}{创新:双CV策略破解成核-扩展统一描述难题}
  \iqbthreecolcompare
    {Full-Path CV}{images/membrane-pore-jc/fig1u.png}{%
      \item 成核+扩展统一
      \item 尾部密度驱动
      \item 无滞后、可逆
      \item 平滑切换
    }
    {Rapid CV}{images/membrane-pore-jc/fig1d.png}{%
      \item 无限大孔模拟
      \item 脂质条带构型
      \item 线张力快速评估
      \item 效率提高数倍
    }
    {开源实现}{images/membrane-pore-jc/plumed.png}{%
      \item PLUMED库实现
      \item GROMACS/LAMMPS
      \item 全原子/粗粒化
      \item 参数可定制
    }
\end{frame}

% ============================================================
% Page 6: Method - Full-Path CV Principle
% ============================================================
\iqbformulaframe{Full-Path:切换函数巧妙结合成核与扩展}{
  \iqbformblock{成核部分 $\text{CV}_{\text{cyl}}$}{
    $$\text{CV}_{\text{cyl}} = 1 - d/\text{CV}_{\text{eq}}$$
  }{$d$: 圆柱内脂质尾部原子数}

  \iqbformblock{扩展部分 $\text{CV}_{\text{radius}}$}{
    $$\text{CV}_{\text{radius}} = r_{\text{min}}/r_{\text{unit}}$$
  }{$r_{\text{min}}$: 孔中心到最近脂质距离}

  \iqbformblock{联合 CV}{
    $$\text{CV} = \text{CV}_{\text{cyl}} s_1 + \text{CV}_{\text{r}} s_2$$
  }{权重函数平滑切换}
}{
    \iqbfigcap[height=0.45\textheight]{images/membrane-pore-jc/fig1u.png}{%
      \textbf{图1}:Full-Path CV设计。同时追踪成核(圆柱内尾部密度CV$_{\text{cyl}}$)和扩展(孔半径CV$_{\text{r}}$);通过sigmoid切换函数平滑过渡。
    }
}

% ============================================================
% Page 7: Method - Switching Functions
% ============================================================
\begin{frame}{平滑过渡:sigmoid函数消除成核-扩展边界}
  \iqblayoutcustom[0.3]{
    \textbf{切换函数 $s_1, s_2$}:
    $$s_1 = \frac{1}{1 + e^{\alpha(\text{CV}_r - \text{CV}_0)}}$$
    $$s_2 = \frac{1}{1 + e^{-\alpha(\text{CV}_r - \text{CV}_0)}}$$

    \medskip

    \iqbitemize{
      \item $\alpha = 20$ :陡峭
      \item $\text{CV}_0 = 0.95$ :切换点
      \item $CV < 0.95$ :主导 $\text{CV}_{\text{cyl}}$
      \item $CV > 0.95$ :主导 $\text{CV}_{\text{radius}}$
    }

    \textbf{关键优势}:
    \iqbitemize{
      \item 平滑过渡,无不连续性
      \item 避免数值问题
      \item 单一CV占主导
    }
  }{
    \iqbfigcap[width=0.9\textwidth]{images/membrane-pore-jc/switching-functions.png}{%
      \textbf{图S2}:Sigmoid切换函数实现平滑过渡。两条曲线在 $\text{CV}_0 = 0.95$ 处相交,权重各为 0.5。
    }
  }
\end{frame}

% ============================================================
% Page 7: Method - Rapid Method
% ============================================================
\begin{frame}{Rapid方法:脂质条带模拟"无限孔"快速估算线张力}
  \iqblayoutcustom[0.4]{
    {\footnotesize
    \textbf{创新思路}:(1) 沿膜平面扩展模拟盒子 (2) 自发形成脂质条带 (3) 周期边界PBC连接 (4) 形成环形"无限孔"

    \medskip

    \textbf{物理原理}:线张力 $\Delta G = 2L\gamma$,其中 $L$ 为孔边缘总长度,因子2考虑两个孔边缘,改变盒子尺寸 $L_x$ = 改变孔边缘长度

    \medskip

    \textbf{关键优势}:参考态通过线性外推自动确定(无需实际模拟完整膜),周期边界消除边缘效应,物理上等价于 $\Delta G = 2\pi r\gamma$(圆形孔)

    \medskip

    \textbf{计算效率}:Rapid 仅需 21窗口 $\times$ 150 ns,Full-Path 需要 65窗口 $\times$ 200 ns,效率提升 $\sim$6倍,准确性与Full-Path高度一致
    }
  }{
    \centering
    \iqbfigcap[height=0.5\textheight]{images/membrane-pore-jc/fig1d.png}{\textbf{图1B}:Rapid方法示意。(A)脂质条带侧视图,两个孔边缘(膜外界面)中间夹水;(B)俯视图,通过周期边界形成环形"无限孔";(C)自由能随盒子尺寸线性增长,斜率 = $2\gamma$,线性拟合确定线张力。}
  }
\end{frame}

% ============================================================
% Page 8: Validation - Pore State Definition
% ============================================================
\iqbsectionframe{Results}{结果}
\begin{frame}{孔状态定义:17层切片追踪跨膜水通道形成}
  \iqblayouttwothirds{
    {\footnotesize
    \textbf{切片划分方法}:
    \iqbitemize{
      \item 膜沿z轴分17个切片,每片厚0.25 nm
      \item 范围:[-2.125, 2.125] nm,高斯权重平滑计数
    }

    \medskip

    \textbf{孔状态$s(t)$计算}:
    \iqbitemize{
      \item $s_i(t)$:高斯平滑水原子计数
      \item $\mathcal{H}(s_i - 1)$:Heaviside判定
      \item $s(t) = \frac{1}{17}\sum \mathcal{H}(s_i - 1)$
    }

    \medskip

    \textbf{物理意义}:
    \iqbitemize{
      \item $s=1$:完全开放跨膜孔;$s=0$:完整无孔膜
      \item $0<s<1$:孔形成/闭合过程
    }

    \medskip

    (B)4种磷脂闭合过程。\textbf{孔寿命趋势}:DMPC $>$ DPPC $>$ POPC $>$ DOPC。饱和度越高、链越短,寿命越长。
    }
  }{
    \iqbfigcap[height=0.65\textheight,keepaspectratio]{images/membrane-pore-jc/fig2.png}{\textbf{图2}:孔状态追踪及闭合动力学。蓝色为含水通道。}
  }
\end{frame}

% ============================================================
% Page 9: Validation - Pore Closure Snapshots
% ============================================================
\begin{frame}{孔闭合过程:尾部碳密度决定膜缺陷稳定性}
  \iqblayouttwo{
    {\footnotesize
    \iqbsteplist{四阶段闭合动力学}{
      \step{(A)平衡孔}{初始稳定孔,连续水柱贯穿膜厚度}
      \step{(B)半径缩小}{孔边缘脂质重排,整体结构保持不变}
      \step{(C)水线程}{关键瞬间,孔只有纤细连续水线相连}
      \step{(D)膜恢复}{孔完全闭合,局部膜厚度恢复}
    }

    \medskip

    % \iqbhighlight{临界发现}{脂质尾部密度 $\leftrightarrow$ 孔寿命 $\tau$ (R²=0.82)}{相关性极强!}
    \textbf{临界发现}:脂质尾部密度 $\leftrightarrow$ 孔寿命 $\tau$ (R²=0.82),相关性极强!

    \tinysep

    \textbf{设计启示}:
    这一强相关性启发了\textbf{尾部密度CV}($\text{CV}_{\text{cyl}}$)的创新设计,追踪圆柱体内脂质尾部原子数。

    \medskip

    \iqbthreelinetable{孔寿命/ns}{
      力场 & DMPC & DPPC & POPC & DOPC \\
    }{
      C36 & 122 & 94 & 34 & 15 \\
      Slipids & 110 & 32 & 27 & 18 \\
    }
    }
  }{
    \iqbfigcap[height=0.7\textheight]{images/membrane-pore-jc/figs6.png}{%
      
    }
  }
\end{frame}

% ============================================================
% Page 10: Results - Full-Path Free Energy Profile
% ============================================================
\begin{frame}{Full-Path结果:正反向拉伸完全重合,CV设计可逆无滞后}
  \iqblayouttwothirds{
    {
    
    \iqbcaptiontext{\textbf{图4}:Full-Path自由能分析。(A)典型孔结构侧视图和俯视图;(B)自由能剖面展示两阶段,正反向完全重合无滞后;(C)不同力场线张力对比,CHARMM36和prosECCo75最准确。}

    \footnotesize

    \iqbthreelinetable{成核阶段 ($\text{CV} < 0.5$): $\Delta G = k \cdot \text{CV}^2 + c$}{
      力场 & $k$ (kJ/mol) \\
    }{
      CHARMM36 & 72.9 \\
      Martini 2.2p & 73.7 \\
    }

    \medskip

    \iqbthreelinetable{扩展阶段 ($\text{CV} > 1.2$): $G(r) = 2\pi r\gamma$}{
      力场 & $\gamma$ (pN) \\
    }{
      CHARMM36 & 32.5 \\
      Martini 2.2p & 49.2 \\
    }

    \medskip

    \textbf{关键发现}:正反向拉伸曲线完全重合,CV设计无滞后,CHARMM36和Martini 2.2p显示不同的线张力
    }
  }{
    \iqbimgcenter[height=0.75\textheight,keepaspectratio]{images/membrane-pore-jc/fig4.png}

    \tinysep
  }
\end{frame}

% ============================================================
% Page 11: Results - Rapid Method
% ============================================================
\begin{frame}{Rapid方法:脂质条带模拟"无限孔"快速提取线张力}
  \iqblayouttwothirds{
    {
    \iqbcaptiontext{\textbf{图5}:Rapid方法示意。(A)脂质条带侧视图,两个孔边缘(膜外界面)中间夹水;(B)俯视图,通过PBC形成环形无限孔;(C)自由能随盒子尺寸线性增长,斜率为$2\pi\gamma$。}
    
    \footnotesize
    % \textbf{创新思路}:(1) 沿膜平面拉伸模拟盒子(2) 形成脂质条带结构(3) 周期边界条件连接(4) 形成环形"无限孔"

    % \medskip

    \textbf{线张力原理}:大孔自由能为 $\Delta G = 2L\gamma$,其中$L$为孔边缘长度,因子2考虑两个孔边缘,通过线性拟合斜率直接提取$\gamma$

    \medskip

    \iqbthreelinetable{计算效率对比}{
      方法 & 窗数 & 时间 & 速率 \\
    }{
      Full-Path & 65 & 200 ns & 标准 \\
      Rapid & 21 & 150 ns & $\sim$6$\times$快 \\
    }

    \medskip

    \iqbthreelinetable{线张力预测结果}{
      脂质系统 & 力场 & $\gamma$ (pN) \\
    }{
      POPC & CHARMM36 & 34.1 \\
      POPC & Martini 2.2p & 49.2 \\
    }
    }
  }{
    \iqbimgcenter[height=0.8\textheight,keepaspectratio]{images/membrane-pore-jc/fig5ab.png}

  }
\end{frame}


% ============================================================
% Page 12: Results - Force Field Comparison
% ============================================================
\begin{frame}{力场筛选:仅CHARMM36和prosECCo75准确复现实验趋势}
  \iqblayouttwothirds{
    % \footnotesize
    \textbf{实验趋势}:POPE $>$ POPS $>$ POPC $>$ POPG

    \bigskip

    \iqbcolorlist{各力场表现}{
      \item {\color{green!60!black}\textbf{C36/prosECCo75}}:准确(NMR调参精细)
      \item {\color{orange}\textbf{Slipids/M2.2p}}:部分正确
      \item {\color{red}\textbf{M2.2/M3}}:失败(粗粒化失效)
    }

    \medskip

    \textbf{核心差异}:阴离子脂质头部与 $\ce{Na+}$/$\ce{Ca^{2+}}$ 结合的描述精度

    \medskip

    \small
    结论:选择头部-离子作用精确的力场至关重要
  }{
    \iqbfigcap[height=0.5\textheight]{images/membrane-pore-jc/fig5c.png}{}
  }
\end{frame}

% ============================================================
% Page 13: Results - Cross-Validation
% ============================================================
\begin{frame}{双重验证:Full-Path与Rapid高度一致,与实验定性吻合}
  \iqblayouttwothirds{
    {
      \textbf{图6}:交叉验证。

    \medskip
      \textbf{(A)与实验对标}:
      \begin{itemize}
        \item OK 定性吻合POPC:POPG膜
        \item PG增加导致线张力降低;$\ce{Ca^{2+}}$导致线张力恢复
        \item 定量偏差:实验40 pN vs 模拟6 pN(离子参数化精度差异)
      \end{itemize}

      \textbf{(B)两方法互补}:
      \begin{itemize}
        \item Full-Path vs Rapid相关性极高
        \item C36 Rapid偏低约2 pN;M2.2p偏高3-5 pN
        \item 原因:孔几何差异(圆柱 vs 环形)
      \end{itemize}

      \textbf{结论}:双CV互补,高可信度
    }
  }{
    \iqbimgcenter[height=0.62\textheight,keepaspectratio]{images/membrane-pore-jc/fig6.png}

    \tinysep
  }
\end{frame}

% ============================================================
% Page 14: Results - Ion Effects
% ============================================================
\begin{frame}{离子效应:$\ce{Ca^{2+}}$恢复被$\ce{Na+}$降低的膜线张力}
  \iqblayouttwo{
    \textbf{图7}:离子效应分析。
    \medskip

      \textbf{(A)$\ce{Ca^{2+}}$恢复膜稳定性}:
      \begin{itemize}
        \item POPC:POPG体系
        \item 仅加$\ce{NaCl}$:线张力降低到29 pN
        \item 再加$\ce{CaCl2}$:线张力升高恢复至35.5 pN
        \item 机制:$\ce{Ca^{2+}}$与PG头部强结合,静电屏蔽
      \end{itemize}

      \textbf{(B)$\ce{Na+}$增强阴离子的效应}:
      \begin{itemize}
        \item POPG/POPS:线张力变化增加30-50\%
        \item POPC/POPE:线张力变化约为零
        \item 结论:$\ce{Na+}$特异增强阴离子脂质膜稳定性
      \end{itemize}

      \textbf{力场性能}:C36精确,prosECCo75偏小,M2.2p会低估
  }{
    \iqbimgcenter[height=0.7\textheight,keepaspectratio]{images/membrane-pore-jc/fig7.png}
  }
\end{frame}

% ============================================================
% Page 14: Inspirations (重点页)
% ============================================================
\setsection{Discussion}
\begin{frame}{Inspirations: 借鉴意义}
  \begin{block}{对 HA/OP 膜相互作用研究的启发}
    \centering
    \footnotesize
    \textbf{核心洞察}:OP 降低膜线张力 $\to$ 促进膜孔形成/稳定 $\to$ 可能的透皮机制
  \end{block}

  \vspace{0.25cm}

  \iqblayouttwo{
    \iqbbulletlist{1. 线张力计算}{
      \item Full-Path CV: 成核+扩展全过程
      \item Rapid CV: 快速评估大孔极限
      \item 伞形采样 + WHAM 自由能
    }

    \medskip

    \iqbbulletlist{2. 膜性质表征}{
      \item Lindemann 指数:脂质流动性
      \item Bond-orientational order:相态
    }

    \medskip

    \iqbbulletlist{3. 力场选择}{
      \item CHARMM36: 准确离子-脂质作用
      \item Martini 3: 快速验证
    }
  }{
    \textbf{与我们研究的关联}:

    \tinysep

    \footnotesize
    \textbf{假说}:

    \begin{itemize}
      \item {\color{orange}OP形成"局部阳离子补丁"}
      \item {\color{orange}特异性结合FFA羧基(-$\ce{COO-}$)}
      \item {\color{orange}降低膜线张力,诱导瞬时孔道}
      \item {\color{orange}纳米凝胶通过形变挤压穿过}
    \end{itemize}

    \medskip

    \textbf{下一步}:

    \begin{itemize}
      \item[$\Rightarrow$] 定量计算OP-FFA膜的线张力
      \item[$\Rightarrow$] 对比纯SC膜 vs OP结合后的$\gamma$
      \item[$\Rightarrow$] 验证线张力降低假说
    \end{itemize}
  }
\end{frame}

% ============================================================
% Page 15: Thank You (致谢页 - 使用固定样式)
% ============================================================
\iqbthankyouframe

\end{document}
