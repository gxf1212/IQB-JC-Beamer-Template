%!TEX root = document.tex

\chapter{多列布局与模块系统}
\label{chap:4}

本章详细介绍 IQB-JC 模板的主题自定义选项,包括颜色修改、字体调整、间距控制等个性化配置。通过这些设置,用户可以根据自己的需求调整模板的外观,同时保持整体的专业性和一致性。

\section{主题自定义}

\subsection{修改主题颜色}

IQB-JC 模板的主题颜色定义在 \lstinline|theme/beamerthemeiqb.sty| 文件中。主要颜色包括:

\begin{table}[!h]
\centering
\caption{IQB-JC 主题颜色定义}
\label{tab:theme-colors}
\begin{tabular}{p{3cm}p{4cm}p{4cm}}
\toprule
\textbf{颜色名称} & \textbf{RGB 值} & \textbf{十六进制} \\
\midrule
\lstinline|iqbblue| & 0, 51, 102 & \#003366 \\
\lstinline|iqbgray| & 85, 85, 85 & \#555555 \\
\lstinline|iqblightgray| & 240, 240, 240 & \#F0F0F0 \\
\bottomrule
\end{tabular}
\end{table}

若要修改主题颜色,可在导言区添加颜色重定义:

\begin{lstlisting}[style=blockstyle,language=TeX]
\definecolor{iqbblue}{RGB}{0, 100, 200}  % 自定义蓝色
\definecolor{iqbgray}{RGB}{100, 100, 100}  % 自定义灰色
\end{lstlisting}

\subsection{替换 Header 图片}

Header 横幅图片可通过重新定义 \lstinline|\iqbheaderimage| 命令更改:

\begin{lstlisting}[style=blockstyle,language=TeX]
\renewcommand{\iqbheaderimage}{path/to/your/header.png}
\end{lstlisting}

建议的 header 图片规格:

\begin{itemize}
  \item 分辨率:1999×204 像素(保持原始 1999:204 比例)
  \item 宽高比:约 9.8:1
  \item 格式:PNG 或 PDF(支持透明度)
\end{itemize}

\subsection{调整字体大小与字体层级系统}

IQB-JC 模板采用精心设计的 5 级字体层级体系,确保页面整体视觉平衡且易于阅读。以下表格展示了完整的字体层级及其应用场景:

\begin{table}[!h]
\centering
\caption{IQB-JC 完整字体层级体系}
\label{tab:font-sizes}
\begin{tabular}{p{2cm}p{4cm}p{2.5cm}p{5.5cm}}
\toprule
\textbf{层级} & \textbf{文本类型} & \textbf{字号} & \textbf{使用场景} \\
\midrule
1 & 封面页标题 (\lstinline|\title|) & 14.4pt & 演示文稿的总标题,仅在封面使用 \\
2 & Frame Title(页标题) & 12pt & 每页最上方的标题栏,所有内容页必有 \\
3 & Section Title(段标题) & 11pt & 页面内的分段标题,使用 \lstinline|\iqbsectiontitle{}| 命令 \\
4 & 正文/Itemize(默认) & 9pt & 页面主要内容、列表项等日常文字 \\
5 & 图注/表注/脚注 & 8pt & 较小的辅助说明文字 \\
\bottomrule
\end{tabular}
\end{table}

\subsubsection{分段标题命令 (\texttt{\textbackslash iqbsectiontitle})}

在页面内使用分段标题来组织内容结构。分段标题会自动应用 11pt 大小、IQB 蓝色及加粗样式:

\begin{lstlisting}[style=blockstyle,language=TeX]
\iqbsectiontitle{FEbuilder - 高通量自由能计算工具}

高度优化的自由能计算框架...
\end{lstlisting}

\subsubsection{字体模式切换 (\texttt{\textbackslash iqbfontsizemode})}

模板提供了两种字体模式,可根据内容密度灵活切换:

\begin{enumerate}
  \item \textbf{normal 模式(默认)}:正文 9pt,适用于内容较少或需要突出视觉效果的页面
  \item \textbf{small 模式}:所有字体缩小 1 级,适用于内容密集的页面
\end{enumerate}

在演示文稿中切换模式的用法:

\begin{lstlisting}[style=blockstyle,language=TeX]
% 在特定页面启用 small 模式(内容较多)
\begin{frame}{内容密集的页面}
  \iqbfontsizemode{small}

  \begin{itemize}
    \item 项目 1
    \item 项目 2
    % ...更多项目
  \end{itemize}
\end{frame}

% 后续页面自动恢复 normal 模式
\end{lstlisting}

\textbf{注意}:字体模式的改变仅在当前 frame 内有效,不会影响其他页面。

\subsubsection{列表环境(iqbitemize)}

IQB-JC 提供了增强版的列表环境 \lstinline|iqbitemize|,支持自动为带标签的项目应用蓝色加粗样式:

\begin{lstlisting}[style=blockstyle,language=TeX]
\begin{iqbitemize}
  \item[标签1] 带标签的项目内容
  \item[标签2] 另一个带标签的项目
  \item 不带标签的普通项目
\end{iqbitemize}
\end{lstlisting}

标签会自动呈现为 IQB 蓝色加粗文字,无需手动指定颜色或加粗。

\subsubsection{描述列表环境(iqbdescription)}

对于包含较长标签(4 字以上)的列表,推荐使用 \lstinline|iqbdescription| 环境。相比 \lstinline|iqbitemize|,\lstinline|iqbdescription| 具有以下优势:

\begin{itemize}
  \item 标签自动右对齐,视觉更整齐
  \item 支持更长的标签文本(最多可显示 4--6 个中文字符)
  \item 可通过参数 \lstinline|[距离]| 灵活调整列表的左边距
\end{itemize}

\paragraph{基本用法}

\begin{lstlisting}[style=blockstyle,language=TeX]
\begin{iqbdescription}
  \item[文献处理] 文献检索、自动写综述、文献讲解PPT生成
  \item[多智能体协作] 多智能体交互助力科研方案设计
  \item[个性化智能体] 私域知识库构建和增量学习
\end{iqbdescription}
\end{lstlisting}

\paragraph{调整左边距}

使用可选参数 \lstinline|[距离]| 可将列表向右移动指定距离,默认为 0em:

\begin{lstlisting}[style=blockstyle,language=TeX]
% 标准用法(无缩进)
\begin{iqbdescription}
  \item[标签1] 内容
\end{iqbdescription}

% 向右移动 2.5cm
\begin{iqbdescription}[2.5cm]
  \item[标签1] 内容
\end{iqbdescription}
\end{lstlisting}

\paragraph{样式说明}

\lstinline|iqbdescription| 中的标签具有以下特点:
\begin{itemize}
  \item 标签显示为 IQB 蓝色(\textcolor{iqbblue}{\textbf{示例}})
  \item 标签文字为加粗显示
  \item 标签在 1cm 宽度内右对齐,确保整齐的竖排对齐
\end{itemize}

\paragraph{何时使用}

\begin{table}[!h]
\centering
\begin{tabular}{p{3cm}p{4cm}p{4cm}}
\toprule
\textbf{列表类型} & \textbf{iqbitemize} & \textbf{iqbdescription} \\
\midrule
无标签或短标签 & \checkmark(推荐) & 可用(不推荐) \\
中等长度标签(2--3 字) & 可用 & \checkmark(推荐) \\
较长标签(4 字以上) & \text{\ding{55}}(不适合) & \checkmark(推荐) \\
需要灵活缩进 & \checkmark & \checkmark(更灵活) \\
\bottomrule
\end{tabular}
\end{table}

\subsubsection{手动字体调整}

若要进行更细致的字体定制,编辑 \lstinline|theme/beamerthemeiqb.sty| 文件中的 \lstinline|\setbeamerfont| 命令。例如,以下代码控制正文字体:

\begin{lstlisting}[style=blockstyle,language=TeX]
% normal 模式下的正文字体
\setbeamerfont{normal text}{size=\footnotesize}  % 9pt

% small 模式下的正文字体
\setbeamerfont{normal text}{size=\scriptsize}   % 8pt
\end{lstlisting}

\section{间距与排版控制}

为了保持演示文稿的视觉一致性和专业感,IQB-JC 提供了三个标准化的间距命令。相比手动使用 \lstinline|\vspace{}|,这些命令能更好地确保全文间距的统一性。

\subsection{标准间距命令}

IQB-JC 模板定义了以下三个间距命令,涵盖了常见的排版需求:

\begin{table}[!h]
\centering
\caption{间距命令一览表}
\label{tab:spacing-commands}
\begin{tabular}{p{2.5cm}p{1.5cm}p{7cm}}
\toprule
\textbf{命令} & \textbf{间距大小} & \textbf{使用场景} \\
\midrule
\lstinline|\iqbsep| & 0.3cm & 标准段落间距,最常用,用于分隔逻辑内容块 \\
\lstinline|\iqbbigsep| & 0.5cm & 大段落间距,用于主要分节之间的间隔 \\
\lstinline|\iqbtinysep| & 0.15cm & 小间距,用于列表项之间或紧凑内容区间 \\
\bottomrule
\end{tabular}
\end{table}

\subsection{使用示例}

以下示例展示了如何在演示文稿中使用这些间距命令:

\begin{lstlisting}[style=blockstyle,language=TeX]
\begin{frame}{多段落内容页面}
  \iqbsectiontitle{背景介绍}

  这是第一段内容,介绍了该部分的基本背景。

  \iqbsep  % 使用标准间距分隔段落

  \iqbsectiontitle{方法论}

  \begin{itemize}
    \item 方法一
    \iqbtinysep  % 使用小间距分隔列表项
    \item 方法二
    \iqbtinysep
    \item 方法三
  \end{itemize}

  \iqbbigsep  % 使用大间距分隔主要内容块

  \iqbsectiontitle{结论}

  这是最后一段内容...
\end{frame}
\end{lstlisting}

\subsection{间距与图片的配合}

当在演示文稿中插入图片时,常需要在图片与文字之间增加合适的间距。使用标准间距命令能确保整体布局的协调性:

\begin{lstlisting}[style=blockstyle,language=TeX]
\begin{frame}{图文结合布局}
  \iqbsectiontitle{研究成果}

  \iqbsep

  \iqbimgcenter[height=0.4\textheight]{images/result.png}

  \iqbtinysep

  \textbf{图 1:}实验结果展示了该方法的有效性。
\end{frame}
\end{lstlisting}

\subsection{推荐实践}

\begin{enumerate}
  \item \textbf{优先使用标准命令}:用 \lstinline|\iqbsep|、\lstinline|\iqbbigsep|、\lstinline|\iqbtinysep| 代替手动 \lstinline|\vspace{}|,以保持全文间距风格一致。

  \item \textbf{逻辑清晰}:用间距命令清晰地分隔演示文稿中的逻辑块(如不同章节、内容组)。

  \item \textbf{避免过度间距}:不要过度使用大间距(\lstinline|\iqbbigsep|),否则页面显得疏散;也不要过度使用小间距,否则内容显得拥挤。

  \item \textbf{与布局工具结合}:在使用 \lstinline|\iqblayouttwo|、\lstinline|\iqbfig| 等布局命令时,合理添加间距能提升整体视觉效果。
\end{enumerate}

\section{演示文稿元数据设置}

IQB-JC 模板支持丰富的演示文稿元数据字段,这些信息会自动显示在封面页上。

\subsection{基础字段}

以下字段是必填的,用于定义演示文稿的基本信息:

\begin{lstlisting}[style=blockstyle,language=TeX]
\title{演示标题}           % 封面主标题
\subtitle{副标题}           % 可选:副标题
\author{作者名字}           % 作者名字
\institute{机构/学院}       % 学院或研究所名称
\date{\today}               % 日期(\today 表示当前日期)
\end{lstlisting}

\subsection{扩展字段}

模板额外支持以下字段,用于学位申请、奖学金申请等场景。这些字段**完全可选**,如不设置则不显示:

\begin{lstlisting}[style=blockstyle,language=TeX]
\stuid{学号}               % 学号(如果设置则在封面显示)
\major{专业名称}           % 专业或学位类型(如果设置则在封面显示)
\advisor{导师名字 教授}    % 指导教师名字(如果设置则在封面显示)
\end{lstlisting}

使用示例:

\begin{lstlisting}[style=blockstyle,language=TeX]
\title{杨咏曼奖学金申请陈述}
\author{高旭帆}
\institute{生命科学学院 \quad 生物物理研究所}
\stuid{12207134}
\major{生物物理学}
\advisor{周如鸿 教授}
\end{lstlisting}

\subsection{日期格式设置}

模板支持中英文日期格式切换,通过 \lstinline|\dateformat{}| 命令控制:

\begin{lstlisting}[style=blockstyle,language=TeX]
% 使用中文日期格式(推荐用于中文演示)
\dateformat{zh}
\date{\today}           % 显示为:"2025 年 11 月 5 日"

% 使用英文日期格式(推荐用于英文演示)
\dateformat{en}
\date{\today}           % 显示为:"5 November 2025"
\end{lstlisting}

\textbf{注意}:\lstinline|\dateformat| 命令必须在 \lstinline|\date{}| 之前调用。