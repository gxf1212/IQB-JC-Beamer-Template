%!TEX root = document.tex

\section{安装与环境配置}
\label{sec:installation}

\subsection{系统要求}

使用 IQB-JC-Beamer 模板的最低系统要求如表 \ref{tab:requirements} 所示。

\begin{table}[h]
\centering
\caption{IQB-JC-Beamer 系统要求}
\label{tab:requirements}
\begin{tabular}{p{2.5cm}p{3cm}p{8cm}}
\toprule
\textbf{组件} & \textbf{最低版本} & \textbf{说明} \\
\midrule
TeX 发行版 & TeXLive 2022 & 或 MiKTeX 21+、MacTeX 2022+ \\
编译器 & XeLaTeX & 必须支持中文渲染 \\
字体 & 见下表 & Windows/Mac/Linux 需要不同字体 \\
磁盘空间 & 500 MB & 用于 TeX 发行版的安装 \\
内存 & 2 GB & 编译大型演示文稿时的推荐配置 \\
\bottomrule
\end{tabular}
\end{table}

\subsection{依赖包}

IQB-JC-Beamer 的核心功能依赖以下 LaTeX 包。大多数现代 TeX 发行版都已内置这些包,无需单独安装。

\begin{table}[h]
\centering
\caption{IQB-JC-Beamer 依赖包列表}
\label{tab:dependencies}
\begin{tabular}{p{2.5cm}p{4cm}p{6.5cm}}
\toprule
\textbf{包名} & \textbf{用途} & \textbf{功能说明} \\
\midrule
\lstinline|beamer| & 演示框架 & Beamer 文档类和主题系统 \\
\lstinline|xeCJK| & 中文支持 & XeLaTeX 中文排版 \\
\lstinline|tikz| & 绘图 & 高级 PDF 绘图和定位 \\
\lstinline|graphicx| & 图片支持 & 图片包含和缩放 \\
\lstinline|amsmath| & 数学公式 & 扩展的数学环境 \\
\lstinline|amssymb| & 数学符号 & 额外的数学符号 \\
\lstinline|booktabs| & 表格美化 & 专业表格排版 \\
\lstinline|mhchem| & 化学公式 & 化学方程式和符号(可选) \\
\lstinline|ifplatform| & 平台检测 & 操作系统检测(可选) \\
\bottomrule
\end{tabular}
\end{table}

\subsection{获取模板}

\subsubsection{方法一:从 GitHub 克隆}

通过 Git 克隆完整的仓库:

\begin{lstlisting}[style=blockstyle,language=bash]
git clone https://github.com/your-org/IQB-JC.git
cd IQB-JC
\end{lstlisting}

此方法获得完整的项目结构,包括所有示例、文档和版本历史。

\subsubsection{方法二:下载 ZIP 包}

从 GitHub 下载最新的 Release 包:

\begin{lstlisting}[style=blockstyle,language=bash]
# 解压到本地目录
unzip IQB-JC-v1.0.zip
cd IQB-JC-master
\end{lstlisting}

\subsection{文件结构}

成功安装后,项目结构应如下所示:

\begin{lstlisting}[style=blockstyle]
IQB-JC-master/
├── theme/                              # 主题和布局工具包
│   ├── beamerthemeiqb.sty             # 主题定义(颜色、字体、页眉页脚)
│   ├── iqb-layouts.sty                # 布局工具包(50+ 布局和模块)
│   └── images/
│       ├── header.png                 # 页眉横幅图片(1999×204px)
│       └── protein.png                # 装饰图片(可选)
├── template/
│   └── jc-template.tex                # 空白模板(推荐新手使用)
├── examples/
│   ├── membrane-pore-jc.tex           # 完整工作示例
│   ├── images/                        # 示例图片资源
│   └── output/                        # 编译输出目录(PDF)
├── tools/
│   └── extract_pdf_page.py            # PDF 页面提取工具(调试用)
├── software-copyright/                # 文档源
│   ├── 1-introduction.tex             # 章节 1:基本信息
│   ├── 2-installation.tex             # 章节 2:安装指南(本文件)
│   └── 3-usage.tex                    # 章节 3:使用指南
├── README.md                          # 项目简介
└── LICENSE                            # MIT 许可证
\end{lstlisting}

\subsection{平台特定的安装步骤}

\subsubsection{Windows 用户}

\textbf{步骤 1:安装 TeX 发行版}

推荐使用 TeXLive 或 MiKTeX:

\begin{itemize}
    \item \textbf{TeXLive}:下载 \lstinline|texlive2022-20220321.iso| 或更新版本,按照安装向导安装
    \item \textbf{MiKTeX}:访问 \url{https://miktex.org},下载 Windows 安装程序
\end{itemize}

在安装过程中,确保勾选以下选项:
\begin{itemize}
    \item XeLaTeX 编译器
    \item 中文支持包(CJK)
    \item 所有推荐的字体和工具包
\end{itemize}

\textbf{步骤 2:验证安装}

打开命令提示符(cmd)或 PowerShell,运行:

\begin{lstlisting}[style=blockstyle,language=bash]
xelatex --version
\end{lstlisting}

若显示版本信息,则安装成功。

\textbf{步骤 3:检查中文字体}

Windows 系统内置的中文字体(SimSun)通常已安装,无需额外配置。若遇到中文乱码,可检查字体位置:

\begin{lstlisting}[style=blockstyle,language=bash]
# 查看已安装的中文字体
fc-list :lang=zh
\end{lstlisting}

\subsubsection{macOS 用户}

\textbf{步骤 1:安装 MacTeX}

MacTeX 是 macOS 上的 TeXLive 发行版,建议通过 Homebrew 安装:

\begin{lstlisting}[style=blockstyle,language=bash]
brew install mactex
\end{lstlisting}

或者从 \url{https://www.tug.org/mactex/} 下载 DMG 安装程序。

完整安装约 3-4 GB,需要 20 分钟左右。

\textbf{步骤 2:验证安装}

\begin{lstlisting}[style=blockstyle,language=bash]
xelatex --version
\end{lstlisting}

\textbf{步骤 3:检查中文字体}

macOS 系统通常内置 Songti SC 等中文字体。若需使用其他字体,可通过字体簿查看已安装的字体。

若 XeLaTeX 无法找到字体,可指定完整路径:

\begin{lstlisting}[style=blockstyle,language=TeX]
\setCJKmainfont[Path=/Library/Fonts/]{Songti SC}
\end{lstlisting}

\subsubsection{Linux 用户}

\textbf{步骤 1:安装 TeXLive}

\textbf{Ubuntu/Debian:}

\begin{lstlisting}[style=blockstyle,language=bash]
sudo apt-get update
sudo apt-get install texlive-xetex texlive-fonts-recommended
sudo apt-get install fonts-noto-cjk  # 中文字体
\end{lstlisting}

\textbf{Fedora/RHEL:}

\begin{lstlisting}[style=blockstyle,language=bash]
sudo dnf install texlive-xetex texlive-fonts-recommended
sudo dnf install google-noto-sans-cjk-fonts
\end{lstlisting}

\textbf{Arch Linux:}

\begin{lstlisting}[style=blockstyle,language=bash]
sudo pacman -S texlive-xetex texlive-fontsrecommended
sudo pacman -S noto-fonts-cjk
\end{lstlisting}

\textbf{步骤 2:验证安装}

\begin{lstlisting}[style=blockstyle,language=bash]
xelatex --version
fc-list :lang=zh | grep "Noto"  # 检查中文字体
\end{lstlisting}

若字体列表为空,可手动安装字体包。

\subsection{编译器配置}

\subsubsection{配置 XeLaTeX 编译命令}

在集成开发环境(IDE)或编辑器中配置 XeLaTeX 为主要编译器。

\textbf{Visual Studio Code(推荐):}

安装 LaTeX Workshop 扩展,在 \lstinline|settings.json| 中配置:

\begin{lstlisting}[style=blockstyle,language=json]
"latex-workshop.latex.tools": [
  {
    "name": "xelatex",
    "command": "xelatex",
    "args": [
      "-interaction=nonstopmode",
      "-file-line-error",
      "%DOCFILE%"
    ]
  }
],
"latex-workshop.latex.recipes": [
  {
    "name": "xelatex",
    "tools": ["xelatex"]
  }
]
\end{lstlisting}

\textbf{Overleaf(在线):}

项目菜单 → Menu → Compiler → 选择 XeLaTeX

\textbf{TeXStudio:}

选项 → 配置 TeXStudio → Build → Default Compiler → 选择 XeLaTeX

\subsection{快速启动}

\subsubsection{使用空白模板}

最快的开始方式是使用预设的模板:

\begin{lstlisting}[style=blockstyle,language=bash]
# 复制模板到你的工作目录
cp -r template/jc-template.tex my-presentation.tex
cd my-presentation

# 或直接编辑模板
xelatex -interaction=nonstopmode jc-template.tex
\end{lstlisting}

\subsubsection{查看完整示例}

查看工作示例以了解所有功能:

\begin{lstlisting}[style=blockstyle,language=bash]
cd examples
xelatex -interaction=nonstopmode membrane-pore-jc.tex
# 打开 membrane-pore-jc.pdf 查看结果
\end{lstlisting}

\subsection{故障排除}

\subsubsection{编译错误:找不到文件}

\textbf{问题:}\lstinline|! I can't find file `../theme/beamerthemeiqb.sty'|

\textbf{原因:}相对路径不正确

\textbf{解决方案:}
\begin{itemize}
    \item 确认 \lstinline|.tex| 文件与 \lstinline|theme/| 目录的相对关系
    \item 若 \lstinline|.tex| 在 \lstinline|examples/| 目录,路径应为 \lstinline|../theme/beamerthemeiqb|
    \item 若 \lstinline|.tex| 在项目根目录,路径应为 \lstinline|theme/beamerthemeiqb|
\end{itemize}

\subsubsection{中文显示为方框}

\textbf{问题:}编译后 PDF 中中文显示为方框或乱码

\textbf{原因:}
\begin{itemize}
    \item 未使用 XeLaTeX 编译(使用了 pdflatex)
    \item 中文字体未安装或未正确指定
\end{itemize}

\textbf{解决方案:}
\begin{enumerate}
    \item 确认使用 XeLaTeX 编译
    \item 检查字体名称:\lstinline|fc-list :lang=zh|
    \item 在 \lstinline|.tex| 中明确指定字体
\end{enumerate}

\subsubsection{缺少依赖包}

\textbf{问题:}\lstinline|! LaTeX Error: File `tikz.sty' not found|

\textbf{解决方案:}

根据具体缺失的包名,使用包管理器安装。例如,缺少 tikz 包:

\textbf{MiKTeX:}
\begin{lstlisting}[style=blockstyle,language=bash]
tlmgr install pgf
\end{lstlisting}

\textbf{TeXLive:}
\begin{lstlisting}[style=blockstyle,language=bash]
tlmgr install pgf --repository ctan
\end{lstlisting}

\subsection{性能优化}

对于大型演示或多次编译,可采取以下优化措施:

\begin{itemize}
    \item \textbf{启用 SyncTeX}:加速编辑器和 PDF 查看器的同步:\lstinline|xelatex -synctex=1|
    \item \textbf{使用草稿模式}:加快编译速度(最终提交前移除):\lstinline|\documentclass[draft]{beamer}|
    \item \textbf{优化图片尺寸}:使用合适分辨率的图片而非超高分辨率
    \item \textbf{缓存 TikZ 输出}:使用 \lstinline|tikzexternalize| 库缓存复杂绘图
\end{itemize}

\subsection{下一步}

安装完成后,建议按以下步骤继续:

\begin{enumerate}
    \item 阅读第 \ref{sec:usage} 章了解基本用法
    \item 查看 \lstinline|template/jc-template.tex| 的配置示例
    \item 运行 \lstinline|examples/membrane-pore-jc.tex| 查看完整功能演示
    \item 根据需要自定义主题和布局
\end{enumerate}