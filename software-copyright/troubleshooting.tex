%!TEX root = document.tex

\chapter{故障排除与最佳实践}
\label{chap:troubleshooting}

本章为用户提供了 IQB-JC 模板的故障排除指南和最佳实践建议。通过系统化的问题诊断和解决方案,帮助用户快速解决使用过程中遇到的各类问题,并掌握高效使用模板的技巧。

\section{故障排除与常见问题}

本节列举用户使用过程中的常见问题及解决方案。

\subsection{编译错误}

\textbf{问题 1:文件找不到 (File not found)}

症状:编译时出现 \lstinline|! I can't find file `../theme/beamerthemeiqb.sty'|

原因:相对路径不正确。该错误通常发生在文件结构与模板预期不符时。

解决方案:
\begin{enumerate}
\item 确认目录结构与模板匹配,如下所示:
\begin{lstlisting}[style=blockstyle]
my-presentation/
├── examples/
│   └── my-jc.tex
├── theme/
│   ├── beamerthemeiqb.sty
│   ├── iqb-layouts.sty
│   └── images/
└── ...
\end{lstlisting}

\item 或在导言区调整相对路径,例如:
\begin{lstlisting}[style=blockstyle,language=TeX]
% 若 .tex 文件在项目根目录
\usepackage{theme/beamerthemeiqb}

% 若 .tex 文件在 examples/ 目录
\usepackage{../theme/beamerthemeiqb}

% 绝对路径(不推荐)
\usepackage{/home/user/my-project/theme/beamerthemeiqb}
\end{lstlisting}
\end{enumerate}

\textbf{问题 2:中文乱码或无法显示}

症状:编译后 PDF 中中文显示为方框或乱码

原因:未使用 XeLaTeX 编译,或未正确配置中文字体

解决方案:
\begin{enumerate}
\item 确认使用 XeLaTeX 编译(非 pdflatex):
\begin{lstlisting}[style=blockstyle,language=bash]
xelatex -interaction=nonstopmode your-file.tex
\end{lstlisting}

\item 检查字体配置:
\begin{lstlisting}[style=blockstyle,language=TeX]
\usepackage{xeCJK}
% Windows 用户
\setCJKmainfont{SimSun}

% macOS 用户
\setCJKmainfont{Songti SC}

% Linux 用户(需先安装 fonts-noto-cjk)
\setCJKmainfont{Noto Serif CJK SC}
\end{lstlisting}

\item Linux 用户若字体缺失,安装字体包:
\begin{lstlisting}[style=blockstyle,language=bash]
sudo apt-get install fonts-noto-cjk
\end{lstlisting}
\end{enumerate}

\textbf{问题 3:化学公式显示错误}

症状:使用 \lstinline|\ce{}| 出现 \lstinline|Undefined control sequence| 错误

原因:未加载 \lstinline|mhchem| 包

解决方案:在导言区添加:
\begin{lstlisting}[style=blockstyle,language=TeX]
\usepackage{mhchem}
\end{lstlisting}

然后使用 \lstinline|\ce{}| 命令:
\begin{lstlisting}[style=blockstyle,language=TeX]
\ce{H2O}        % 水
\ce{Ca2+}       % 钙离子
\ce{Na-Cl}      % 离子键
\end{lstlisting}

\subsection{布局与排版问题}

\textbf{问题 4:内容超出页面 (Overfull hbox/vbox)}

症状:编译时出现警告 \lstinline|Overfull \hbox| 或 \lstinline|Overfull \vbox|

原因:某一行或某一页的内容超过了允许的宽度或高度

解决方案:
\begin{enumerate}
\item 减少文字内容,提炼关键点
\item 调整图片大小,例如:
\begin{lstlisting}[style=blockstyle,language=TeX]
% 超过高度,改用更小的图片
\iqbfig[height=0.4\textheight]{image.png}{图注}

% 或拆分为两页
\end{lstlisting}

\item 使用更紧凑的布局,如三列而不是两列:
\begin{lstlisting}[style=blockstyle,language=TeX]
% 改为三列(节省空间)
\iqblayoutthree{内容1}{内容2}{内容3}
\end{lstlisting}

\item 启用详细的编译报告以定位具体行数:
\begin{lstlisting}[style=blockstyle,language=bash]
xelatex --file-line-error your-file.tex
\end{lstlisting}
\end{enumerate}

\textbf{问题 5:页眉或页脚不显示}

症状:某些页面缺少 header/footer

原因:使用了 \lstinline|[plain]| 或 \lstinline|[plain,noframenumbering]| 选项

这是正常的——这两个选项用于隐藏 header/footer(用于封面和致谢页)。

若要显示 header/footer,移除 \lstinline|[plain]| 选项:
\begin{lstlisting}[style=blockstyle,language=TeX]
% 错误(隐藏 header/footer)
\begin{frame}[plain]{标题}
  内容
\end{frame}

% 正确(显示 header/footer)
\begin{frame}{标题}
  内容
\end{frame}
\end{lstlisting}

\textbf{问题 6:图片不居中或对齐错误}

症状:图片在列中向左对齐或大小不符预期

原因:布局命令的参数设置或图片大小参数不当

解决方案:
\begin{enumerate}
\item 为列中的内容添加 \lstinline|\centering|:
\begin{lstlisting}[style=blockstyle,language=TeX]
\iqblayouttwo{
  \centering
  \includegraphics[height=0.5\textheight]{img1.png}
}{
  \centering
  \includegraphics[height=0.5\textheight]{img2.png}
}
\end{lstlisting}

\item 确保使用 \lstinline|height| 而不是 \lstinline|width|(宽度会导致比例失调):
\begin{lstlisting}[style=blockstyle,language=TeX]
% 正确:指定高度,宽度自动计算
\includegraphics[height=0.5\textheight,keepaspectratio]{image.png}

% 错误:只指定宽度可能拉伸图片
\includegraphics[width=\textwidth]{image.png}
\end{lstlisting}
\end{enumerate}

\subsection{功能与命令问题}

\textbf{问题 7:图片计数器不递增(使用 \texttt{iqbfig} 系列命令)}

症状:多个图片都显示"图1"而不是递增编号

原因:图片计数器未正确初始化或重置

解决方案:若要重置计数器(例如在新 section 开始),使用:
\begin{lstlisting}[style=blockstyle,language=TeX]
% 在新 section 开始处
\setcounter{iqbfigure}{0}

% 然后继续使用 \iqbfig 命令
\iqbfig{...}{...}  % 会重新从 图1 开始计数
\end{lstlisting}

\textbf{问题 8:自定义颜色不生效}

症状:\lstinline|\textcolor{iqbblue}{...}| 无法改变文字颜色

原因:颜色未在主题中定义

解决方案:检查 \lstinline|beamerthemeiqb.sty| 中的颜色定义,或自定义新颜色:
\begin{lstlisting}[style=blockstyle,language=TeX]
% 已定义的颜色(可直接使用)
% iqbblue, iqbdarkblue, iqblightblue, iqbgray, iqborange, iqbgreen, iqbred

% 自定义新颜色
\usepackage{xcolor}
\definecolor{mycolor}{RGB}{255,128,0}

% 然后使用
\textcolor{mycolor}{文字}
\end{lstlisting}

\textbf{问题 9:间距命令不起作用}

症状:使用 \lstinline|\iqbsep| 等间距命令但间距不变

原因:命令使用位置不当(例如在 block 内部或列环境边界处)

解决方案:
\begin{enumerate}
\item 在段落之间使用(而非在环境开始/结束处):
\begin{lstlisting}[style=blockstyle,language=TeX]
% 正确
\begin{itemize}
  \item 第一点
\end{itemize}
\iqbsep
\begin{block}{标题}
  内容
\end{block}

% 错误
\iqbsep
\begin{block}{...}
  ...
\end{block}
\end{lstlisting}

\item 若要调整 block 内部间距,使用标准的 LaTeX 命令:
\begin{lstlisting}[style=blockstyle,language=TeX]
\begin{block}{标题}
  \begin{itemize}
    \item 项 1
    \itemsep=0.1em  % 项目间距
    \item 项 2
  \end{itemize}
\end{block}
\end{lstlisting}
\end{enumerate}

\subsection{性能与编译速度}

\textbf{问题 10:编译速度很慢}

症状:每次编译需要 10 秒以上

原因:
\begin{itemize}
\item 加载了过多的图片(尤其是高分辨率图片)
\item 使用了复杂的 TikZ 绘图
\item 多次编译以更新交叉引用
\end{itemize}

解决方案:
\begin{enumerate}
\item 优化图片大小(使用合适的分辨率而非超高分辨率)
\item 若大量使用 TikZ,考虑使用 \lstinline|externalize| 库缓存 TikZ 输出:
\begin{lstlisting}[style=blockstyle,language=TeX]
\usepackage{tikz}
\usetikzlibrary{external}
\tikzexternalize[prefix=tikz/]
\end{lstlisting}

\item 只在最终版本时进行双重编译;开发阶段单次编译即可
\end{enumerate}

\subsection{其他常见问题}

\textbf{问题 11:主题颜色或字体无法修改}

若需完全自定义主题颜色或字体,编辑 \lstinline|theme/beamerthemeiqb.sty|:

\begin{lstlisting}[style=blockstyle,language=TeX]
% 修改 IQB 主题色(原值 #003366)
\definecolor{iqbblue}{RGB}{0, 51, 102}  % 改为你的颜色

% 修改标题字体大小
\setbeamerfont{frametitle}{size=\large, series=\bfseries}
\end{lstlisting}

然后重新编译演示文稿。

\textbf{问题 12:无法使用 \texttt{iqbauthorstwophoto} 加载图片}

症状:编译时出现图片路径错误

原因:图片路径不正确

解决方案:确保相对路径正确,例如:
\begin{lstlisting}[style=blockstyle,language=TeX]
% 若 .tex 在 examples/ 目录,图片在 examples/images/ 目录
\iqbauthorstwophoto{images/author1.jpg}{...}{...}{...}{...}{...}{...}{...}{...}

% 或使用绝对路径
\iqbauthorstwophoto{/absolute/path/to/author1.jpg}{...}{...}{...}{...}{...}{...}{...}{...}
\end{lstlisting}

\section{获取帮助}

若上述解决方案未能解决问题,建议:

\begin{enumerate}
\item 查看完整示例:\lstinline|examples/membrane-pore-jc.tex|
\item 查看模板源代码中的注释:
\begin{itemize}
  \item \lstinline|theme/beamerthemeiqb.sty|:主题相关问题
  \item \lstinline|theme/iqb-layouts.sty|:布局相关问题
  \item \lstinline|template/jc-template.tex|:最小示例和配置
\end{itemize}
\item 参考 Beamer 官方文档:\url{https://ctan.org/pkg/beamer}
\item 参考 LaTeX Wikibooks 中文排版:\url{https://en.wikibooks.org/wiki/LaTeX}
\end{enumerate}

\section{最佳实践指南}

\subsection{模板使用最佳实践}

\textbf{1. 文档组织结构}

推荐的项目目录结构:
\begin{lstlisting}[style=blockstyle]
project-root/
├── main.tex               % 主文档
├── theme/                 % 主题文件(从模板复制)
│   ├── beamerthemeiqb.sty
│   ├── iqb-layouts.sty
│   └── images/
├── images/               % 项目图片
├── sections/              % 各章节内容
│   ├── introduction.tex
│   ├── methods.tex
│   └── results.tex
└── references.bib        % 参考文献
\end{lstlisting}

\textbf{2. 编译工作流}

推荐的编译流程:
\begin{enumerate}
\item \textbf{开发阶段}:单次编译,快速验证
\begin{lstlisting}[style=blockstyle,language=bash]
xelatex main.tex
\end{lstlisting}

\item \textbf{最终版本}:双重编译,确保引用正确
\begin{lstlisting}[style=blockstyle,language=bash]
xelatex main.tex
xelatex main.tex
\end{lstlisting}

\item \textbf{包含参考文献}:使用 bibTeX/biber
\begin{lstlisting}[style=blockstyle,language=bash]
xelatex main.tex
biber main
xelatex main.tex
xelatex main.tex
\end{lstlisting}
\end{enumerate}

\textbf{3. 图片管理最佳实践}

\begin{itemize}
\item \textbf{图片格式}:优先使用 PNG(矢量图用 PDF)
\item \textbf{图片尺寸}:控制分辨率,避免过大文件
\item \textbf{图片路径}:使用相对路径,便于移植
\item \textbf{图片命名}:采用有意义的命名,如 \lstinline|method_comparison.png|
\end{itemize}

\textbf{4. 内容组织最佳实践}

\begin{itemize}
\item \textbf{页面标题}:每页都有明确的 punchline 或结论概括
\item \textbf{内容密度}:每页控制在 10-12 行以内
\item \textbf{图文结合}:尽量每页都有图片配合文字说明
\item \textbf{渐进式呈现}:使用 \lstinline|\pause| 或渐进式 itemize
\end{itemize}

\subsection{布局选择最佳实践}

\textbf{1. 根据内容类型选择布局}

表 \ref{tab:layout-selection} 提供了在不同内容类型下的布局选择建议。

\begin{table}[!h]
\centering
\caption{不同内容类型的布局选择}
\label{tab:layout-selection}
\begin{tabular}{p{3cm}p{3cm}p{6cm}}
\toprule
\textbf{内容类型} & \textbf{推荐布局} & \textbf{说明} \\
\midrule
并行对比 & 双列 50-50 & 两个等权内容 \\
图文结合 & 1/3+2/3 & 竖版图片 + 说明 \\
多个独立内容 & 三列或网格 & 3 个或 4 个内容 \\
方法流程 & 时间线 & 流程步骤展示 \\
关键信息 & 关键要点 & 突出核心发现 \\
\bottomrule
\end{tabular}
\end{table}

\textbf{2. 图片布局策略}

\begin{itemize}
\item \textbf{竖版图片}(高>宽):必须使用横向 column 布局,图片占一列,文字占另一列
\item \textbf{宽版图片}(宽≥高×1.5):可使用 2/3+1/3 布局,图片占大部分
\item \textbf{标准图片}:根据内容重要性决定图片占比
\item \textbf{多图片对比}:使用并排布局(双图、三图或网格)
\end{itemize}

\textbf{3. 文字排版最佳实践}

\begin{itemize}
\item \textbf{字体大小}:正文使用 \lstinline|\scriptsize|,标题使用 \lstinline|\large|
\item \textbf{行间距}:保持 150\% point size,增加可读性
\item \textbf{对齐方式}:所有文字左对齐,沿同一左边距
\item \textbf{换行控制}:手动使用 \lstinline|\\| 断行,避免自动换行
\end{itemize}

\subsection{性能优化最佳实践}

\textbf{1. 编译速度优化}

\begin{enumerate}
\item \textbf{优化图片大小}
\begin{itemize}
\item 使用适当的分辨率,避免过高 DPI
\item 压缩图片文件,减少文件大小
\item 预处理图片,避免 LaTeX 内部调整
\end{itemize}

\item \textbf{减少复杂内容}
\begin{itemize}
\item 简化 TikZ 图形
\item 避免过多的数学公式
\item 减少复杂的表格
\end{itemize}

\item \textbf{使用编译缓存}
\begin{itemize}
\item 启用 TikZ externalize 库
\item 使用 \lstinline|.aux| 文件重用
\item 避免不必要的重新编译
\end{itemize}
\end{enumerate}

\textbf{2. 内存管理}

\begin{itemize}
\item 使用 \lstinline|\includeonly| 选择性编译章节
\item 分割大型文档为多个文件
\item 定期清理临时文件
\end{itemize}

\section{版本控制与协作}

\subsection{Git 工作流最佳实践}

\textbf{1. 推荐的 Git 配置}

\begin{lstlisting}[style=blockstyle,language=bash]
# .gitignore 示例
*.aux
*.log
*.out
*.toc
*.snm
*.nav
*.fls
*.fdb_latexmk
*.synctex.gz
*.bbl
*.blg
*.run.xml
*.bcf
*.lof
*.lot
.tikz/
\end{lstlisting}

\textbf{2. 分支策略}

\begin{itemize}
\item \textbf{main/master}:稳定版本,仅发布时更新
\item \textbf{develop}:开发分支,日常开发
\item \textbf{feature/xxx}:功能分支,开发新特性
\item \textbf{fix/xxx}:修复分支,修复问题
\end{itemize}

\textbf{3. 提交信息规范}

遵循 Conventional Commits 规范:
\begin{itemize}
\item \textbf{feat:} 新功能
\item \textbf{fix:} 修复问题
\item \textbf{docs:} 文档更新
\item \textbf{style:} 代码格式化
\item \textbf{refactor:} 重构
\item \textbf{test:} 测试相关
\end{itemize}

示例:
\begin{lstlisting}[style=blockstyle,language=bash]
feat: add three-column comparison layout
fix: resolve header image alignment issue
docs: update usage documentation for v2.0
\end{lstlisting}

\subsection{团队协作最佳实践}

\textbf{1. 代码审查}

\begin{itemize}
\item 所有关键更改都需要 Pull Request
\item 代码审查关注点:
  \begin{itemize}
  \item 功能完整性
  \item 代码风格一致性
  \item 文档更新
  \item 测试覆盖
  \end{itemize}
\item 及时响应审查意见
\end{itemize}

\textbf{2. 文档管理}

\begin{itemize}
\item 保持文档与代码同步更新
\item 使用统一的文档风格
\item 提供清晰的使用示例
\item 记录重要更改历史
\end{itemize}

\section{后续发展计划}

\subsection{Phase 2 后续计划}

未来版本计划添加:
\begin{itemize}
\item \textbf{数据可视化增强}:集成 \lstinline|pgfplots| 用于快速绘制科学图表
\item \textbf{交互式目录}:自动生成章节导航和进度指示
\item \textbf{多主题支持}:提供暗色、学术、企业等预设主题
\item \textbf{参考文献管理}:集成 BibTeX 和自动引文格式化
\item \textbf{媒体集成}:支持视频、音频和动画嵌入
\end{itemize}

\subsection{社区贡献指南}

我们鼓励用户参与模板的开发和改进:

\begin{enumerate}
\item \textbf{报告问题}:通过 GitHub Issues 报告 bug 或提出建议
\item \textbf{贡献代码}:提交 Pull Request 贡献新功能
\item \textbf{分享示例}:分享使用模板创建的优秀演示文稿
\item \textbf{编写文档}:帮助改进和完善文档
\end{enumerate}

\subsection{联系方式与支持}

\begin{itemize}
\item \textbf{项目主页}:GitHub 仓库
\item \textbf{问题报告}:GitHub Issues
\item \textbf{功能请求}:GitHub Discussions
\item \textbf{邮件联系}:维护者邮箱
\end{itemize}

通过遵循本章提供的故障排除指南和最佳实践,用户可以更高效地使用 IQB-JC 模板,创建出专业、美观的学术演示文稿。