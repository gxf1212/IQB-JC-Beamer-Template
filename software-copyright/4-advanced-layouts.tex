%!TEX root = document.tex

\chapter{高级布局与快捷命令}
\label{chap:advanced-layouts}

本章详细介绍 IQB-JC 模板的高级布局模块和快捷命令系统,帮助用户更高效地创建复杂的学术演示文稿。这些高级功能包括自动编号图表模块、作者信息模块、智能强调系统和一系列代码简化命令。

\section{图片局部居中}

\subsection{问题描述}

在多列布局中(如 \lstinline|\iqblayouttwo| 等),使用 \lstinline|\centering| 命令会影响整个列的后续文字对齐方式,导致本意只想居中图片,结果文字也被迫居中。

\subsection{解决方案:\texttt{\textbackslash iqbimgcenter}}

使用 \lstinline|\iqbimgcenter| 命令可以实现图片局部居中,不影响后续文字的对齐。该命令使用 \lstinline|\begingroup...\endgroup| 来局限 \lstinline|\centering| 的作用范围。

命令格式:

\begin{lstlisting}[style=blockstyle,language=TeX]
\iqbimgcenter[图片选项]{图片路径}
\end{lstlisting}

参数说明:
\begin{itemize}
  \item \lstinline|[图片选项]| - 可选,图片尺寸设置,如 \lstinline|width=0.5\textwidth|、\lstinline|height=0.6\textheight| 等
  \item \lstinline|{图片路径}| - 必需,图片的相对或绝对路径
\end{itemize}

使用示例:

\begin{lstlisting}[style=blockstyle,language=TeX]
\iqblayouttwo{
  \iqbimgcenter[width=0.5\textwidth]{images/photo.jpg}
  \iqbtinysep
  \textbf{说明文字:}
  \begin{itemize}
    \item 文字项目1
    \item 文字项目2
  \end{itemize}
}{
  右列内容...
}
\end{lstlisting}

效果:图片会在列内居中显示,而下方的"说明文字"仍然保持左对齐。

\subsection{与 \texttt{center} 环境的对比}

\begin{table}[!h]
\centering
\caption{图片居中方式对比}
\label{tab:imgcenter-compare}
\begin{tabular}{p{3cm}p{3.5cm}p{6cm}}
\toprule
\textbf{方式} & \textbf{使用方式} & \textbf{特点} \\
\midrule
\lstinline|\iqbimgcenter| & \lstinline|\iqbimgcenter[opts]{path}| & 局部生效,不影响后续文字 \\
\lstinline|\centering| & \lstinline|\centering\includegraphics...| & 全局生效,影响后续所有文字 \\
\lstinline|\begin{center}| & \lstinline|\begin{center}\includegraphics\end{center}| & 局部生效,独占一行 \\
\bottomrule
\end{tabular}
\end{table}

\textbf{推荐}:在多列布局中,优先使用 \lstinline|\iqbimgcenter|;在简单场景中,\lstinline|\begin{center}| 也很有效。

\section{图表模块(自动编号)}

IQB-JC 模板提供了带自动编号和统一格式的图表模块。所有图表自动生成"图X:"格式的编号,caption 左对齐显示。

\subsection{单图模块}

命令格式:

\begin{lstlisting}[style=blockstyle,language=TeX]
\iqbfig[height=0.5\textheight]{图片路径}{图说文字}
\end{lstlisting}

参数说明见表 \ref{tab:iqbfig-params}。

\begin{table}[!h]
\centering
\caption{iqbfig 命令参数}
\label{tab:iqbfig-params}
\begin{tabular}{p{2.5cm}p{2.5cm}p{8cm}}
\toprule
\textbf{参数} & \textbf{默认值} & \textbf{说明} \\
\midrule
\lstinline|[选项]| & \lstinline|height=0.5\textheight| & 图片尺寸,支持 \lstinline|height| 或 \lstinline|width| \\
图片路径 & 必需 & 相对或绝对路径,支持 PNG/PDF \\
图说文字 & 必需 & 图片下方的说明文字,自动编号 \\
\bottomrule
\end{tabular}
\end{table}

使用示例:

\begin{lstlisting}[style=blockstyle,language=TeX]
\begin{frame}{单图展示}
  \iqbfig[height=0.55\textheight]{results.png}{
    结果展示图:实验数据对比分析
  }
\end{frame}
\end{lstlisting}

\subsection{双图模块}

命令格式:

\begin{lstlisting}[style=blockstyle,language=TeX]
\iqbtwofig[height=0.45\textheight]{img1}{caption1}{img2}{caption2}
\end{lstlisting}

该模块将两张图片并排显示,各占 48\% 宽度。使用示例:

\begin{lstlisting}[style=blockstyle,language=TeX]
\begin{frame}{双图对比}
  \iqbtwofig[height=0.5\textheight]{
    method_a.png
  }{
    方法 A 的结果展示
  }{
    method_b.png
  }{
    方法 B 的结果展示
  }
\end{frame}
\end{lstlisting}

\subsection{三图模块}

命令格式:

\begin{lstlisting}[style=blockstyle,language=TeX]
\iqbthreefig[height=0.35\textheight]{img1}{cap1}{img2}{cap2}{img3}{cap3}
\end{lstlisting}

三张图片均匀分布在一行,各占 31\% 宽度。该布局适合展示三个相关的分析结果。

\subsection{四图网格模块}

命令格式:

\begin{lstlisting}[style=blockstyle,language=TeX]
\iqbfourfig[height=0.35\textheight]{
  img1}{cap1}{img2}{cap2}{img3}{cap3}{img4}{cap4}
\end{lstlisting}

四张图片排列为 2×2 网格。该布局提供清晰的图表组织方式,适合展示多阶段或多方法的结果对比。

\section{作者信息模块}

IQB-JC 模板提供了专门的作者信息展示模块,适用于文献汇报时介绍论文作者的信息。

\subsection{双作者模块(无照片)}

命令格式:

\begin{lstlisting}[style=blockstyle,language=TeX]
\iqbauthorstwo{
  通讯作者姓名}{通讯作者单位}{通讯作者网址}{通讯作者研究方向}{
  第一作者姓名}{第一作者单位}{第一作者网址}{第一作者研究方向
}
\end{lstlisting}

该模块以 50-50 双列布局展示通讯作者和第一作者的信息。使用示例:

\begin{lstlisting}[style=blockstyle,language=TeX]
\begin{frame}{作者信息}
  \iqbauthorstwo{
    Prof. John Smith
  }{
    Department of Biology\\
    University of Example
  }{
    https://example.edu/smith
  }{
    Molecular dynamics, Protein folding
  }{
    Dr. Jane Doe
  }{
    Institute of Computational Science\\
    Example University
  }{
    https://example.edu/doe
  }{
    MD simulations, Free energy calculations
  }
\end{frame}
\end{lstlisting}

\subsection{双作者模块(带照片)}

命令格式:

\begin{lstlisting}[style=blockstyle,language=TeX]
\setauthorfirstfield{第一作者研究领域}
\iqbauthorstwophoto{
  通讯作者照片路径}{通讯作者姓名}{...}{...}{通讯作者研究方向}{
  第一作者照片路径}{第一作者姓名}{...}{...}
}
\end{lstlisting}

使用 \lstinline|\setauthorfirstfield{}| 命令单独设置第一作者的专业领域。使用示例:

\begin{lstlisting}[style=blockstyle,language=TeX]
\begin{frame}{论文作者}
  \setauthorfirstfield{计算生物物理、膜蛋白模拟、自由能计算}
  \iqbauthorstwophoto{
    author1.jpg
  }{
    Prof. Alice Brown
  }{
    College of Science\\
    Tech University
  }{
    https://example.org/alice
  }{
    Protein structure prediction, Force field development
  }{
    author2.jpg
  }{
    Dr. Bob White
  }{
    Center for Molecular Dynamics\\
    Research Institute
  }{
    https://example.org/bob
  }
\end{frame}
\end{lstlisting}

\subsection{单作者模块}

命令格式:

\begin{lstlisting}[style=blockstyle,language=TeX]
\iqbauthorone{作者姓名}{作者单位}{作者网址}{研究方向}
\end{lstlisting}

该模块适用于单作者论文或特邀评论文章。信息居中显示。

\section{Punchline 智能强调系统}

\label{sec:punchline}

IQB-JC 模板提供了灵活的 Punchline 系统,用于在幻灯片页面中突出核心观点、关键发现或学术申请陈述的重点内容。该系统支持多种颜色、对齐方式和框风格的组合,满足不同的表达需求。

\subsection{基础 Punchline 命令}

\textbf{简单指定颜色}

命令格式:

\begin{lstlisting}[style=blockstyle,language=TeX]
\iqbpunchline[颜色]{文本}
\end{lstlisting}

默认蓝色,文本自动居中。使用示例:

\begin{lstlisting}[style=blockstyle,language=TeX]
\iqbpunchline[iqbgreen]{计算驱动创新,精准设计引领药物研发新方向!}

\iqbpunchline[iqborange]{警示:需要特别关注的重要发现}
\end{lstlisting}

\subsection{色彩快捷命令(推荐)}

提供三种颜色的快捷命令,对应不同的内容类型:

\textbf{蓝色 Punchline(核心观点、主要发现)}

\begin{lstlisting}[style=blockstyle,language=TeX]
% 最简单:默认无框,居中对齐
\iqbpunchlineblue{核心观点文本}

% 左对齐(适合多行文本)
\iqbpunchlineblue[l]{
  多行文本行 1\\
  多行文本行 2
}

% 有框,居中对齐
\iqbpunchlineblue[b]{有框的蓝色 punchline}

% 有框,左对齐(多行+框)
\iqbpunchlineblue[bl]{
  多行文本:\\
  第一行\\
  第二行
}
\end{lstlisting}

\textbf{绿色 Punchline(成功验证、正面结果)}

\begin{lstlisting}[style=blockstyle,language=TeX]
% 默认:无框,居中对齐
\iqbpunchlinegreen{验证通过:实验结果符合预期}

% 有框,居中对齐
\iqbpunchlinegreen[b]{成功案例:有框显示}

% 长文本左对齐(推荐多行内容)
\iqbpunchlinegreen[l]{PKNAP整合FEPaML、NEOM、MultiTAP等多个肿瘤免疫服务器,\\
融合物理、知识与AI,与FEbuilder、PolyglotMol等形成完整计算药物设计平台!}
\end{lstlisting}

\textbf{橙色 Punchline(警示、挑战、需要关注)}

\begin{lstlisting}[style=blockstyle,language=TeX]
\iqbpunchlineorange{警示:需要关注的重要事项}

\iqbpunchlineorange[l]{多行警告\\
需要特别注意}
\end{lstlisting}

\textbf{参数说明}

表 \ref{tab:punchline-params} 详细说明了可选参数及其含义。

\begin{table}[!h]
\centering
\caption{Punchline 快捷命令参数}
\label{tab:punchline-params}
\begin{tabular}{p{2cm}p{3cm}p{7cm}}
\toprule
\textbf{参数} & \textbf{效果} & \textbf{适用场景} \\
\midrule
空(默认) & 无框,居中对齐 & 单行短文本 \\
\lstinline|[l]| & 无框,左对齐 & 多行长文本 \\
\lstinline|[b]| & 有框,居中对齐 & 需要突出强调的单行 \\
\lstinline|[bl]| 或 \lstinline|[lb]| & 有框,左对齐 & 多行文本需要框突显 \\
\bottomrule
\end{tabular}
\end{table}

\subsection{向后兼容命令}

为了兼容之前的代码,以下命令仍然可用(等价于 \lstinline|[b]| 参数版本):

\begin{lstlisting}[style=blockstyle,language=TeX]
\iqbpunchlineboxgreen{有框的绿色 punchline}

\iqbpunchlineboxorange{有框的橙色 punchline}
\end{lstlisting}

\section{内容模块}

\subsection{关键要点模块}

命令格式:

\begin{lstlisting}[style=blockstyle,language=TeX]
\iqbkeypoints{
  \begin{itemize}
    \item 要点 1
    \item 要点 2
  \end{itemize}
}
\end{lstlisting}

该模块将内容包装在突出显示的 block 环境中,背景色为 IQB 浅灰色,标题为"关键要点"。适合提炼每页的核心内容。使用示例:

\begin{lstlisting}[style=blockstyle,language=TeX]
\begin{frame}{研究成果}
  \iqbkeypoints{
    \begin{itemize}
      \item 发现了新的分子动力学计算方法
      \item 提高了计算精度 50\%
      \item 减少了计算时间 30\%
    \end{itemize}
  }
\end{frame}
\end{lstlisting}

\subsection{核心问题模块}

命令格式:

\begin{lstlisting}[style=blockstyle,language=TeX]
\iqbquestion{该项研究的核心问题是什么?}
\end{lstlisting}

该模块突出显示研究的核心科学问题,文字居中排列。适合在研究背景部分强调关键问题。

\subsection{结论模块}

命令格式:

\begin{lstlisting}[style=blockstyle,language=TeX]
\iqbconclusion{
  通过 XX 方法,我们发现了 YY 现象,为 ZZ 领域的发展奠定了基础。
}
\end{lstlisting}

该模块用于总结研究的主要结论,适合放在讨论或总结页面。文字自动居中显示。

\section{高级功能模块}

\subsection{公式与解释布局}

命令格式:

\begin{lstlisting}[style=blockstyle,language=TeX]
\iqbformulaexplain{公式内容}{公式右侧解释文字}
\end{lstlisting}

该模块采用 1/3 + 2/3 布局,左侧放置公式(居中),右侧放置解释说明。使用示例:

\begin{lstlisting}[style=blockstyle,language=TeX]
\begin{frame}{关键公式推导}
  \iqbformulaexplain{
    $\Delta G = \Delta H - T\Delta S$
  }{
    \textbf{含义:}
    \begin{itemize}
      \item $\Delta G$ 是自由能变化
      \item $\Delta H$ 是焓变
      \item $T\Delta S$ 是熵项贡献
    \end{itemize}
  }
\end{frame}
\end{lstlisting}

\subsection{时间线/流程图模块}

命令格式:

\begin{lstlisting}[style=blockstyle,language=TeX]
\iqbtimeline{步骤1标题}{步骤1内容}{步骤2标题}{步骤2内容}{
  步骤3标题}{步骤3内容}
\end{lstlisting}

该模块创建一个三步的流程图,使用 TikZ 绘制连接的方框。每个方框包含标题和内容,步骤间有箭头连接。使用示例:

\begin{lstlisting}[style=blockstyle,language=TeX]
\begin{frame}{研究方法流程}
  \iqbtimeline{
    数据采集}{收集 1000 个样本}{
    数据处理}{质量控制与预处理}{
    统计分析}{方差分析与回归
  }
\end{frame}
\end{lstlisting}

\subsection{双列对比模块}

命令格式:

\begin{lstlisting}[style=blockstyle,language=TeX]
\iqbcomparetwo{标题 A}{内容 A}{标题 B}{内容 B}
\end{lstlisting}

该模块在双列布局中展示两个方法或方案的对比。每列包含标题和内容,自动对齐。

\subsection{三列对比模块}

命令格式:

\begin{lstlisting}[style=blockstyle,language=TeX]
\iqbcomparethree{标题A}{内容A}{标题B}{内容B}{标题C}{内容C}
\end{lstlisting}

该模块采用三列布局展示三个方法或方案的对比分析。

\subsection{精确定位(高级)}

对于需要自定义精确布局的情况,模板提供了底层的定位工具。

\textbf{文本精确定位}

命令格式:

\begin{lstlisting}[style=blockstyle,language=TeX]
\iqbplace{x}{y}{宽度}{内容}
\end{lstlisting}

其中坐标 (0,0) 为页面左上角,x 和 y 单位为 cm。使用示例:

\begin{lstlisting}[style=blockstyle,language=TeX]
\begin{frame}{}
  \iqbplace{2}{3}{10cm}{
    自定义位置的文本内容
  }
\end{frame}
\end{lstlisting}

\textbf{图片精确定位}

命令格式:

\begin{lstlisting}[style=blockstyle,language=TeX]
\iqbplaceimage{x}{y}{宽度}{高度}{图片路径}
\end{lstlisting}

\section{Phase 1: 高级快捷命令(Advanced Shortcuts)}

\subsection{概述}

为了进一步简化用户编写,模板在 \lstinline|iqb-layouts.sty| 中提供了一系列高级宏命令。这些命令将常见的Frame模式标准化和模板化,显著减少代码重复,提升编写效率。所有新命令与现有代码完全兼容,用户可以渐进式采用。

\subsection{Frame 快捷命令}

\textbf{标准 Frame}

命令格式:

\begin{lstlisting}[style=blockstyle,language=TeX]
\iqbframe{标题}{内容}
\end{lstlisting}

用于快速创建标准单栏 frame,相比原始写法减少 67\% 的代码。

\textbf{文字+图片 Frame(左文右图)}

命令格式:

\begin{lstlisting}[style=blockstyle,language=TeX]
\iqbframetextfig[高度]{标题}{文字内容}{图片路径}
\end{lstlisting}

自动应用 1/3 左 + 2/3 右的布局。可选参数 \lstinline|[高度]| 默认为 \lstinline|0.55\textheight|。

使用示例:

\begin{lstlisting}[style=blockstyle,language=TeX]
\iqbframetextfig{膜孔形成的重要性}{
  \textbf{应用领域}:
  \begin{itemize}
    \item 抗菌肽设计
    \item 药物递送系统
    \item 细胞通透性调控
  \end{itemize}
}{images/membrane-pore.png}
\end{lstlisting}

\textbf{图片+文字 Frame(左图右文)}

命令格式:

\begin{lstlisting}[style=blockstyle,language=TeX]
\iqbframefigttext[高度]{标题}{图片路径}{文字内容}
\end{lstlisting}

图片占 1/3 左,文字占 2/3 右。

\textbf{文字+双图 Frame}

命令格式:

\begin{lstlisting}[style=blockstyle,language=TeX]
\iqbframetexttwofig[高度]{标题}{文字}{img1}{cap1}{img2}{cap2}
\end{lstlisting}

文字占 1/3 左,两张并排的图片占 2/3 右。

\subsection{Section 管理快捷}

\textbf{自动 Section 分隔页}

命令格式:

\begin{lstlisting}[style=blockstyle,language=TeX]
\iqbsectionframe{SectionName}{中文标题}
\end{lstlisting}

自动生成 section 分隔页,并更新 footer 中的 section 进度标识,无需手动调用 \lstinline|\setsection|。

使用示例:

\begin{lstlisting}[style=blockstyle,language=TeX]
\iqbsectionframe{Methods}{方法}
% 后续该 section 下的 frame 会自动显示 "Methods" 在 footer 中

\iqbsectionframe{Results}{结果}
% section 切换,footer 自动更新
\end{lstlisting}

\subsection{公式+解释专用布局}

\textbf{公式 Frame}

命令格式:

\begin{lstlisting}[style=blockstyle,language=TeX]
\iqbformulaframe{标题}{公式内容}{右侧图文}
\end{lstlisting}

左侧 1/3 放置公式推导,右侧 2/3 放置图片或补充说明。代码量减少 70\%。

\textbf{公式块辅助命令}

命令格式:

\begin{lstlisting}[style=blockstyle,language=TeX]
\iqbformblock{标题}{公式}{解释}
\end{lstlisting}

自动管理公式块的间距和字体。使用示例:

\begin{lstlisting}[style=blockstyle,language=TeX]
\iqbformulaframe{Full-Path CV 原理}{
  \iqbformblock{成核部分 $\text{CV}_{\text{cyl}}$}{
    $$\text{CV}_{\text{cyl}} = 1 - d/\text{CV}_{\text{eq}}$$
  }{$d$: 圆柱内脂质尾部原子数}

  \iqbformblock{扩展部分 $\text{CV}_{\text{radius}}$}{
    $$\text{CV}_{\text{radius}} = r_{\text{min}}/r_{\text{unit}}$$
  }{$r_{\text{min}}$: 孔中心到最近脂质距离}
}{
  \iqbfig[height=0.55\textheight]{fig1.png}{集体变量示意图}
}
\end{lstlisting}

\subsection{对比展示快捷}

\textbf{三列对比}

命令格式:

\begin{lstlisting}[style=blockstyle,language=TeX]
\iqbthreecolcompare{title1}{img1}{items1}{title2}{img2}{items2}{title3}{img3}{content3}
\end{lstlisting}

三列均匀分布,每列可包含标题、图片、列表。代码量减少 83\%。

使用示例:

\begin{lstlisting}[style=blockstyle,language=TeX]
\begin{frame}{创新方法对比}
  \iqbthreecolcompare
    {Full-Path CV}{fig1u.png}{\item 成核+扩展统一 \item 无滞后}
    {Rapid CV}{fig1d.png}{\item 无限孔模拟 \item 效率高 10×}
    {开源实现}{plumed.png}{PLUMED 库\\GROMACS/LAMMPS}
\end{frame}
\end{lstlisting}

\textbf{双列对比}

命令格式:

\begin{lstlisting}[style=blockstyle,language=TeX]
\iqbtwocolcompare{title1}{img1}{items1}{title2}{img2}{items2}
\end{lstlisting}

\subsection{结果展示快捷}

\textbf{结果展示 Frame}

命令格式:

\begin{lstlisting}[style=blockstyle,language=TeX]
\iqbresultframe{标题}{左侧文字}{右侧图片}
\end{lstlisting}

左侧 1/3 放置关键发现和数据,右侧 2/3 放置结果图片。代码量减少 80\%。

使用示例:

\begin{lstlisting}[style=blockstyle,language=TeX]
\iqbresultframe{孔闭合过程分析}{
  \textbf{四阶段闭合}:
  \begin{enumerate}
    \item 平衡孔:初始稳定孔,连续水柱贯穿膜
    \item 半径缩小:孔边缘脂质重排,结构保持
    \item 水线程:闭合最后瞬间,仅剩连续水线
    \item 膜变薄:孔完全闭合,局部膜缺陷
  \end{enumerate}

  \textbf{关键发现}:脂质尾部密度与孔寿命强相关 (R²=0.82)
}{images/closure.png}
\end{lstlisting}

\textbf{对比结果 Frame}

命令格式:

\begin{lstlisting}[style=blockstyle,language=TeX]
\iqbcomparisonframe{标题}{左图}{左标题}{右图}{右标题}
\end{lstlisting}

并排展示两个对比结果,自动编号图片。代码量减少 90\%。

\subsection{块级快捷命令}

\textbf{快速创建 Block}

命令格式:

\begin{lstlisting}[style=blockstyle,language=TeX]
\iqbblock{标题}{内容}
\end{lstlisting}

替代冗长的 \lstinline|\begin{block}...\end{block}| 写法。

\textbf{快速创建列表}

命令格式:

\begin{lstlisting}[style=blockstyle,language=TeX]
\iqbitemize{\item 项1 \item 项2 \item 项3}

\iqbenumerate{\item 项1 \item 项2 \item 项3}
\end{lstlisting}

\subsection{VSCode Snippets 集成}

配置文件位置:\lstinline|.vscode/latex.code-snippets|

在 VSCode 中使用快捷方式,输入缩写后按 Tab 键自动展开完整命令模板:

\begin{table}[!h]
\centering
\caption{VSCode Snippets 快捷方式列表}
\label{tab:vscode-snippets}
\begin{tabular}{p{2cm}p{3cm}p{6cm}}
\toprule
\textbf{快捷方式} & \textbf{对应命令} & \textbf{功能说明} \\
\midrule
\lstinline|iqbf| & \lstinline|\iqbframe| & 快速创建标准 frame \\
\lstinline|iqbtf| & \lstinline|\iqbframetextfig| & 文字+图片 frame \\
\lstinline|iqbfit| & \lstinline|\iqbframefigttext| & 图片+文字 frame \\
\lstinline|iqbff| & \lstinline|\iqbformulaframe| & 公式+解释 frame \\
\lstinline|iqbsec| & \lstinline|\iqbsectionframe| & Section 分隔页 \\
\lstinline|iqb3c| & \lstinline|\iqbthreecolcompare| & 三列对比 \\
\lstinline|iqb2c| & \lstinline|\iqbtwocolcompare| & 双列对比 \\
\lstinline|iqbrf| & \lstinline|\iqbresultframe| & 结果展示 \\
\lstinline|iqbcf| & \lstinline|\iqbcomparisonframe| & 对比结果 \\
\lstinline|iqbitem| & \lstinline|\iqbitemize| & 列表 \\
\lstinline|iqbenum| & \lstinline|\iqbenumerate| & 编号列表 \\
\lstinline|iqbfb| & \lstinline|\iqbformblock| & 公式块 \\
\bottomrule
\end{tabular}
\end{table}

使用示例:在 VSCode 中输入 \lstinline|iqbtf| 并按 Tab,自动展开为:

\begin{lstlisting}[style=blockstyle,language=TeX]
\iqbframetextfig{标题}{
  文字内容...
}{images/xxx.png}
\end{lstlisting}

\subsection{使用建议与最佳实践}

\begin{itemize}
\item \textbf{优先使用快捷命令}:在日常编写中优先采用快捷命令,保持代码简洁明了。
\item \textbf{灵活使用可选参数}:大多数命令支持可选参数微调,如 \lstinline|[0.6\textheight]| 自定义图片高度。
\item \textbf{完全兼容旧代码}:新命令与现有代码完全兼容,现有幻灯片无需修改。
\item \textbf{IDE 集成}:配合 VSCode 和 Snippets 获得最佳编写体验。
\item \textbf{图注管理}:每个图片都应有图注,可以是详细描述或文字解读。当图片非常高时,可以在旁边文字栏中编写解释。
\end{itemize}

\subsection{代码改进效果统计}

表 \ref{tab:code-improvement} 展示了使用快捷命令后的代码行数改进。平均而言,使用这些快捷命令可将代码量减少约 79\%。

\begin{table}[!h]
\centering
\caption{快捷命令代码改进统计}
\label{tab:code-improvement}
\begin{tabular}{lcccr}
\toprule
\textbf{模板类型} & \textbf{改进前} & \textbf{改进后} & \textbf{减少} & \textbf{对应命令} \\
\midrule
标准 Frame & 3 行 & 1 行 & 67\% & \lstinline|\iqbframe| \\
文字+图片 & 6 行 & 1 行 & 83\% & \lstinline|\iqbframetextfig| \\
公式+解释 & 40 行 & 12 行 & 70\% & \lstinline|\iqbformulaframe| \\
三列对比 & 48 行 & 8 行 & 83\% & \lstinline|\iqbthreecolcompare| \\
结果展示 & 15 行 & 3 行 & 80\% & \lstinline|\iqbresultframe| \\
对比结果 & 10 行 & 1 行 & 90\% & \lstinline|\iqbcomparisonframe| \\
\midrule
\textbf{平均} & - & - & \textbf{79\%} & - \\
\bottomrule
\end{tabular}
\end{table}

对于一个典型的 14 页 Journal Club 演示,使用快捷命令可以将源代码从约 270 行减少至 80 行,显著提升维护效率和可读性。

\section{Phase 1.5: 高级模式快捷命令(P0+P1 Priority Commands)}

\subsection{概述}

继Phase 1(基础快捷命令)之后,Phase 1.5进一步识别并抽象了8个常见的重复模式,提供6个新的快捷命令,可再减少30-50行代码。

\subsection{P0优先级(立即实现)}

\textbf{1. 封面页一行式模板:\lstinline|\iqbcoverframe|}

将20行的手动封面页代码简化为1行,自动从元数据生成:

\begin{lstlisting}[style=blockstyle,language=TeX]
% 文档开头定义元数据
\papertitlechn{破解膜孔之谜:双CV联手揭示}
\papertitlechnsub{从成核到扩展的完整能量图景}
\author{高旭帆}
\institute{IQB Lab}

% 文档中调用
\iqbcoverframe  % 1 line instead of 20!
\end{lstlisting}

\textbf{优势}:代码减少 95\%,修改标题只需改元数据一处

\textbf{2. 高亮发现块:\lstinline|\iqbhighlight|}

用于突出核心科学发现或临界结论:

\begin{lstlisting}[style=blockstyle,language=TeX]
\iqbhighlight{临界发现}{
  脂质尾部密度 $\leftrightarrow$ 孔寿命 $\tau$ (R²=0.82)
}{相关性极强!}
\end{lstlisting}

\textbf{优势}:代码减少 80\%,自动颜色和格式管理

\textbf{3. 分步骤列表:\lstinline|\iqbsteplist| 和 \lstinline|\step|}

清晰展示多阶段流程:

\begin{lstlisting}[style=blockstyle,language=TeX]
\iqbsteplist{四阶段闭合动力学}{
  \step{(A) 平衡孔}{初始稳定孔,连续水柱贯穿膜厚度}
  \step{(B) 半径缩小}{孔边缘脂质重排,整体结构保持}
  \step{(C) 水线程}{关键瞬间,孔只有纤细连续水线}
  \step{(D) 膜恢复}{孔完全闭合,局部膜厚度恢复}
}
\end{lstlisting}

\textbf{优势}:代码减少 50\%,自动编号和间距管理

---

\subsection{P1优先级(下一批)}

\textbf{4. 带色彩强调的列表:\lstinline|\iqbcolorlist|}

用于展示带颜色分类的列表(✓准确 / ~部分正确 / ✗失败):

\begin{lstlisting}[style=blockstyle,language=TeX]
\iqbcolorlist{力场表现}{
  \item {\color{green!60!black}\textbf{C36}}:准确
  \item {\color{orange}\textbf{Slipids}}:部分正确
  \item {\color{red}\textbf{M2.2/M3}}:失败
}
\end{lstlisting}

\textbf{优势}:代码减少 57\%

\textbf{5. Bullet列表快捷命令:\lstinline|\iqbbulletlist|}

简化bullet列表:

\begin{lstlisting}[style=blockstyle,language=TeX]
\iqbbulletlist{1. 线张力计算}{
  \item Full-Path CV: 成核+扩展全过程
  \item Rapid CV: 快速评估大孔极限
}
\end{lstlisting}

\textbf{优势}:代码减少 63\%

\textbf{6. 增强型公式参数块:\lstinline|\iqbformparam| 和 \lstinline|\param|}

将公式和参数表格化展示:

\begin{lstlisting}[style=blockstyle,language=TeX]
\iqbformparam{切换函数}{
  $$s = \frac{1}{1 + e^{\alpha(\text{CV} - \text{CV}_0)}}$$
}{
  \param{\alpha = 20}{陡峭}
  \param{\text{CV}_0 = 0.95}{切换点}
}
\end{lstlisting}

\textbf{优势}:代码减少 40\%,参数表格化易读

---

\subsection{Phase 1.5 改进统计}

\begin{table}[!h]
\centering
\caption{Phase 1.5 新命令代码改进统计}
\label{tab:p1.5-improvement}
\begin{tabular}{lcccr}
\toprule
\textbf{命令} & \textbf{原方式} & \textbf{新方式} & \textbf{减少} & \textbf{级别} \\
\midrule
\lstinline|\iqbcoverframe| & 20行 & 1行 & 95\% & P0 \\
\lstinline|\iqbhighlight| & 5行 & 1行 & 80\% & P0 \\
\lstinline|\iqbsteplist| & 8行 & 4行 & 50\% & P0 \\
\lstinline|\iqbcolorlist| & 7行 & 3行 & 57\% & P1 \\
\lstinline|\iqbbulletlist| & 8行 & 3行 & 63\% & P1 \\
\lstinline|\iqbformparam| & 10行 & 6行 & 40\% & P1 \\
\bottomrule
\end{tabular}
\end{table}

对于典型演示,Phase 1.5 可额外减少 \textbf{30-50 行代码}。整合Phase 1+Phase 1.5,平均代码改进达到 \textbf{82\%}(从270行→50行)。