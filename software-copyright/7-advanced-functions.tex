%!TEX root = document.tex

\chapter{高级功能模块与快捷命令系统}
\label{chap:7}

本章详细介绍 IQB-JC 模板的图表模块、作者信息模块和 Punchline 智能强调系统。这些模块提供了丰富的内容展示功能,帮助用户创建更加专业、美观的学术演示文稿。

\section{图表模块(自动编号)}

IQB-JC 模板提供了带自动编号和统一格式的图表模块。所有图表自动生成"图X:"格式的编号,caption 左对齐显示。

\subsection{单图模块}

命令格式:

\begin{lstlisting}[style=blockstyle,language=TeX]
\iqbfig[height=0.5\textheight]{图片路径}{图说文字}
\end{lstlisting}

参数说明见表 \ref{tab:iqbfig-params}。

\begin{table}[!h]
\centering
\caption{iqbfig 命令参数}
\label{tab:iqbfig-params}
\begin{tabular}{p{2.5cm}p{2.5cm}p{8cm}}
\toprule
\textbf{参数} & \textbf{默认值} & \textbf{说明} \\
\midrule
\lstinline|[选项]| & \lstinline|height=0.5\textheight| & 图片尺寸,支持 \lstinline|height| 或 \lstinline|width| \\
图片路径 & 必需 & 相对或绝对路径,支持 PNG/PDF \\
图说文字 & 必需 & 图片下方的说明文字,自动编号 \\
\bottomrule
\end{tabular}
\end{table}

使用示例:

\begin{lstlisting}[style=blockstyle,language=TeX]
\begin{frame}{单图展示}
  \iqbfig[height=0.55\textheight]{results.png}{
    结果展示图:实验数据对比分析
  }
\end{frame}
\end{lstlisting}

\subsection{双图模块(带图号和图注)}

当需要两张图片并排显示且各有独立的图号和图注时,使用 \lstinline|\iqbtwofig| 命令。该命令自动为每张图片编号("图1"、"图2"等)。

命令格式:

\begin{lstlisting}[style=blockstyle,language=TeX]
\iqbtwofig[height=0.45\textheight]{img1}{caption1}{img2}{caption2}
\end{lstlisting}

该模块将两张图片并排显示,各占 48\% 宽度,自动添加图号前缀。使用示例:

\begin{lstlisting}[style=blockstyle,language=TeX]
\begin{frame}{双图对比}
  \iqbtwofig[height=0.5\textheight]{
    method_a.png
  }{
    方法 A 的结果展示
  }{
    method_b.png
  }{
    方法 B 的结果展示
  }
\end{frame}
\end{lstlisting}

\subsection{双图并排(无caption,支持间距控制)}

当需要并排放置两张图片,但不需要图号和图注时,使用 \lstinline|\iqbtwoimgside| 命令。该命令支持自定义两图间距。

命令格式:

\begin{lstlisting}[style=blockstyle,language=TeX]
\iqbtwoimgside[间距]{左图内容}{右图内容}
\end{lstlisting}

参数说明:
\begin{itemize}
  \item \texttt{[间距]}(可选,默认 0.3cm):控制两张图片之间的距离
  \begin{itemize}
    \item 正值(如 \texttt{0.2cm}):增加间距
    \item 负值(如 \texttt{-0.3cm}):减少间距,使图片靠近
  \end{itemize}
  \item 左图内容:通常使用 \lstinline|\iqbimgcenter| 或其他图片命令
  \item 右图内容:同上
\end{itemize}

使用示例:

\begin{lstlisting}[style=blockstyle,language=TeX]
\begin{frame}{两个奖项展示}
  \iqbtwoimgside[-0.3cm]{
    \iqbimgcenter[height=0.4\textheight]{award1.png}
  }{
    \iqbimgcenter[height=0.4\textheight]{award2.png}
  }
\end{frame}
\end{lstlisting}

\subsection{三图模块}

命令格式:

\begin{lstlisting}[style=blockstyle,language=TeX]
\iqbthreefig[height=0.35\textheight]{img1}{cap1}{img2}{cap2}{img3}{cap3}
\end{lstlisting}

三张图片均匀分布在一行,各占 31\% 宽度。该布局适合展示三个相关的分析结果。

\subsection{四图网格模块}

命令格式:

\begin{lstlisting}[style=blockstyle,language=TeX]
\iqbfourfig[height=0.35\textheight]{
  img1}{cap1}{img2}{cap2}{img3}{cap3}{img4}{cap4}
\end{lstlisting}

四张图片排列为 2×2 网格。该布局提供清晰的图表组织方式,适合展示多阶段或多方法的结果对比。

\section{作者信息模块}

IQB-JC 模板提供了专门的作者信息展示模块,适用于文献汇报时介绍论文作者的信息。

\subsection{双作者模块(无照片)}

命令格式:

\begin{lstlisting}[style=blockstyle,language=TeX]
\iqbauthorstwo{
  通讯作者姓名}{通讯作者单位}{通讯作者网址}{通讯作者研究方向}{
  第一作者姓名}{第一作者单位}{第一作者网址}{第一作者研究方向
}
\end{lstlisting}

该模块以 50-50 双列布局展示通讯作者和第一作者的信息。使用示例:

\begin{lstlisting}[style=blockstyle,language=TeX]
\begin{frame}{作者信息}
  \iqbauthorstwo{
    Prof. John Smith
  }{
    Department of Biology\\
    University of Example
  }{
    https://example.edu/smith
  }{
    Molecular dynamics, Protein folding
  }{
    Dr. Jane Doe
  }{
    Institute of Computational Science\\
    Example University
  }{
    https://example.edu/doe
  }{
    MD simulations, Free energy calculations
  }
\end{frame}
\end{lstlisting}

\subsection{双作者模块(带照片)}

命令格式:

\begin{lstlisting}[style=blockstyle,language=TeX]
\setauthorfirstfield{第一作者研究领域}
\iqbauthorstwophoto{
  通讯作者照片路径}{通讯作者姓名}{...}{...}{通讯作者研究方向}{
  第一作者照片路径}{第一作者姓名}{...}{...}
}
\end{lstlisting}

使用 \lstinline|\setauthorfirstfield{}| 命令单独设置第一作者的专业领域。使用示例:

\begin{lstlisting}[style=blockstyle,language=TeX]
\begin{frame}{论文作者}
  \setauthorfirstfield{计算生物物理、膜蛋白模拟、自由能计算}
  \iqbauthorstwophoto{
    author1.jpg
  }{
    Prof. Alice Brown
  }{
    College of Science\\
    Tech University
  }{
    https://example.org/alice
  }{
    Protein structure prediction, Force field development
  }{
    author2.jpg
  }{
    Dr. Bob White
  }{
    Center for Molecular Dynamics\\
    Research Institute
  }{
    https://example.org/bob
  }
\end{frame}
\end{lstlisting}

\subsection{单作者模块}

命令格式:

\begin{lstlisting}[style=blockstyle,language=TeX]
\iqbauthorone{作者姓名}{作者单位}{作者网址}{研究方向}
\end{lstlisting}

该模块适用于单作者论文或特邀评论文章。信息居中显示。

\section{Punchline 智能强调系统}

\label{sec:punchline}

IQB-JC 模板提供了灵活的 Punchline 系统,用于在幻灯片页面中突出核心观点、关键发现或学术申请陈述的重点内容。该系统支持多种颜色、对齐方式和框风格的组合,满足不同的表达需求。

\subsection{基础 Punchline 命令}

\textbf{简单指定颜色}

命令格式:

\begin{lstlisting}[style=blockstyle,language=TeX]
\iqbpunchline[颜色]{文本}
\end{lstlisting}

默认蓝色,文本自动居中。使用示例:

\begin{lstlisting}[style=blockstyle,language=TeX]
\iqbpunchline[iqbgreen]{计算驱动创新,精准设计引领药物研发新方向!}

\iqbpunchline[iqborange]{警示:需要特别关注的重要发现}
\end{lstlisting}

\subsection{色彩快捷命令(推荐)}

提供三种颜色的快捷命令,对应不同的内容类型:

\textbf{蓝色 Punchline(核心观点、主要发现)}

\begin{lstlisting}[style=blockstyle,language=TeX]
% 最简单:默认无框,居中对齐
\iqbpunchlineblue{核心观点文本}

% 左对齐(适合多行文本)
\iqbpunchlineblue[l]{
  多行文本行 1\\
  多行文本行 2
}

% 有框,居中对齐
\iqbpunchlineblue[b]{有框的蓝色 punchline}

% 有框,左对齐(多行+框)
\iqbpunchlineblue[bl]{
  多行文本:\\
  第一行\\
  第二行
}
\end{lstlisting}

\textbf{绿色 Punchline(成功验证、正面结果)}

\begin{lstlisting}[style=blockstyle,language=TeX]
% 默认:无框,居中对齐
\iqbpunchlinegreen{验证通过:实验结果符合预期}

% 有框,居中对齐
\iqbpunchlinegreen[b]{成功案例:有框显示}

% 长文本左对齐(推荐多行内容)
\iqbpunchlinegreen[l]{PKNAP整合FEPaML、NEOM、MultiTAP等多个肿瘤免疫服务器,\\
融合物理、知识与AI,与FEbuilder、PolyglotMol等形成完整计算药物设计平台!}
\end{lstlisting}

\textbf{橙色 Punchline(警示、挑战、需要关注)}

\begin{lstlisting}[style=blockstyle,language=TeX]
\iqbpunchlineorange{警示:需要关注的重要事项}

\iqbpunchlineorange[l]{多行警告\\
需要特别注意}
\end{lstlisting}

\textbf{参数说明}

表 \ref{tab:punchline-params} 详细说明了可选参数及其含义。

\begin{table}[!h]
\centering
\caption{Punchline 快捷命令参数}
\label{tab:punchline-params}
\begin{tabular}{p{2cm}p{3cm}p{7cm}}
\toprule
\textbf{参数} & \textbf{效果} & \textbf{适用场景} \\
\midrule
空(默认) & 无框,居中对齐 & 单行短文本 \\
\lstinline|[l]| & 无框,左对齐 & 多行长文本 \\
\lstinline|[b]| & 有框,居中对齐 & 需要突出强调的单行 \\
\lstinline|[bl]| 或 \lstinline|[lb]| & 有框,左对齐 & 多行文本需要框突显 \\
\bottomrule
\end{tabular}
\end{table}

\subsection{向后兼容命令}

为了兼容之前的代码,以下命令仍然可用(等价于 \lstinline|[b]| 参数版本):

\begin{lstlisting}[style=blockstyle,language=TeX]
\iqbpunchlineboxgreen{有框的绿色 punchline}

\iqbpunchlineboxorange{有框的橙色 punchline}
\end{lstlisting}