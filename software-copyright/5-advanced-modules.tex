\chapter{高级布局模块}
\label{chap:5}

本章详细介绍 IQB-JC 模板的多列布局系统、图表布局模块、图文混排功能,以及相关的辅助工具。这些模块为用户提供了丰富的布局选择,满足不同内容类型的展示需求。

\section{多列布局模块}

以下字段是必填的,用于定义演示文稿的基本信息:

\begin{lstlisting}[style=blockstyle,language=TeX]
\title{演示标题}           % 封面主标题
\subtitle{副标题}           % 可选:副标题
\author{作者名字}           % 作者名字
\institute{机构/学院}       % 学院或研究所名称
\date{\today}               % 日期(\today 表示当前日期)
\end{lstlisting}

\subsection{扩展字段}

模板额外支持以下字段,用于学位申请、奖学金申请等场景。这些字段**完全可选**,如不设置则不显示:

\begin{lstlisting}[style=blockstyle,language=TeX]
\stuid{学号}               % 学号(如果设置则在封面显示)
\major{专业名称}           % 专业或学位类型(如果设置则在封面显示)
\advisor{导师名字 教授}    % 指导教师名字(如果设置则在封面显示)
\end{lstlisting}

使用示例:

\begin{lstlisting}[style=blockstyle,language=TeX]
\title{杨咏曼奖学金申请陈述}
\author{高旭帆}
\institute{生命科学学院 \quad 生物物理研究所}
\stuid{12207134}
\major{生物物理学}
\advisor{周如鸿 教授}
\end{lstlisting}

\subsection{日期格式设置}

模板支持中英文日期格式切换,通过 \lstinline|\dateformat{}| 命令控制:

\begin{lstlisting}[style=blockstyle,language=TeX]
% 使用中文日期格式(推荐用于中文演示)
\dateformat{zh}
\date{\today}           % 显示为:"2025 年 11 月 5 日"

% 使用英文日期格式(推荐用于英文演示)
\dateformat{en}
\date{\today}           % 显示为:"5 November 2025"
\end{lstlisting}

\textbf{注意}:\lstinline|\dateformat| 命令必须在 \lstinline|\date{}| 之前调用。

\section{多列布局模块}

IQB-JC 模板通过 \lstinline|iqb-layouts.sty| 包提供了多种预设布局工具,满足不同的内容排版需求。

\subsection{双列布局(50-50)}

命令格式:

\begin{lstlisting}[style=blockstyle,language=TeX]
\iqblayouttwo{左列内容}{右列内容}
\end{lstlisting}

该命令将页面平均分为两列,各占 48\% 宽度,中间留 4\% 间距。适用于并行展示两个相关内容的场景。使用示例:

\begin{lstlisting}[style=blockstyle,language=TeX]
\begin{frame}{双列对比}
  \iqblayouttwo{
    \textbf{方法 A}
    \begin{itemize}
      \item 优点 1
      \item 优点 2
    \end{itemize}
  }{
    \textbf{方法 B}
    \begin{itemize}
      \item 优点 1
      \item 优点 2
    \end{itemize}
  }
\end{frame}
\end{lstlisting}

\subsection{不对称双列布局}

\textbf{1/3 + 2/3 布局}

命令格式:

\begin{lstlisting}[style=blockstyle,language=TeX]
\iqblayoutonethird{左列内容(占 31\%)}{右列内容(占 65\%)}
\end{lstlisting}

该布局将页面分为 31\% 和 65\% 两列,适合在右侧放置大图,左侧放置说明文字。使用示例:

\begin{lstlisting}[style=blockstyle,language=TeX]
\begin{frame}{竖版图片布局}
  \iqblayoutonethird{
    \textbf{说明文字}

    该图展示了关键结果
  }{
    \includegraphics[height=0.6\textheight,keepaspectratio]{image.png}
  }
\end{frame}
\end{lstlisting}

\textbf{2/3 + 1/3 布局}

命令格式:

\begin{lstlisting}[style=blockstyle,language=TeX]
\iqblayouttwothirds{左列内容(占 65\%)}{右列内容(占 31\%)}
\end{lstlisting}

该布局与上述布局对称,适合在左侧放置大图,右侧放置说明文字。

\subsection{三列布局}

命令格式:

\begin{lstlisting}[style=blockstyle,language=TeX]
\iqblayoutthree{左列内容}{中列内容}{右列内容}
\end{lstlisting}

将页面平均分为三列,各占 31\% 宽度。适合展示三个平行的内容或对比三种方法。使用示例:

\begin{lstlisting}[style=blockstyle,language=TeX]
\begin{frame}{三方法对比}
  \iqblayoutthree{
    \textbf{方法 A}

    优点与缺点分析
  }{
    \textbf{方法 B}

    优点与缺点分析
  }{
    \textbf{方法 C}

    优点与缺点分析
  }
\end{frame}
\end{lstlisting}

\section{网格与图表布局}

\subsection{2×2 网格布局}

命令格式:

\begin{lstlisting}[style=blockstyle,language=TeX]
\iqbgridtwobytwo{左上}{右上}{左下}{右下}
\end{lstlisting}

该布局适合展示四个相关的图片或内容块,自动排列成 2×2 网格。使用示例:

\begin{lstlisting}[style=blockstyle,language=TeX]
\begin{frame}{四个关键结果}
  \iqbgridtwobytwo{
    \includegraphics[height=0.35\textheight]{fig1.png}
  }{
    \includegraphics[height=0.35\textheight]{fig2.png}
  }{
    \includegraphics[height=0.35\textheight]{fig3.png}
  }{
    \includegraphics[height=0.35\textheight]{fig4.png}
  }
\end{frame}
\end{lstlisting}

\subsection{3×2 网格布局}

命令格式:

\begin{lstlisting}[style=blockstyle,language=TeX]
\iqbgridthreebytwo{左上}{中上}{右上}{左下}{中下}{右下}
\end{lstlisting}

该布局展示六个内容块,排列为 3 列 2 行。各列占 31\% 宽度,垂直间距为 0.5em。

\section{图文混排模块}

\subsection{图片左、文字右}

命令格式:

\begin{lstlisting}[style=blockstyle,language=TeX]
\iqbimagetext[图片选项]{图片路径}{右侧文字内容}
\end{lstlisting}

默认图片选项为 \lstinline|width=0.45\textwidth|。使用示例:

\begin{lstlisting}[style=blockstyle,language=TeX]
\begin{frame}{实验结果}
  \iqbimagetext[width=0.4\textwidth]{experiment.png}{
    \textbf{关键发现:}
    \begin{itemize}
      \item 发现 1
      \item 发现 2
      \item 发现 3
    \end{itemize}
  }
\end{frame}
\end{lstlisting}

\subsection{文字左、图片右}

命令格式:

\begin{lstlisting}[style=blockstyle,language=TeX]
\iqbtextimage[图片选项]{左侧文字内容}{图片路径}
\end{lstlisting}

这是上述布局的镜像版本,文字显示在左列,图片显示在右列。使用示例:

\begin{lstlisting}[style=blockstyle,language=TeX]
\begin{frame}{方法流程}
  \iqbtextimage[height=0.55\textheight]{
    \textbf{研究步骤:}
    \begin{enumerate}
      \item 数据收集
      \item 预处理
      \item 分析
    \end{enumerate}
  }{flowchart.png}
\end{frame}
\end{lstlisting}

\section{辅助工具}

\subsection{PDF 转 PPTX(嵌入图片)}

\subsubsection{用途与特点}

\lstinline|pdf_to_pptx.py| 工具用于快速将编译后的 PDF 演示文稿转换为 PPTX 格式。该工具将 PDF 的每一页转换为高分辨率图片并嵌入到 PPTX 中,适用于以下场景:

\begin{itemize}
  \item 需要快速生成 PowerPoint 格式供他人浏览
  \item 保留 PDF 中的完整布局和样式(包括 header、footer、排版等)
  \item 无需在 PowerPoint 中编辑内容,只需要演示
\end{itemize}

\paragraph{使用方法}

基本用法:

\begin{lstlisting}[style=blockstyle,language=bash]
python tools/pdf_to_pptx.py examples/Xufan.pdf output.pptx
\end{lstlisting}

高分辨率转换(可选):

\begin{lstlisting}[style=blockstyle,language=bash]
python tools/pdf_to_pptx.py examples/Xufan.pdf output.pptx --dpi 300
\end{lstlisting}

添加页码注释:

\begin{lstlisting}[style=blockstyle,language=bash]
python tools/pdf_to_pptx.py examples/Xufan.pdf output.pptx --add-notes
\end{lstlisting}

\paragraph{系统依赖}

使用前需要安装必要的依赖:

\begin{lstlisting}[style=blockstyle,language=bash]
# Python 依赖
pip install pdf2image python-pptx pillow

# 系统依赖(Linux/WSL)
sudo apt install poppler-utils

# 系统依赖(macOS)
brew install poppler
\end{lstlisting}

\paragraph{限制}

因为是嵌入图片,生成的 PPTX:
\begin{itemize}
  \item 不可编辑(文字无法选中和修改)
  \item 文件较大(由于高分辨率图片)
  \item 无法提取或修改单个元素(如字体、颜色)
\end{itemize}

\subsection{PDF 提取页面为 PNG}

\subsubsection{用途与特点}

\lstinline|extract_pdf_page.py| 工具用于提取 PDF 中的指定页面并保存为 PNG 图片,用于视觉验证和调试。该工具常配合 pdf-layout-reviewer agent 使用,用于检查页面布局是否符合模板要求。

\paragraph{使用方法}

提取第 6 页到默认路径 \lstinline|/tmp/pdf_page_6.png|:

\begin{lstlisting}[style=blockstyle,language=bash]
python tools/extract_pdf_page.py examples/Xufan.pdf 6
\end{lstlisting}

指定输出路径:

\begin{lstlisting}[style=blockstyle,language=bash]
python tools/extract_pdf_page.py examples/Xufan.pdf 6 my_output.png
\end{lstlisting}

\paragraph{系统依赖}

\begin{lstlisting}[style=blockstyle,language=bash]
# Python 依赖
pip install pdf2image pillow

# 系统依赖(Linux/WSL)
sudo apt install poppler-utils

# 系统依赖(macOS)
brew install poppler
\end{lstlisting}

\subsection{生成可编辑 PPTX(布局保持)}

\subsubsection{用途与特点}

\lstinline|pdf_to_editable_pptx.py| 工具生成真正可编辑的 PPTX 文件,采用 \textbf{混合方法(背景图 + 文本框)},实现:
\begin{itemize}
  \item \textbf{完全相同的布局}:PDF 页面作为 300 DPI 高清背景图,确保视觉完全一致
  \item \textbf{完全可编辑的文字}:透明文本框精确覆盖原文字位置,保持样式
  \item \textbf{最佳的可用性}:既有专业的视觉效果,又能在 PowerPoint 中自由编辑
\end{itemize}

\paragraph{核心特性}

\begin{itemize}
  \item \textbf{样式完整保留}:字体名称、字号、颜色、粗体、斜体等
  \item \textbf{布局精确映射}:保留精确的文本位置(坐标映射)
  \item \textbf{页面比例精确匹配}:自动从 PDF 获取真实页面尺寸,确保 PPTX 与 PDF 页面比例完全一致
  \item \textbf{小文件体积}:40--50KB(相比嵌入图片方式小 100 倍)
  \item \textbf{完全可编辑}:文本可选、可修改、可重新设置样式
  \item \textbf{快速转换}:无需 OCR,直接提取 PDF 结构信息
\end{itemize}

\paragraph{适用场景}

\begin{itemize}
  \item 需要在 PowerPoint 中进一步编辑内容
  \item 需要保持原始样式(颜色、字体、大小)的可编辑性
  \item 文件大小有限制(邮件、云同步等)
  \item 需要提取和复用幻灯片中的文本内容
  \item 需要修改某些 PDF 内容但保留整体样式
\end{itemize}

\paragraph{限制与注意}

\begin{itemize}
  \item 不能保留 PDF 中的复杂矢量图形(但可提取图片)
  \item 某些特殊 PostScript 字体可能被替换为相似的可用字体
  \item 复杂多列布局的重排可能需要手动微调
  \item 超复杂的 LaTeX 数学公式可能无法完美还原
\end{itemize}

\paragraph{使用方法}

标准用法:

\begin{lstlisting}[style=blockstyle,language=bash]
python tools/pdf_to_editable_pptx.py input.pdf output.pptx
\end{lstlisting}

示例:

\begin{lstlisting}[style=blockstyle,language=bash]
# 转换为可编辑 PPTX(保留所有样式)
python tools/pdf_to_editable_pptx.py examples/Xufan.pdf my_presentation.pptx

# 转换后的 PPTX 可在 PowerPoint/LibreOffice 中编辑
# - 文字可选择和修改
# - 颜色、字体保持不变
# - 位置和大小也能准确还原
\end{lstlisting}

\paragraph{系统依赖}

\begin{lstlisting}[style=blockstyle,language=bash]
# Python 依赖
pip install pymupdf python-pptx pillow

# 系统依赖(仅 Linux/WSL 需要,Windows 不需要)
# 如果不使用 pdf_to_pptx.py,不需要安装 poppler
sudo apt install poppler-utils
\end{lstlisting}

\paragraph{工作原理}

该工具的转换流程:

\begin{enumerate}
  \item 使用 PyMuPDF 的 \lstinline|get_text("dict")| 提取 PDF 每个 span 的详细信息
  \item 提取内容:文本内容、字体名称、字号、RGB 颜色、粗体/斜体标志、精确坐标
  \item 坐标映射:将 PDF 坐标(左下角原点)转换为 PPTX 坐标(左上角原点)
  \item 创建文本框:根据提取的样式逐个创建 PPTX 文本框,应用对应的字体/颜色/大小
  \item 生成 PPTX:保存为可在 PowerPoint/LibreOffice 中编辑的 PPTX 文件
\end{enumerate}

\paragraph{工作流推荐}

\begin{enumerate}
  \item 在 LaTeX 中完成演示文稿的设计和编辑
  \item 编译得到最终 PDF
  \item 根据需求选择:
    \begin{itemize}
      \item 仅需 PowerPoint 查看 $\Rightarrow$ 使用 \lstinline|pdf_to_pptx.py|(快速,文件完整)
      \item 需要在 PowerPoint 中编辑 $\Rightarrow$ 使用 \lstinline|pdf_to_editable_pptx.py|(灵活,可编辑)
    \end{itemize}
  \item 如需验证 PDF 布局是否正确 $\Rightarrow$ 使用 \lstinline|extract_pdf_page.py| + pdf-layout-reviewer
