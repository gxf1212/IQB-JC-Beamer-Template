%!TEX root = document.tex

\chapter{主题自定义与字体设置}
\label{chap:3}

\Name 是一个功能完整的学术演示文稿 LaTeX Beamer 模板,专为 IQB Lab 日志会(Journal Club)汇报设计。通过提供预设的主题、布局工具包和丰富的内容模块,用户可以快速创建专业、美观的学术幻灯片演示。本章详细说明该模板的基础使用方法和核心功能模块。

\section{基础使用与最小示例}

\subsection{最小示例代码}

以下是创建一个完整演示文稿的最小示例。用户可以以此为起点进行扩展:

\begin{lstlisting}[style=blockstyle,language=TeX]
\documentclass[aspectratio=169,11pt]{beamer}

% 加载 IQB 主题和布局工具包
\usepackage{theme/beamerthemeiqb}
\usepackage{theme/iqb-layouts}

% 设置 header 图片路径
\renewcommand{\iqbheaderimage}{theme/images/header.png}

% 中文支持(使用 XeLaTeX 编译)
\usepackage{xeCJK}
\setCJKmainfont{SimSun}

% 文档元数据
\title{你的演示标题}
\subtitle{副标题(可选)}
\author{你的名字}
\institute{IQB Lab}
\stuid{学号}                    % 可选:学生学号
\major{专业名称}                 % 可选:专业/学位类型
\advisor{导师名字 教授}          % 可选:指导教师名字
\dateformat{zh}                % 可选:zh 为中文日期,en 为英文日期(默认英文)
\date{\today}

% ============================================================
% Footer Configuration(页脚配置)
% ============================================================
% 格式:\iqbfootline{left}{center}{right}
% - left:   default 或自定义文本
% - center: section(章节名)或 title(当前页标题)
% - right:  ratio(N / 总数)、pageofy(Page N/总数)、zhpageofy(第N/总数页)
\iqbfootline{default}{title}{pageofy}

\begin{document}

% 封面页(无 header/footer)
\begin{frame}[plain,noframenumbering]
  \titlepage
\end{frame}

% 内容页示例
\setsection{Background}
\begin{frame}{第一页内容}
  \begin{itemize}
    \item 要点 1
    \item 要点 2
    \item 要点 3
  \end{itemize}
\end{frame}

% 致谢页(无 header/footer)
\begin{frame}[plain]
  \centering
  {\Huge Thank You!}
\end{frame}

\end{document}
\end{lstlisting}

\subsection{编译方法}

使用 XeLaTeX 编译器(推荐,支持中文):

\begin{lstlisting}[style=blockstyle,language=bash]
xelatex -interaction=nonstopmode your-presentation.tex
\end{lstlisting}

若需要生成含交叉引用的完整 PDF,需运行两次编译:

\begin{lstlisting}[style=blockstyle,language=bash]
xelatex -interaction=nonstopmode your-presentation.tex
xelatex -interaction=nonstopmode your-presentation.tex
\end{lstlisting}

\subsection{文档类选项说明}

表 \ref{tab:documentclass-options} 列出了常用的文档类选项及其含义。

\begin{table}[!h]
\centering
\caption{Beamer 文档类常用选项}
\label{tab:documentclass-options}
\begin{tabular}{p{3.5cm}p{9.5cm}}
\toprule
\textbf{选项} & \textbf{说明} \\
\midrule
\lstinline|aspectratio=169| & 设置幻灯片宽高比为 16:9(推荐用于现代显示器) \\
\lstinline|aspectratio=43| & 设置幻灯片宽高比为 4:3(传统比例) \\
\lstinline|11pt| & 设置基础字体大小为 11pt \\
\lstinline|12pt| & 设置基础字体大小为 12pt(默认) \\
\bottomrule
\end{tabular}
\end{table}

\subsection{中文支持配置}

\textbf{跨平台中文字体配置(自动)}

模板已内置跨平台中文字体自动配置,用户无需手动设置。\lstinline|beamerthemeiqb.sty| 会根据操作系统自动选择最优字体:

\begin{table}[!h]
\centering
\caption{自动CJK字体配置(按操作系统)}
\label{tab:auto-cjk-fonts}
\begin{tabular}{p{2cm}p{3cm}p{3cm}p{3cm}}
\toprule
\textbf{操作系统} & \textbf{正文字体} & \textbf{加粗字体} & \textbf{说明} \\
\midrule
Windows & SimSun(宋体) & SimHei(黑体) & 系统内置,无需额外配置 \\
macOS & Songti SC & STHeiti(黑体) & 系统内置,无需额外配置 \\
Linux & Noto Serif CJK SC & Noto Sans CJK SC(思源黑体) & 需先安装字体包 \\
\bottomrule
\end{tabular}
\end{table}

\textbf{中文加粗支持}

重要改进:现在 \lstinline|\textbf{中文}| 能够正确显示加粗效果!这是通过在各个平台配置对应的黑体字体实现的。

使用示例:

\begin{lstlisting}[style=blockstyle,language=TeX]
\textbf{加粗的中文文字}    % 现在能正确显示加粗

\textbf{关键要点}:描述内容

{\bfseries 也可以用这种方式} 加粗
\end{lstlisting}

\textbf{手动配置(可选)}

如果需要覆盖自动配置,可在文档导言区手动指定字体:

\begin{lstlisting}[style=blockstyle,language=TeX]
% Windows 用户
\setCJKmainfont[BoldFont=SimHei]{SimSun}

% macOS 用户
\setCJKmainfont[BoldFont=STHeiti]{Songti SC}

% Linux 用户(需先安装 fonts-noto-cjk)
\setCJKmainfont[BoldFont=Noto Sans CJK SC]{Noto Serif CJK SC}
\end{lstlisting}

\textbf{Linux 用户注意事项}

若 Linux 系统中文字体缺失,运行以下命令安装:

\begin{lstlisting}[style=blockstyle,language=bash]
sudo apt-get install fonts-noto-cjk    % Ubuntu/Debian
sudo pacman -S noto-fonts-cjk           % Arch Linux
\end{lstlisting}

\section{页脚(Footer)配置}
\label{sec:footer-config}

\subsection{概述}

IQB Beamer 模板提供灵活的页脚配置系统,允许用户在导言区用一行代码即可自定义页脚的三段式内容(左 | 中 | 右)。页脚通常显示在所有内容页的底部,包含机构信息、章节/标题、和页码等内容。

\subsection{基本用法}

\textbf{配置命令}:

\begin{lstlisting}[style=blockstyle,language=TeX]
\iqbfootline{left}{center}{right}
\end{lstlisting}

该命令应在导言区(\lstinline|\usepackage| 之后,\lstinline|\begin{document}| 之前)调用。

\subsection{参数说明}

\begin{table}[!h]
\centering
\caption{Footer 配置参数}
\label{tab:footer-params}
\begin{tabular}{p{2cm}p{3.5cm}p{7cm}}
\toprule
\textbf{参数位置} & \textbf{可选值} & \textbf{说明} \\
\midrule
\multirow{2}{*}{\textbf{left}} & \lstinline|default| & 显示机构名称(来自 \lstinline|\institute| 命令) \\
& 自定义文本 & 显示用户指定的文本,例如 \lstinline|"IQB Lab"| 或 \lstinline|"团队名称"| \\
\midrule
\multirow{2}{*}{\textbf{center}} & \lstinline|section| & 显示当前章节名(需配合 \lstinline|\setsection{章节名}| 使用) \\
& \lstinline|title| & 显示当前页面的标题(Frame title) \\
\midrule
\multirow{3}{*}{\textbf{right}} & \lstinline|ratio| & 显示格式 \lstinline|N / 总数|(如 \lstinline|5 / 25|) \\
& \lstinline|pageofy| & 显示格式 \lstinline|Page N/总数|(如 \lstinline|Page 5/25|) \\
& \lstinline|zhpageofy| & 显示格式 \lstinline|第N/总数页|(如 \lstinline|第5/25页|) \\
\bottomrule
\end{tabular}
\end{table}

\subsection{使用示例}

\subsubsection{示例 1:显示 Frame 标题与英文页码}

\begin{lstlisting}[style=blockstyle,language=TeX]
\iqbfootline{default}{title}{pageofy}
\end{lstlisting}

\textbf{效果}:页脚显示为 \lstinline|机构名 | 当前页标题 | Page N/总数|

此配置适用于希望强调每一页核心内容的演示。

\subsubsection{示例 2:显示章节名与中文页码}

\begin{lstlisting}[style=blockstyle,language=TeX]
\iqbfootline{default}{section}{zhpageofy}
\end{lstlisting}

\textbf{效果}:页脚显示为 \lstinline|机构名 | 章节名 | 第N/总数页|

此配置需要在文档中使用 \lstinline|\setsection{章节名}| 来设置章节。示例:

\begin{lstlisting}[style=blockstyle,language=TeX]
% Part 1: Fundamentals
\setsection{Fundamentals}
\begin{frame}{Introduction to QSAR}
  % 页面内容
\end{frame}

% Part 2: Methods
\setsection{Methods}
\begin{frame}{Molecular Descriptors}
  % 页面内容
\end{frame}
\end{lstlisting}

\subsubsection{示例 3:自定义左侧文本}

\begin{lstlisting}[style=blockstyle,language=TeX]
\iqbfootline{Molecular Modeling Lab}{title}{ratio}
\end{lstlisting}

\textbf{效果}:页脚显示为 \lstinline|Molecular Modeling Lab | 当前页标题 | N / 总数|

此配置适用于想要完全自定义左侧机构或团队信息的情况。

\subsubsection{示例 4:混合配置}

\begin{lstlisting}[style=blockstyle,language=TeX]
\iqbfootline{自定义文本}{section}{zhpageofy}
\end{lstlisting}

用户可灵活组合三个参数以满足不同的演示需求。

\subsection{高级用法}

\subsubsection{设置总页数(可选)}

默认情况下,模板会自动检测文档的总页数。如需手动覆盖,可使用:

\begin{lstlisting}[style=blockstyle,language=TeX]
\iqbfoottotal{25}  % 强制将总页数显示为 25
\end{lstlisting}

此命令适用于以下场景:
\begin{itemize}
  \item 生成的 PDF 包含不计入页码的附录页
  \item 希望显示实际讲解页数而非技术总页数
\end{itemize}

\subsection{常见问题}

\textbf{Q: 为什么页脚没有显示?}

A: 请检查以下几点:
\begin{enumerate}
  \item \lstinline|\iqbfootline| 命令必须在 \lstinline|\begin{document}| 之前调用
  \item 确保已加载 \lstinline|beamerthemeiqb| 主题包
  \item 某些特殊页面(如 \lstinline|[plain]| 选项的页面)可能不显示页脚
\end{enumerate}

\textbf{Q: 如何在部分页面隐藏页脚?}

A: 使用 Beamer 的 \lstinline|plain| 选项创建不显示页脚的页面:

\begin{lstlisting}[style=blockstyle,language=TeX]
\begin{frame}[plain]
  \centering
  {\Huge Thank You!}
\end{frame}
\end{lstlisting}

\textbf{Q: 中文文本在页脚显示不正常?}

A: 确保使用 XeLaTeX 编译器并已正确配置中文字体支持(见第 \ref{sec:cjk} 节)。

