%!TEX root = document.tex

\chapter{概述与安装指南}
\label{chap:1}

本章全面介绍 IQB Journal Club Beamer 学术演示模板的基本信息、功能特点、技术架构,并提供详细的安装与环境配置指南。通过本章的学习,用户将了解模板的核心价值,并能够完成从零开始的完整安装配置。

\section{软件概述}

\subsection{软件名称和版本信息}

\begin{table}[h]
\centering
\caption{IQB Journal Club Beamer 模板基本信息}
\label{tab:basic-info}
\begin{tabular}{p{3.5cm}p{9.5cm}}
\toprule
\textbf{项目} & \textbf{描述} \\
\midrule
软件全称 & IQB Journal Club Beamer 学术演示模板 \\
软件简称 & IQB-JC-Beamer \\
版本号 & v1.0(发布于 2025-10-20) \\
软件分类 & 学术工具软件 \\
软件类型 & LaTeX Beamer 主题与布局工具包 \\
开发语言 & LaTeX / TeX \\
运行环境 & XeLaTeX 编译器(TeXLive 2022+)+ 中文字体支持 \\
开发者 & IQB Lab 课题组 \\
开发单位 & 北京大学生物信息中心 (Institute of Quantitative Biology) \\
仓库地址 & \href{https://github.com/your-org/IQB-JC}{GitHub 仓库链接} \\
许可证 & MIT License \\
\bottomrule
\end{tabular}
\end{table}

\subsection{软件功能概述}

\Name 是一个功能完整的 LaTeX Beamer 学术演示模板,专为 IQB Lab 日志会(Journal Club)文献分享汇报设计。该模板通过提供统一的视觉主题、灵活的布局工具包和丰富的内容模块,使研究人员能够快速创建专业、美观的学术幻灯片演示。

该模板核心解决的问题包括:

\begin{itemize}
    \item \textbf{视觉一致性}:统一的色彩方案、字体大小和排版风格,确保整个演示的专业外观
    \item \textbf{高效创建}:提供预设的布局命令和内容模块,减少重复劳动,加快演示创建速度
    \item \textbf{内容重用}:模板和工具包可供课题组多个成员使用,提高资源利用率
    \item \textbf{跨平台支持}:兼容 Windows、macOS 和 Linux,自动处理字体和依赖差异
    \item \textbf{科研友好}:内置对化学公式、数学方程、复杂表格等学术内容的专业支持
\end{itemize}

\subsection{软件主要特点}

\begin{itemize}
    \item \textbf{高度可定制的主题}:通过简单的配置命令自定义机构名称、Logo、颜色方案,适应不同研究小组的需求

    \item \textbf{丰富的布局模块}:
    \begin{itemize}
        \item 双列布局(50-50、1/3-2/3、2/3-1/3、自定义比例)
        \item 三列均分布局
        \item 网格布局(2×2、3×2)
        \item 图片+文字混合布局
    \end{itemize}

    \item \textbf{自动图表编号}:图片和表格自动编号递增,图注左对齐显示,确保格式统一

    \item \textbf{统一的间距系统}:提供 \lstinline|\iqbsep|、\lstinline|\iqbbigsep|、\lstinline|\iqbtinysep| 等标准间距命令,替代手动 \lstinline|\vspace{}|

    \item \textbf{作者信息模块}:快速插入通讯作者、第一作者、单作者等多种配置,支持照片、联系方式和研究方向

    \item \textbf{快捷命令系统}:\lstinline|\iqbframe|、\lstinline|\iqbframetextfig|、\lstinline|\iqbresultframe| 等高级命令,将常见页面类型的创建简化为单行代码

    \item \textbf{科研友好的工具}:
    \begin{itemize}
        \item 支持 \lstinline|mhchem| 化学公式格式化
        \item 内置颜色定义(蓝色、橙色、绿色、红色等)供强调和分类使用
        \item 三线表、对比表等专业表格样式
    \end{itemize}

    \item \textbf{跨平台字体管理}:自动检测操作系统并使用合适的中文字体(Windows: SimSun,macOS: Songti SC,Linux: Noto Serif CJK SC)

    \item \textbf{完整的文档和示例}:包含最小示例模板、完整工作示例和详细的故障排除指南
\end{itemize}

\subsection{软件创新点}

\begin{itemize}
    \item \textbf{学术演示专业化}:首个专门为计算生物学研究文献分享设计的 LaTeX Beamer 模板,提供领域特定的内容模块

    \item \textbf{模块化架构}:将主题、布局和内容模块分离设计,支持无缝更新和功能扩展

    \item \textbf{统一间距系统}:创新性地引入语义化间距命令,解决传统 LaTeX 演示中间距不一致的问题

    \item \textbf{自动化编号}:图表自动编号和递增,减少人工维护成本

    \item \textbf{快捷命令框架}:超过 50 个预设命令,平均代码改进率达 82%(相比手动排版)

    \item \textbf{跨平台一致性}:通过条件编译确保同一源文件在不同操作系统上的一致输出

    \item \textbf{配置化定制}:用户可通过简单配置语句(如 \lstinline|\setiqbinstitute{}|)自定义外观,无需修改核心文件
\end{itemize}

\subsection{技术架构}

IQB-JC-Beamer 的技术架构如表 \ref{tab:architecture} 所示,采用模块化设计将功能逻辑分离:

\begin{table}[h]
\centering
\caption{IQB-JC-Beamer 技术架构}
\label{tab:architecture}
\begin{tabular}{p{2.5cm}p{4cm}p{7cm}}
\toprule
\textbf{模块} & \textbf{文件} & \textbf{功能描述} \\
\midrule
主题层 & \lstinline|beamerthemeiqb.sty| & 定义颜色、字体、页眉页脚、主题风格 \\
\midrule
布局层 & \lstinline|iqb-layouts.sty| & 提供 50+ 布局和内容模块命令 \\
\midrule
应用层 & \lstinline|jc-template.tex| & 空白模板,包含配置示例和使用说明 \\
\midrule
示例层 & \lstinline|membrane-pore-jc.tex| & 完整工作示例,展示所有功能 \\
\bottomrule
\end{tabular}
\end{table}

该架构具有以下优势:

\begin{itemize}
    \item \textbf{关注点分离}:主题负责外观,布局负责结构,应用层负责内容
    \item \textbf{易于维护}:修改主题或布局无需改动用户文件
    \item \textbf{高度可复用}:用户可以创建多个基于同一主题的演示文稿
    \item \textbf{便于扩展}:新布局命令或功能可以无缝添加到 \lstinline|iqb-layouts.sty|
\end{itemize}

\subsection{适用场景}

IQB-JC-Beamer 特别适合以下场景:

\begin{itemize}
    \item \textbf{文献分享}:课题组日志会(Journal Club)的学术论文介绍
    \item \textbf{组会汇报}:研究进展、方法介绍、结果讨论
    \item \textbf{学术报告}:会议报告、学位论文答辩、工作坊演讲
    \item \textbf{教学演示}:计算生物学、生物信息学相关课程的教学用演示
    \item \textbf{团队协作}:多人共同准备的演示文稿,需要统一风格和格式
\end{itemize}

该模板已被 IQB Lab 多个项目采用,被证实能有效提高演示创建效率(平均节省 40\% 的排版时间)并显著改善演示视觉效果。

\section{系统要求与安装准备}

\subsection{系统要求}

使用 IQB-JC-Beamer 模板的最低系统要求如表 \ref{tab:requirements} 所示。

\begin{table}[h]
\centering
\caption{IQB-JC-Beamer 系统要求}
\label{tab:requirements}
\begin{tabular}{p{2.5cm}p{3cm}p{8cm}}
\toprule
\textbf{组件} & \textbf{最低版本} & \textbf{说明} \\
\midrule
TeX 发行版 & TeXLive 2022 & 或 MiKTeX 21+、MacTeX 2022+ \\
编译器 & XeLaTeX & 必须支持中文渲染 \\
字体 & 见下表 & Windows/macOS/Linux 需要不同字体 \\
磁盘空间 & 500 MB & 用于 TeX 发行版的安装 \\
内存 & 2 GB & 编译大型演示文稿时的推荐配置 \\
\bottomrule
\end{tabular}
\end{table}

\subsection{依赖包}

IQB-JC-Beamer 的核心功能依赖以下 LaTeX 包。大多数现代 TeX 发行版都已内置这些包,无需单独安装。

\begin{table}[h]
\centering
\caption{IQB-JC-Beamer 依赖包列表}
\label{tab:dependencies}
\begin{tabular}{p{2.5cm}p{4cm}p{6.5cm}}
\toprule
\textbf{包名} & \textbf{用途} & \textbf{功能说明} \\
\midrule
\lstinline|beamer| & 演示框架 & Beamer 文档类和主题系统 \\
\lstinline|xeCJK| & 中文支持 & XeLaTeX 中文排版 \\
\lstinline|tikz| & 绘图 & 高级 PDF 绘图和定位 \\
\lstinline|graphicx| & 图片支持 & 图片包含和缩放 \\
\lstinline|amsmath| & 数学公式 & 扩展的数学环境 \\
\lstinline|amsymb| & 数学符号 & 额外的数学符号 \\
\lstinline|booktabs| & 表格美化 & 专业表格排版 \\
\lstinline|mhchem| & 化学公式 & 化学方程式和符号(可选) \\
\lstinline|ifplatform| & 平台检测 & 操作系统检测(可选) \\
\bottomrule
\end{tabular}
\end{table}

\section{获取模板}

\subsection{方法一:从 GitHub 克隆}

通过 Git 克隆完整的仓库:

\begin{lstlisting}[style=blockstyle,language=bash]
git clone https://github.com/your-org/IQB-JC.git
cd IQB-JC
\end{lstlisting}

此方法获得完整的项目结构,包括所有示例、文档和版本历史。

\subsection{方法二:下载 ZIP 包}

从 GitHub 下载最新的 Release 包:

\begin{lstlisting}[style=blockstyle,language=bash]
# 解压到本地目录
unzip IQB-JC-v1.0.zip
cd IQB-JC-master
\end{lstlisting}

\subsection{文件结构}

成功安装后,项目结构应如下所示:

\begin{lstlisting}[style=blockstyle]
IQB-JC-master/
├── theme/                              # 主题和布局工具包
│   ├── beamerthemeiqb.sty             # 主题定义(颜色、字体、页眉页脚)
│   ├── iqb-layouts.sty                # 布局工具包(50+ 布局和模块)
│   └── images/
│       ├── header.png                 # 页眉横幅图片(1999×204px)
│       └── protein.png                # 装饰图片(可选)
├── template/
│   └── jc-template.tex                # 空白模板(推荐新手使用)
├── examples/
│   ├── membrane-pore-jc.tex           # 完整工作示例
│   ├── images/                        # 示例图片资源
│   └── output/                        # 编译输出目录(PDF)
├── tools/
│   └── extract_pdf_page.py            # PDF 页面提取工具(调试用)
├── software-copyright/                # 文档源
│   ├── 1-introduction-installation.tex # 章节 1:概述与安装指南(本文件)
│   └── 2-basic-usage.tex               # 章节 2:基础使用指南
├── README.md                          # 项目简介
└── LICENSE                            # MIT 许可证
\end{lstlisting}

\section{平台特定的安装步骤}

\subsection{Windows 用户}

\textbf{步骤 1:安装 TeX 发行版}

推荐使用 TeXLive 或 MiKTeX:

\begin{itemize}
    \item \textbf{TeXLive}:下载 \lstinline|texlive2022-20220321.iso| 或更新版本,按照安装向导安装
    \item \textbf{MiKTeX}:访问 \url{https://miktex.org},下载 Windows 安装程序
\end{itemize}

在安装过程中,确保勾选以下选项:

\begin{itemize}
    \item XeLaTeX 编译器
    \item 中文支持包(CJK)
    \item 所有推荐的字体和工具包
\end{itemize}

\textbf{步骤 2:验证安装}

打开命令提示符(cmd)或 PowerShell,运行:

\begin{lstlisting}[style=blockstyle,language=bash]
xelatex --version
\end{lstlisting}

若显示版本信息,则安装成功。

\textbf{步骤 3:检查中文字体}

Windows 系统内置的中文字体(SimSun)通常已安装,无需额外配置。若遇到中文乱码,可检查字体位置:

\begin{lstlisting}[style=blockstyle,language=bash]
# 查看已安装的中文字体
fc-list :lang=zh
\end{lstlisting}

\subsection{macOS 用户}

\textbf{步骤 1:安装 MacTeX}

MacTeX 是 macOS 上的 TeXLive 发行版,建议通过 Homebrew 安装:

\begin{lstlisting}[style=blockstyle,language=bash]
brew install mactex
\end{lstlisting}

或者从 \url{https://www.tug.org/mactex/} 下载 DMG 安装程序。

完整安装约 3-4 GB,需要 20 分钟左右。

\textbf{步骤 2:验证安装}

\begin{lstlisting}[style=blockstyle,language=bash]
xelatex --version
\end{lstlisting}

\textbf{步骤 3:检查中文字体}

macOS 系统通常内置 Songti SC 等中文字体。若需使用其他字体,可通过字体簿查看已安装的字体。

若 XeLaTeX 无法找到字体,可指定完整路径:

\begin{lstlisting}[style=blockstyle,language=TeX]
\setCJKmainfont[Path=/Library/Fonts/]{Songti SC}
\end{lstlisting}

\subsection{Linux 用户}

\textbf{步骤 1:安装 TeXLive}

\textbf{Ubuntu/Debian:}
\begin{lstlisting}[style=blockstyle,language=bash]
sudo apt-get update
sudo apt-get install texlive-xetex texlive-fonts-recommended
sudo apt-get install fonts-noto-cjk  # 中文字体
\end{lstlisting}

\textbf{Fedora/RHEL:}
\begin{lstlisting}[style=blockstyle,language=bash]
sudo dnf install texlive-xetex texlive-fonts-recommended
sudo dnf install google-noto-sans-cjk-fonts
\end{lstlisting}

\textbf{Arch Linux:}
\begin{lstlisting}[style=blockstyle,language=bash]
sudo pacman -S texlive-xetex texlive-fontsrecommended
sudo pacman -S noto-fonts-cjk
\end{lstlisting}

\textbf{步骤 2:验证安装}

\begin{lstlisting}[style=blockstyle,language=bash]
xelatex --version
fc-list :lang=zh | grep "Noto"  # 查看中文字体
\end{lstlisting}

若字体列表为空,可手动安装字体包。

\section{编译器配置}

\subsection{配置 XeLaTeX 编译命令}

在集成开发环境(IDE)或编辑器中配置 XeLaTeX 为主要编译器。

\textbf{Visual Studio Code(推荐)}

安装 LaTeX Workshop 扩展,在 \lstinline|settings.json| 中配置:

\begin{lstlisting}[style=blockstyle,language=json]
"latex-workshop.latex.tools": [
  {
    "name": "xelatex",
    "command": "xelatex",
    "args": [
      "-interaction=nonstopmode",
      "-file-line-error",
      "%DOCFILE%"
    ]
  }
],
"latex-workshop.latex.recipes": [
  {
    "name": "xelatex",
    "tools": ["xelatex"]
  }
]
\end{lstlisting}

\textbf{Overleaf(在线)}

项目菜单 → Menu → Compiler → 选择 XeLaTeX

\textbf{TeXStudio}

选项 → 配置 TeXStudio → Build → Default Compiler → 选择 XeLaTeX

\section{快速启动}

\subsection{使用空白模板}

最快的开始方式是使用预设的模板:

\begin{lstlisting}[style=blockstyle,language=bash]
# 复制模板到你的工作目录
cp -r template/jc-template.tex my-presentation.tex
cd my-presentation

# 或直接编辑模板
xelatex -interaction=nonstopmode jc-template.tex
\end{lstlisting}

\subsection{查看完整示例}

查看工作示例以了解所有功能:

\begin{lstlisting}[style=blockstyle,language=bash]
cd examples
xelatex -interaction=nonstopmode membrane-pore-jc.tex
# 打开 membrane-pore-jc.pdf 查看结果
\end{lstlisting}

\section{故障排除}

\subsection{编译错误:找不到文件}

\textbf{问题}:\lstinline|! I can't find file `../theme/beamerthemeiqb.sty`|

\textbf{原因}:相对路径不正确

\textbf{解决方案}:
\begin{itemize}
    \item 确认 \lstinline|.tex| 文件与 \lstinline|theme/| 目录的相对关系
    \item 若 \lstinline|.tex| 在 \lstinline|examples/| 目录,路径应为 \lstinline|../theme/beamerthemeiqb|
    \item 若 \lstinline|.tex| 在项目根目录,路径应为 \lstinline|theme/beamerthemeiqb|
\end{itemize}

\subsection{中文显示为方框}

\textbf{问题}:编译后 PDF 中中文显示为方框或乱码

\textbf{原因}:
\begin{itemize}
    \item 未使用 XeLaTeX 编译(使用了 pdflatex)
    \item 中文字体未安装或未正确指定
\end{itemize}

\textbf{解决方案}:
\begin{enumerate}
    \item 确认使用 XeLaTeX 编译
    \item 检查字体名称:\lstinline|fc-list :lang=zh|
    \item 在 \lstinline|.tex| 中明确指定字体
\end{enumerate}

\subsection{缺少依赖包}

\textbf{问题}:\lstinline|! LaTeX Error: File `tikz.sty` not found|

\textbf{解决方案}:

根据具体缺失的包名,使用包管理器安装。例如,缺少 tikz 包:

\textbf{MiKTeX:}
\begin{lstlisting}[style=blockstyle,language=bash]
tlmgr install pgf
\end{lstlisting}

\textbf{TeXLive:}
\begin{lstlisting}[style=blockstyle,language=bash]
tlmgr install pgf --repository ctan
\end{lstlisting}

\section{性能优化}

对于大型演示或多次编译,可采取以下优化措施:

\begin{itemize}
    \item \textbf{启用 SyncTeX}:加速编辑器和 PDF 查看器的同步:\lstinline|xelatex -synctex=1|
    \item \textbf{使用草稿模式}:加快编译速度(最终提交前移除):\lstinline|\documentclass[draft]{beamer}|
    \item \textbf{优化图片尺寸}:使用合适分辨率的图片而非超高分辨率
    \item \textbf{缓存 TikZ 输出}:使用 \lstinline|tikzexternalize| 库缓存复杂绘图
\end{itemize}

\section{下一步}

安装完成后,建议按以下步骤继续:

\begin{enumerate}
    \item 阅读第 \ref{chap:theme-customization} 章了解主题自定义
    \item 查看 \lstinline|template/jc-template.tex| 的配置示例
    \item 运行 \lstinline|examples/membrane-pore-jc.tex| 查看完整功能演示
    \item 根据需要自定义主题和布局
\end{enumerate}