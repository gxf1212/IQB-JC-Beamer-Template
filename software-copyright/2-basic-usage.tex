%!TEX root = document.tex

\chapter{基础使用与核心模块}
\label{chap:2}

\Name 是一个功能完整的学术演示文稿 LaTeX Beamer 模板,专为 IQB Lab 日志会(Journal Club)汇报设计。通过提供预设的主题、布局工具包和丰富的内容模块,用户可以快速创建专业、美观的学术幻灯片演示。本章详细说明该模板的基础使用方法和核心功能模块。

\section{基础使用与最小示例}

\subsection{最小示例代码}

以下是创建一个完整演示文稿的最小示例。用户可以以此为起点进行扩展:

\begin{lstlisting}[style=blockstyle,language=TeX]
\documentclass[aspectratio=169,11pt]{beamer}

% 加载 IQB 主题和布局工具包
\usepackage{theme/beamerthemeiqb}
\usepackage{theme/iqb-layouts}

% 设置 header 图片路径
\renewcommand{\iqbheaderimage}{theme/images/header.png}

% 中文支持(使用 XeLaTeX 编译)
\usepackage{xeCJK}
\setCJKmainfont{SimSun}

% 文档元数据
\title{你的演示标题}
\subtitle{副标题(可选)}
\author{你的名字}
\institute{IQB Lab}
\stuid{学号}                    % 可选:学生学号
\major{专业名称}                 % 可选:专业/学位类型
\advisor{导师名字 教授}          % 可选:指导教师名字
\dateformat{zh}                % 可选:zh 为中文日期,en 为英文日期(默认英文)
\date{\today}

\begin{document}

% 封面页(无 header/footer)
\begin{frame}[plain,noframenumbering]
  \titlepage
\end{frame}

% 内容页示例
\setsection{Background}
\begin{frame}{第一页内容}
  \begin{itemize}
    \item 要点 1
    \item 要点 2
    \item 要点 3
  \end{itemize}
\end{frame}

% 致谢页(无 header/footer)
\begin{frame}[plain]
  \centering
  {\Huge Thank You!}
\end{frame}

\end{document}
\end{lstlisting}

\subsection{编译方法}

使用 XeLaTeX 编译器(推荐,支持中文):

\begin{lstlisting}[style=blockstyle,language=bash]
xelatex -interaction=nonstopmode your-presentation.tex
\end{lstlisting}

若需要生成含交叉引用的完整 PDF,需运行两次编译:

\begin{lstlisting}[style=blockstyle,language=bash]
xelatex -interaction=nonstopmode your-presentation.tex
xelatex -interaction=nonstopmode your-presentation.tex
\end{lstlisting}

\subsection{文档类选项说明}

表 \ref{tab:documentclass-options} 列出了常用的文档类选项及其含义。

\begin{table}[!h]
\centering
\caption{Beamer 文档类常用选项}
\label{tab:documentclass-options}
\begin{tabular}{p{3.5cm}p{9.5cm}}
\toprule
\textbf{选项} & \textbf{说明} \\
\midrule
\lstinline|aspectratio=169| & 设置幻灯片宽高比为 16:9(推荐用于现代显示器) \\
\lstinline|aspectratio=43| & 设置幻灯片宽高比为 4:3(传统比例) \\
\lstinline|11pt| & 设置基础字体大小为 11pt \\
\lstinline|12pt| & 设置基础字体大小为 12pt(默认) \\
\bottomrule
\end{tabular}
\end{table}

\subsection{中文支持配置}

\textbf{跨平台中文字体配置(自动)}

模板已内置跨平台中文字体自动配置,用户无需手动设置。\lstinline|beamerthemeiqb.sty| 会根据操作系统自动选择最优字体:

\begin{table}[!h]
\centering
\caption{自动CJK字体配置(按操作系统)}
\label{tab:auto-cjk-fonts}
\begin{tabular}{p{2cm}p{3cm}p{3cm}p{3cm}}
\toprule
\textbf{操作系统} & \textbf{正文字体} & \textbf{加粗字体} & \textbf{说明} \\
\midrule
Windows & SimSun(宋体) & SimHei(黑体) & 系统内置,无需额外配置 \\
macOS & Songti SC & STHeiti(黑体) & 系统内置,无需额外配置 \\
Linux & Noto Serif CJK SC & Noto Sans CJK SC(思源黑体) & 需先安装字体包 \\
\bottomrule
\end{tabular}
\end{table}

\textbf{中文加粗支持}

重要改进:现在 \lstinline|\textbf{中文}| 能够正确显示加粗效果!这是通过在各个平台配置对应的黑体字体实现的。

使用示例:

\begin{lstlisting}[style=blockstyle,language=TeX]
\textbf{加粗的中文文字}    % 现在能正确显示加粗

\textbf{关键要点}:描述内容

{\bfseries 也可以用这种方式} 加粗
\end{lstlisting}

\textbf{手动配置(可选)}

如果需要覆盖自动配置,可在文档导言区手动指定字体:

\begin{lstlisting}[style=blockstyle,language=TeX]
% Windows 用户
\setCJKmainfont[BoldFont=SimHei]{SimSun}

% macOS 用户
\setCJKmainfont[BoldFont=STHeiti]{Songti SC}

% Linux 用户(需先安装 fonts-noto-cjk)
\setCJKmainfont[BoldFont=Noto Sans CJK SC]{Noto Serif CJK SC}
\end{lstlisting}

\textbf{Linux 用户注意事项}

若 Linux 系统中文字体缺失,运行以下命令安装:

\begin{lstlisting}[style=blockstyle,language=bash]
sudo apt-get install fonts-noto-cjk    % Ubuntu/Debian
sudo pacman -S noto-fonts-cjk           % Arch Linux
\end{lstlisting}

\section{页面结构控制}

IQB-JC Beamer 模板具有三种页面类型:封面页、标准内容页和特殊页面(如致谢页)。

\subsection{页面类型详解}

表 \ref{tab:frame-types} 总结了不同页面类型的特性及适用场景。

\begin{table}[!h]
\centering
\caption{IQB-JC 模板中的页面类型}
\label{tab:frame-types}
\begin{tabular}{p{2.5cm}p{2.5cm}p{2.5cm}p{4cm}}
\toprule
\textbf{页面类型} & \textbf{Header 显示} & \textbf{Footer 显示} & \textbf{应用场景} \\
\midrule
标准内容页 & 是 & 是 & 演示主要内容 \\
Plain 页面 & 否 & 否 & 封面、致谢、转场页 \\
\bottomrule
\end{tabular}
\end{table}

\subsection{标准内容页}

标准内容页同时显示 header 横幅和 footer 三段式设计。使用如下方式创建:

\begin{lstlisting}[style=blockstyle,language=TeX]
\setsection{Methods}  % 设置 footer 中间的 section 标识
\begin{frame}{页面标题}
  页面内容
\end{frame}
\end{lstlisting}

其中 \lstinline|\setsection{}| 命令设置 footer 中间部分显示的章节名称。该设置对后续所有页面有效,直到被重新设置。

\subsection{Plain 页面(无 Header/Footer)}

Plain 页面隐藏 header 和 footer,适用于封面、致谢等特殊页面。创建方式如下:

\begin{lstlisting}[style=blockstyle,language=TeX]
\begin{frame}[plain,noframenumbering]
  \titlepage
\end{frame}
\end{lstlisting}

其中:

\begin{itemize}
    \item \lstinline|[plain]| - 隐藏 header、footer 和页码
    \item \lstinline|[noframenumbering]| - 不计入总页数统计(适用于封面、致谢等)
\end{itemize}

\subsection{Footer Section 设置}

Footer 底部的三段式设计包含:左侧固定文字 "IQB Lab"、中间 section 标识、右侧页码显示。使用 \lstinline|\setsection{}| 命令管理中间部分:

\begin{lstlisting}[style=blockstyle,language=TeX]
\setsection{Background}
\begin{frame}{背景介绍}
  % footer 中间显示 "Background"
\end{frame}

\setsection{Methods}
\begin{frame}{研究方法}
  % footer 中间显示 "Methods"
\end{frame}
\end{lstlisting}

建议在每个主要章节的第一页调用 \lstinline|\setsection{}| 命令以保持 footer 信息的准确性。