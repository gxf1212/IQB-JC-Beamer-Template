%!TEX root = document.tex

\chapter{基础使用与核心模块}
\label{chap:basic-usage}

\Name 是一个功能完整的学术演示文稿 LaTeX Beamer 模板,专为 IQB Lab 日志会(Journal Club)汇报设计。通过提供预设的主题、布局工具包和丰富的内容模块,用户可以快速创建专业、美观的学术幻灯片演示。本章详细说明该模板的基础使用方法和核心功能模块。

\section{基础使用与最小示例}

\subsection{最小示例代码}

以下是创建一个完整演示文稿的最小示例。用户可以以此为起点进行扩展:

\begin{lstlisting}[style=blockstyle,language=TeX]
\documentclass[aspectratio=169,11pt]{beamer}

% 加载 IQB 主题和布局工具包
\usepackage{theme/beamerthemeiqb}
\usepackage{theme/iqb-layouts}

% 设置 header 图片路径
\renewcommand{\iqbheaderimage}{theme/images/header.png}

% 中文支持(使用 XeLaTeX 编译)
\usepackage{xeCJK}
\setCJKmainfont{SimSun}

% 文档元数据
\title{你的演示标题}
\subtitle{副标题(可选)}
\author{你的名字}
\institute{IQB Lab}
\stuid{学号}                    % 可选:学生学号
\major{专业名称}                 % 可选:专业/学位类型
\advisor{导师名字 教授}          % 可选:指导教师名字
\dateformat{zh}                % 可选:zh 为中文日期,en 为英文日期(默认英文)
\date{\today}

\begin{document}

% 封面页(无 header/footer)
\begin{frame}[plain,noframenumbering]
  \titlepage
\end{frame}

% 内容页示例
\setsection{Background}
\begin{frame}{第一页内容}
  \begin{itemize}
    \item 要点 1
    \item 要点 2
    \item 要点 3
  \end{itemize}
\end{frame}

% 致谢页(无 header/footer)
\begin{frame}[plain]
  \centering
  {\Huge Thank You!}
\end{frame}

\end{document}
\end{lstlisting}

\subsection{编译方法}

使用 XeLaTeX 编译器(推荐,支持中文):

\begin{lstlisting}[style=blockstyle,language=bash]
xelatex -interaction=nonstopmode your-presentation.tex
\end{lstlisting}

若需要生成含交叉引用的完整 PDF,需运行两次编译:

\begin{lstlisting}[style=blockstyle,language=bash]
xelatex -interaction=nonstopmode your-presentation.tex
xelatex -interaction=nonstopmode your-presentation.tex
\end{lstlisting}

\subsection{文档类选项说明}

表 \ref{tab:documentclass-options} 列出了常用的文档类选项及其含义。

\begin{table}[!h]
\centering
\caption{Beamer 文档类常用选项}
\label{tab:documentclass-options}
\begin{tabular}{p{3.5cm}p{9.5cm}}
\toprule
\textbf{选项} & \textbf{说明} \\
\midrule
\lstinline|aspectratio=169| & 设置幻灯片宽高比为 16:9(推荐用于现代显示器) \\
\lstinline|aspectratio=43| & 设置幻灯片宽高比为 4:3(传统比例) \\
\lstinline|11pt| & 设置基础字体大小为 11pt \\
\lstinline|12pt| & 设置基础字体大小为 12pt(默认) \\
\bottomrule
\end{tabular}
\end{table}

\subsection{中文支持配置}

\textbf{跨平台中文字体配置(自动)}

模板已内置跨平台中文字体自动配置,用户无需手动设置。\lstinline|beamerthemeiqb.sty| 会根据操作系统自动选择最优字体:

\begin{table}[!h]
\centering
\caption{自动CJK字体配置(按操作系统)}
\label{tab:auto-cjk-fonts}
\begin{tabular}{p{2cm}p{3cm}p{3cm}p{3cm}}
\toprule
\textbf{操作系统} & \textbf{正文字体} & \textbf{加粗字体} & \textbf{说明} \\
\midrule
Windows & SimSun(宋体) & SimHei(黑体) & 系统内置,无需额外配置 \\
macOS & Songti SC & STHeiti(黑体) & 系统内置,无需额外配置 \\
Linux & Noto Serif CJK SC & Noto Sans CJK SC(思源黑体) & 需先安装字体包 \\
\bottomrule
\end{tabular}
\end{table}

\textbf{中文加粗支持}

重要改进:现在 \lstinline|\textbf{中文}| 能够正确显示加粗效果!这是通过在各个平台配置对应的黑体字体实现的。

使用示例:

\begin{lstlisting}[style=blockstyle,language=TeX]
\textbf{加粗的中文文字}    % 现在能正确显示加粗

\textbf{关键要点}:描述内容

{\bfseries 也可以用这种方式} 加粗
\end{lstlisting}

\textbf{手动配置(可选)}

如果需要覆盖自动配置,可在文档导言区手动指定字体:

\begin{lstlisting}[style=blockstyle,language=TeX]
% Windows 用户
\setCJKmainfont[BoldFont=SimHei]{SimSun}

% macOS 用户
\setCJKmainfont[BoldFont=STHeiti]{Songti SC}

% Linux 用户(需先安装 fonts-noto-cjk)
\setCJKmainfont[BoldFont=Noto Sans CJK SC]{Noto Serif CJK SC}
\end{lstlisting}

\textbf{Linux 用户注意事项}

若 Linux 系统中文字体缺失,运行以下命令安装:

\begin{lstlisting}[style=blockstyle,language=bash]
sudo apt-get install fonts-noto-cjk    % Ubuntu/Debian
sudo pacman -S noto-fonts-cjk           % Arch Linux
\end{lstlisting}

\section{页面结构控制}

IQB-JC Beamer 模板具有三种页面类型:封面页、标准内容页和特殊页面(如致谢页)。

\subsection{页面类型详解}

表 \ref{tab:frame-types} 总结了不同页面类型的特性及适用场景。

\begin{table}[!h]
\centering
\caption{IQB-JC 模板中的页面类型}
\label{tab:frame-types}
\begin{tabular}{p{2.5cm}p{3cm}p{3cm}p{4cm}}
\toprule
\textbf{页面类型} & \textbf{Header 显示} & \textbf{Footer 显示} & \textbf{应用场景} \\
\midrule
标准内容页 & 是 & 是 & 演示主要内容 \\
Plain 页面 & 否 & 否 & 封面、致谢、转场页 \\
\bottomrule
\end{tabular}
\end{table}

\subsection{标准内容页}

标准内容页同时显示 header 横幅和 footer 三段式设计。使用如下方式创建:

\begin{lstlisting}[style=blockstyle,language=TeX]
\setsection{Methods}  % 设置 footer 中间的 section 标识
\begin{frame}{页面标题}
  页面内容
\end{frame}
\end{lstlisting}

其中 \lstinline|\setsection{}| 命令设置 footer 中间部分显示的章节名称。该设置对后续所有页面有效,直到被重新设置。

\subsection{Plain 页面(无 Header/Footer)}

Plain 页面隐藏 header 和 footer,适用于封面、致谢等特殊页面。创建方式如下:

\begin{lstlisting}[style=blockstyle,language=TeX]
\begin{frame}[plain,noframenumbering]
  \titlepage
\end{frame}
\end{lstlisting}

其中:

\begin{itemize}
\item \lstinline|[plain]| - 隐藏 header、footer 和页码
\item \lstinline|[noframenumbering]| - 不计入总页数统计(适用于封面、致谢等)
\end{itemize}

\subsection{Footer Section 设置}

Footer 底部的三段式设计包含:左侧固定文字 "IQB Lab"、中间 section 标识、右侧页码显示。使用 \lstinline|\setsection{}| 命令管理中间部分:

\begin{lstlisting}[style=blockstyle,language=TeX]
\setsection{Background}
\begin{frame}{背景介绍}
  % footer 中间显示 "Background"
\end{frame}

\setsection{Methods}
\begin{frame}{研究方法}
  % footer 中间显示 "Methods"
\end{frame}
\end{lstlisting}

建议在每个主要章节的第一页调用 \lstinline|\setsection{}| 命令以保持 footer 信息的准确性。

\section{主题自定义}

\subsection{修改主题颜色}

IQB-JC 模板的主题颜色定义在 \lstinline|theme/beamerthemeiqb.sty| 文件中。主要颜色包括:

\begin{table}[!h]
\centering
\caption{IQB-JC 主题颜色定义}
\label{tab:theme-colors}
\begin{tabular}{p{3cm}p{4cm}p{4cm}}
\toprule
\textbf{颜色名称} & \textbf{RGB 值} & \textbf{十六进制} \\
\midrule
\lstinline|iqbblue| & 0, 51, 102 & \#003366 \\
\lstinline|iqbgray| & 85, 85, 85 & \#555555 \\
\lstinline|iqblightgray| & 240, 240, 240 & \#F0F0F0 \\
\bottomrule
\end{tabular}
\end{table}

若要修改主题颜色,可在导言区添加颜色重定义:

\begin{lstlisting}[style=blockstyle,language=TeX]
\definecolor{iqbblue}{RGB}{0, 100, 200}  % 自定义蓝色
\definecolor{iqbgray}{RGB}{100, 100, 100}  % 自定义灰色
\end{lstlisting}

\subsection{替换 Header 图片}

Header 横幅图片可通过重新定义 \lstinline|\iqbheaderimage| 命令更改:

\begin{lstlisting}[style=blockstyle,language=TeX]
\renewcommand{\iqbheaderimage}{path/to/your/header.png}
\end{lstlisting}

建议的 header 图片规格:

\begin{itemize}
\item 分辨率:1999×204 像素(保持原始 1999:204 比例)
\item 宽高比:约 9.8:1
\item 格式:PNG 或 PDF(支持透明度)
\end{itemize}

\subsection{调整字体大小与字体层级系统}

IQB-JC 模板采用精心设计的 5 级字体层级体系,确保页面整体视觉平衡且易于阅读。以下表格展示了完整的字体层级及其应用场景:

\begin{table}[!h]
\centering
\caption{IQB-JC 完整字体层级体系}
\label{tab:font-sizes}
\begin{tabular}{p{2cm}p{4cm}p{2.5cm}p{5.5cm}}
\toprule
\textbf{层级} & \textbf{文本类型} & \textbf{字号} & \textbf{使用场景} \\
\midrule
1 & 封面页标题 (\lstinline|\title|) & 14.4pt & 演示文稿的总标题,仅在封面使用 \\
2 & Frame Title(页标题) & 12pt & 每页最上方的标题栏,所有内容页必有 \\
3 & Section Title(段标题) & 11pt & 页面内的分段标题,使用 \lstinline|\iqbsectiontitle{}| 命令 \\
4 & 正文/Itemize(默认) & 9pt & 页面主要内容、列表项等日常文字 \\
5 & 图注/表注/脚注 & 8pt & 较小的辅助说明文字 \\
\bottomrule
\end{tabular}
\end{table}

\subsubsection{分段标题命令 (\texttt{\textbackslash iqbsectiontitle})}

在页面内使用分段标题来组织内容结构。分段标题会自动应用 11pt 大小、IQB 蓝色及加粗样式:

\begin{lstlisting}[style=blockstyle,language=TeX]
\iqbsectiontitle{FEbuilder - 高通量自由能计算工具}

高度优化的自由能计算框架...
\end{lstlisting}

\subsubsection{字体模式切换 (\texttt{\textbackslash iqbfontsizemode})}

模板提供了两种字体模式,可根据内容密度灵活切换:

\begin{enumerate}
  \item \textbf{normal 模式(默认)}:正文 9pt,适用于内容较少或需要突出视觉效果的页面
  \item \textbf{small 模式}:所有字体缩小 1 级,适用于内容密集的页面
\end{enumerate}

在演示文稿中切换模式的用法:

\begin{lstlisting}[style=blockstyle,language=TeX]
% 在特定页面启用 small 模式(内容较多)
\begin{frame}{内容密集的页面}
  \iqbfontsizemode{small}

  \begin{itemize}
    \item 项目 1
    \item 项目 2
    % ...更多项目
  \end{itemize}
\end{frame}

% 后续页面自动恢复 normal 模式
\end{lstlisting}

\textbf{注意}:字体模式的改变仅在当前 frame 内有效,不会影响其他页面。

\subsubsection{列表环境(iqbitemize)}

IQB-JC 提供了增强版的列表环境 \lstinline|iqbitemize|,支持自动为带标签的项目应用蓝色加粗样式:

\begin{lstlisting}[style=blockstyle,language=TeX]
\begin{iqbitemize}
  \item[标签1] 带标签的项目内容
  \item[标签2] 另一个带标签的项目
  \item 不带标签的普通项目
\end{iqbitemize}
\end{lstlisting}

标签会自动呈现为 IQB 蓝色加粗文字,无需手动指定颜色或加粗。

\subsubsection{描述列表环境(iqbdescription)}

对于包含较长标签(4 字以上)的列表,推荐使用 \lstinline|iqbdescription| 环境。相比 \lstinline|iqbitemize|,\lstinline|iqbdescription| 具有以下优势:

\begin{itemize}
  \item 标签自动右对齐,视觉更整齐
  \item 支持更长的标签文本(最多可显示 4--6 个中文字符)
  \item 可通过参数 \lstinline|[距离]| 灵活调整列表的左边距
\end{itemize}

\paragraph{基本用法}

\begin{lstlisting}[style=blockstyle,language=TeX]
\begin{iqbdescription}
  \item[文献处理] 文献检索、自动写综述、文献讲解PPT生成
  \item[多智能体协作] 多智能体交互助力科研方案设计
  \item[个性化智能体] 私域知识库构建和增量学习
\end{iqbdescription}
\end{lstlisting}

\paragraph{调整左边距}

使用可选参数 \lstinline|[距离]| 可将列表向右移动指定距离,默认为 0em:

\begin{lstlisting}[style=blockstyle,language=TeX]
% 标准用法(无缩进)
\begin{iqbdescription}
  \item[标签1] 内容
\end{iqbdescription}

% 向右移动 2.5cm
\begin{iqbdescription}[2.5cm]
  \item[标签1] 内容
\end{iqbdescription}
\end{lstlisting}

\paragraph{样式说明}

\lstinline|iqbdescription| 中的标签具有以下特点:
\begin{itemize}
  \item 标签显示为 IQB 蓝色(\textcolor{iqbblue}{\textbf{示例}})
  \item 标签文字为加粗显示
  \item 标签在 1cm 宽度内右对齐,确保整齐的竖排对齐
\end{itemize}

\paragraph{何时使用}

\begin{table}[!h]
\centering
\begin{tabular}{p{3cm}p{4cm}p{4cm}}
\toprule
\textbf{列表类型} & \textbf{iqbitemize} & \textbf{iqbdescription} \\
\midrule
无标签或短标签 & \checkmark(推荐) & 可用(不推荐) \\
中等长度标签(2--3 字) & 可用 & \checkmark(推荐) \\
较长标签(4 字以上) & \text{\ding{55}}(不适合) & \checkmark(推荐) \\
需要灵活缩进 & \checkmark & \checkmark(更灵活) \\
\bottomrule
\end{tabular}
\end{table}

\subsubsection{手动字体调整}

若要进行更细致的字体定制,编辑 \lstinline|theme/beamerthemeiqb.sty| 文件中的 \lstinline|\setbeamerfont| 命令。例如,以下代码控制正文字体:

\begin{lstlisting}[style=blockstyle,language=TeX]
% normal 模式下的正文字体
\setbeamerfont{normal text}{size=\footnotesize}  % 9pt

% small 模式下的正文字体
\setbeamerfont{normal text}{size=\scriptsize}   % 8pt
\end{lstlisting}

\subsection{间距与排版控制}

为了保持演示文稿的视觉一致性和专业感,IQB-JC 提供了三个标准化的间距命令。相比手动使用 \lstinline|\vspace{}|,这些命令能更好地确保全文间距的统一性。

\subsubsection{标准间距命令}

IQB-JC 模板定义了以下三个间距命令,涵盖了常见的排版需求:

\begin{table}[!h]
\centering
\caption{间距命令一览表}
\label{tab:spacing-commands}
\begin{tabular}{p{2.5cm}p{1.5cm}p{7cm}}
\toprule
\textbf{命令} & \textbf{间距大小} & \textbf{使用场景} \\
\midrule
\lstinline|\iqbsep| & 0.3cm & 标准段落间距,最常用,用于分隔逻辑内容块 \\
\lstinline|\iqbbigsep| & 0.5cm & 大段落间距,用于主要分节之间的间隔 \\
\lstinline|\iqbtinysep| & 0.15cm & 小间距,用于列表项之间或紧凑内容区间 \\
\bottomrule
\end{tabular}
\end{table}

\subsubsection{使用示例}

以下示例展示了如何在演示文稿中使用这些间距命令:

\begin{lstlisting}[style=blockstyle,language=TeX]
\begin{frame}{多段落内容页面}
  \iqbsectiontitle{背景介绍}

  这是第一段内容,介绍了该部分的基本背景。

  \iqbsep  % 使用标准间距分隔段落

  \iqbsectiontitle{方法论}

  \begin{itemize}
    \item 方法一
    \iqbtinysep  % 使用小间距分隔列表项
    \item 方法二
    \iqbtinysep
    \item 方法三
  \end{itemize}

  \iqbbigsep  % 使用大间距分隔主要内容块

  \iqbsectiontitle{结论}

  这是最后一段内容...
\end{frame}
\end{lstlisting}

\subsubsection{间距与图片的配合}

当在演示文稿中插入图片时,常需要在图片与文字之间增加合适的间距。使用标准间距命令能确保整体布局的协调性:

\begin{lstlisting}[style=blockstyle,language=TeX]
\begin{frame}{图文结合布局}
  \iqbsectiontitle{研究成果}

  \iqbsep

  \iqbimgcenter[height=0.4\textheight]{images/result.png}

  \iqbtinysep

  \textbf{图 1:}实验结果展示了该方法的有效性。
\end{frame}
\end{lstlisting}

\subsubsection{推荐实践}

\begin{enumerate}
  \item \textbf{优先使用标准命令}:用 \lstinline|\iqbsep|、\lstinline|\iqbbigsep|、\lstinline|\iqbtinysep| 代替手动 \lstinline|\vspace{}|,以保持全文间距风格一致。

  \item \textbf{逻辑清晰}:用间距命令清晰地分隔演示文稿中的逻辑块(如不同章节、内容组)。

  \item \textbf{避免过度间距}:不要过度使用大间距(\lstinline|\iqbbigsep|),否则页面显得疏散;也不要过度使用小间距,否则内容显得拥挤。

  \item \textbf{与布局工具结合}:在使用 \lstinline|\iqblayouttwo|、\lstinline|\iqbfig| 等布局命令时,合理添加间距能提升整体视觉效果。
\end{enumerate}

\section{演示文稿元数据设置}

IQB-JC 模板支持丰富的演示文稿元数据字段,这些信息会自动显示在封面页上。

\subsection{基础字段}

以下字段是必填的,用于定义演示文稿的基本信息:

\begin{lstlisting}[style=blockstyle,language=TeX]
\title{演示标题}           % 封面主标题
\subtitle{副标题}           % 可选:副标题
\author{作者名字}           % 作者名字
\institute{机构/学院}       % 学院或研究所名称
\date{\today}               % 日期(\today 表示当前日期)
\end{lstlisting}

\subsection{扩展字段}

模板额外支持以下字段,用于学位申请、奖学金申请等场景。这些字段**完全可选**,如不设置则不显示:

\begin{lstlisting}[style=blockstyle,language=TeX]
\stuid{学号}               % 学号(如果设置则在封面显示)
\major{专业名称}           % 专业或学位类型(如果设置则在封面显示)
\advisor{导师名字 教授}    % 指导教师名字(如果设置则在封面显示)
\end{lstlisting}

使用示例:

\begin{lstlisting}[style=blockstyle,language=TeX]
\title{杨咏曼奖学金申请陈述}
\author{高旭帆}
\institute{生命科学学院 \quad 生物物理研究所}
\stuid{12207134}
\major{生物物理学}
\advisor{周如鸿 教授}
\end{lstlisting}

\subsection{日期格式设置}

模板支持中英文日期格式切换,通过 \lstinline|\dateformat{}| 命令控制:

\begin{lstlisting}[style=blockstyle,language=TeX]
% 使用中文日期格式(推荐用于中文演示)
\dateformat{zh}
\date{\today}           % 显示为:"2025 年 11 月 5 日"

% 使用英文日期格式(推荐用于英文演示)
\dateformat{en}
\date{\today}           % 显示为:"5 November 2025"
\end{lstlisting}

\textbf{注意}:\lstinline|\dateformat| 命令必须在 \lstinline|\date{}| 之前调用。

\section{多列布局模块}

IQB-JC 模板通过 \lstinline|iqb-layouts.sty| 包提供了多种预设布局工具,满足不同的内容排版需求。

\subsection{双列布局(50-50)}

命令格式:

\begin{lstlisting}[style=blockstyle,language=TeX]
\iqblayouttwo{左列内容}{右列内容}
\end{lstlisting}

该命令将页面平均分为两列,各占 48\% 宽度,中间留 4\% 间距。适用于并行展示两个相关内容的场景。使用示例:

\begin{lstlisting}[style=blockstyle,language=TeX]
\begin{frame}{双列对比}
  \iqblayouttwo{
    \textbf{方法 A}
    \begin{itemize}
      \item 优点 1
      \item 优点 2
    \end{itemize}
  }{
    \textbf{方法 B}
    \begin{itemize}
      \item 优点 1
      \item 优点 2
    \end{itemize}
  }
\end{frame}
\end{lstlisting}

\subsection{不对称双列布局}

\textbf{1/3 + 2/3 布局}

命令格式:

\begin{lstlisting}[style=blockstyle,language=TeX]
\iqblayoutonethird{左列内容(占 31\%)}{右列内容(占 65\%)}
\end{lstlisting}

该布局将页面分为 31\% 和 65\% 两列,适合在右侧放置大图,左侧放置说明文字。使用示例:

\begin{lstlisting}[style=blockstyle,language=TeX]
\begin{frame}{竖版图片布局}
  \iqblayoutonethird{
    \textbf{说明文字}

    该图展示了关键结果
  }{
    \includegraphics[height=0.6\textheight,keepaspectratio]{image.png}
  }
\end{frame}
\end{lstlisting}

\textbf{2/3 + 1/3 布局}

命令格式:

\begin{lstlisting}[style=blockstyle,language=TeX]
\iqblayouttwothirds{左列内容(占 65\%)}{右列内容(占 31\%)}
\end{lstlisting}

该布局与上述布局对称,适合在左侧放置大图,右侧放置说明文字。

\subsection{三列布局}

命令格式:

\begin{lstlisting}[style=blockstyle,language=TeX]
\iqblayoutthree{左列内容}{中列内容}{右列内容}
\end{lstlisting}

将页面平均分为三列,各占 31\% 宽度。适合展示三个平行的内容或对比三种方法。使用示例:

\begin{lstlisting}[style=blockstyle,language=TeX]
\begin{frame}{三方法对比}
  \iqblayoutthree{
    \textbf{方法 A}

    优点与缺点分析
  }{
    \textbf{方法 B}

    优点与缺点分析
  }{
    \textbf{方法 C}

    优点与缺点分析
  }
\end{frame}
\end{lstlisting}

\section{网格与图表布局}

\subsection{2×2 网格布局}

命令格式:

\begin{lstlisting}[style=blockstyle,language=TeX]
\iqbgridtwobytwo{左上}{右上}{左下}{右下}
\end{lstlisting}

该布局适合展示四个相关的图片或内容块,自动排列成 2×2 网格。使用示例:

\begin{lstlisting}[style=blockstyle,language=TeX]
\begin{frame}{四个关键结果}
  \iqbgridtwobytwo{
    \includegraphics[height=0.35\textheight]{fig1.png}
  }{
    \includegraphics[height=0.35\textheight]{fig2.png}
  }{
    \includegraphics[height=0.35\textheight]{fig3.png}
  }{
    \includegraphics[height=0.35\textheight]{fig4.png}
  }
\end{frame}
\end{lstlisting}

\subsection{3×2 网格布局}

命令格式:

\begin{lstlisting}[style=blockstyle,language=TeX]
\iqbgridthreebytwo{左上}{中上}{右上}{左下}{中下}{右下}
\end{lstlisting}

该布局展示六个内容块,排列为 3 列 2 行。各列占 31\% 宽度,垂直间距为 0.5em。

\section{图文混排模块}

\subsection{图片左、文字右}

命令格式:

\begin{lstlisting}[style=blockstyle,language=TeX]
\iqbimagetext[图片选项]{图片路径}{右侧文字内容}
\end{lstlisting}

默认图片选项为 \lstinline|width=0.45\textwidth|。使用示例:

\begin{lstlisting}[style=blockstyle,language=TeX]
\begin{frame}{实验结果}
  \iqbimagetext[width=0.4\textwidth]{experiment.png}{
    \textbf{关键发现:}
    \begin{itemize}
      \item 发现 1
      \item 发现 2
      \item 发现 3
    \end{itemize}
  }
\end{frame}
\end{lstlisting}

\subsection{文字左、图片右}

命令格式:

\begin{lstlisting}[style=blockstyle,language=TeX]
\iqbtextimage[图片选项]{左侧文字内容}{图片路径}
\end{lstlisting}

这是上述布局的镜像版本,文字显示在左列,图片显示在右列。使用示例:

\begin{lstlisting}[style=blockstyle,language=TeX]
\begin{frame}{方法流程}
  \iqbtextimage[height=0.55\textheight]{
    \textbf{研究步骤:}
    \begin{enumerate}
      \item 数据收集
      \item 预处理
      \item 分析
    \end{enumerate}
  }{flowchart.png}
\end{frame}
\end{lstlisting}

\section{辅助工具}

\subsection{PDF 转 PPTX(嵌入图片)}

\subsubsection{用途与特点}

\lstinline|pdf_to_pptx.py| 工具用于快速将编译后的 PDF 演示文稿转换为 PPTX 格式。该工具将 PDF 的每一页转换为高分辨率图片并嵌入到 PPTX 中,适用于以下场景:

\begin{itemize}
  \item 需要快速生成 PowerPoint 格式供他人浏览
  \item 保留 PDF 中的完整布局和样式(包括 header、footer、排版等)
  \item 无需在 PowerPoint 中编辑内容,只需要演示
\end{itemize}

\paragraph{使用方法}

基本用法:

\begin{lstlisting}[style=blockstyle,language=bash]
python tools/pdf_to_pptx.py examples/Xufan.pdf output.pptx
\end{lstlisting}

高分辨率转换(可选):

\begin{lstlisting}[style=blockstyle,language=bash]
python tools/pdf_to_pptx.py examples/Xufan.pdf output.pptx --dpi 300
\end{lstlisting}

添加页码注释:

\begin{lstlisting}[style=blockstyle,language=bash]
python tools/pdf_to_pptx.py examples/Xufan.pdf output.pptx --add-notes
\end{lstlisting}

\paragraph{系统依赖}

使用前需要安装必要的依赖:

\begin{lstlisting}[style=blockstyle,language=bash]
# Python 依赖
pip install pdf2image python-pptx pillow

# 系统依赖(Linux/WSL)
sudo apt install poppler-utils

# 系统依赖(macOS)
brew install poppler
\end{lstlisting}

\paragraph{限制}

因为是嵌入图片,生成的 PPTX:
\begin{itemize}
  \item 不可编辑(文字无法选中和修改)
  \item 文件较大(由于高分辨率图片)
  \item 无法提取或修改单个元素(如字体、颜色)
\end{itemize}

\subsection{PDF 提取页面为 PNG}

\subsubsection{用途与特点}

\lstinline|extract_pdf_page.py| 工具用于提取 PDF 中的指定页面并保存为 PNG 图片,用于视觉验证和调试。该工具常配合 pdf-layout-reviewer agent 使用,用于检查页面布局是否符合模板要求。

\paragraph{使用方法}

提取第 6 页到默认路径 \lstinline|/tmp/pdf_page_6.png|:

\begin{lstlisting}[style=blockstyle,language=bash]
python tools/extract_pdf_page.py examples/Xufan.pdf 6
\end{lstlisting}

指定输出路径:

\begin{lstlisting}[style=blockstyle,language=bash]
python tools/extract_pdf_page.py examples/Xufan.pdf 6 my_output.png
\end{lstlisting}

\paragraph{系统依赖}

\begin{lstlisting}[style=blockstyle,language=bash]
# Python 依赖
pip install pdf2image pillow

# 系统依赖(Linux/WSL)
sudo apt install poppler-utils

# 系统依赖(macOS)
brew install poppler
\end{lstlisting}

\subsection{生成可编辑 PPTX(布局保持)}

\subsubsection{用途与特点}

\lstinline|pdf_to_editable_pptx.py| 工具生成真正可编辑的 PPTX 文件,采用 \textbf{混合方法(背景图 + 文本框)},实现:
\begin{itemize}
  \item \textbf{完全相同的布局}:PDF 页面作为 300 DPI 高清背景图,确保视觉完全一致
  \item \textbf{完全可编辑的文字}:透明文本框精确覆盖原文字位置,保持样式
  \item \textbf{最佳的可用性}:既有专业的视觉效果,又能在 PowerPoint 中自由编辑
\end{itemize}

\paragraph{核心特性}

\begin{itemize}
  \item \textbf{样式完整保留}:字体名称、字号、颜色、粗体、斜体等
  \item \textbf{布局精确映射}:保留精确的文本位置(坐标映射)
  \item \textbf{页面比例精确匹配}:自动从 PDF 获取真实页面尺寸,确保 PPTX 与 PDF 页面比例完全一致
  \item \textbf{小文件体积}:40--50KB(相比嵌入图片方式小 100 倍)
  \item \textbf{完全可编辑}:文本可选、可修改、可重新设置样式
  \item \textbf{快速转换}:无需 OCR,直接提取 PDF 结构信息
\end{itemize}

\paragraph{适用场景}

\begin{itemize}
  \item 需要在 PowerPoint 中进一步编辑内容
  \item 需要保持原始样式(颜色、字体、大小)的可编辑性
  \item 文件大小有限制(邮件、云同步等)
  \item 需要提取和复用幻灯片中的文本内容
  \item 需要修改某些 PDF 内容但保留整体样式
\end{itemize}

\paragraph{限制与注意}

\begin{itemize}
  \item 不能保留 PDF 中的复杂矢量图形(但可提取图片)
  \item 某些特殊 PostScript 字体可能被替换为相似的可用字体
  \item 复杂多列布局的重排可能需要手动微调
  \item 超复杂的 LaTeX 数学公式可能无法完美还原
\end{itemize}

\paragraph{使用方法}

标准用法:

\begin{lstlisting}[style=blockstyle,language=bash]
python tools/pdf_to_editable_pptx.py input.pdf output.pptx
\end{lstlisting}

示例:

\begin{lstlisting}[style=blockstyle,language=bash]
# 转换为可编辑 PPTX(保留所有样式)
python tools/pdf_to_editable_pptx.py examples/Xufan.pdf my_presentation.pptx

# 转换后的 PPTX 可在 PowerPoint/LibreOffice 中编辑
# - 文字可选择和修改
# - 颜色、字体保持不变
# - 位置和大小也能准确还原
\end{lstlisting}

\paragraph{系统依赖}

\begin{lstlisting}[style=blockstyle,language=bash]
# Python 依赖
pip install pymupdf python-pptx pillow

# 系统依赖(仅 Linux/WSL 需要,Windows 不需要)
# 如果不使用 pdf_to_pptx.py,不需要安装 poppler
sudo apt install poppler-utils
\end{lstlisting}

\paragraph{工作原理}

该工具的转换流程:

\begin{enumerate}
  \item 使用 PyMuPDF 的 \lstinline|get_text("dict")| 提取 PDF 每个 span 的详细信息
  \item 提取内容:文本内容、字体名称、字号、RGB 颜色、粗体/斜体标志、精确坐标
  \item 坐标映射:将 PDF 坐标(左下角原点)转换为 PPTX 坐标(左上角原点)
  \item 创建文本框:根据提取的样式逐个创建 PPTX 文本框,应用对应的字体/颜色/大小
  \item 生成 PPTX:保存为可在 PowerPoint/LibreOffice 中编辑的 PPTX 文件
\end{enumerate}

\paragraph{工作流推荐}

\begin{enumerate}
  \item 在 LaTeX 中完成演示文稿的设计和编辑
  \item 编译得到最终 PDF
  \item 根据需求选择:
    \begin{itemize}
      \item 仅需 PowerPoint 查看 $\Rightarrow$ 使用 \lstinline|pdf_to_pptx.py|(快速,文件完整)
      \item 需要在 PowerPoint 中编辑 $\Rightarrow$ 使用 \lstinline|pdf_to_editable_pptx.py|(灵活,可编辑)
    \end{itemize}
  \item 如需验证 PDF 布局是否正确 $\Rightarrow$ 使用 \lstinline|extract_pdf_page.py| + pdf-layout-reviewer
\end{enumerate}