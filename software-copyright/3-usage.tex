%!TEX root = document.tex

\section{使用指南}
\label{sec:usage}

\Name 是一个功能完整的学术演示文稿 LaTeX Beamer 模板,专为 IQB Lab 日志会(Journal Club)汇报设计。通过提供预设的主题、布局工具包和丰富的内容模块,用户可以快速创建专业、美观的学术幻灯片演示。本章详细说明该模板的使用方法,涵盖基础配置、核心功能模块和实际应用示例。

\subsection{基础使用与最小示例}

\subsubsection{最小示例代码}

以下是创建一个完整演示文稿的最小示例。用户可以以此为起点进行扩展:

\begin{lstlisting}[style=blockstyle,language=TeX]
\documentclass[aspectratio=169,11pt]{beamer}

% 加载 IQB 主题和布局工具包
\usepackage{theme/beamerthemeiqb}
\usepackage{theme/iqb-layouts}

% 设置 header 图片路径
\renewcommand{\iqbheaderimage}{theme/images/header.png}

% 中文支持(使用 XeLaTeX 编译)
\usepackage{xeCJK}
\setCJKmainfont{SimSun}

% 文档元数据
\title{你的演示标题}
\subtitle{副标题(可选)}
\author{你的名字}
\institute{IQB Lab}
\date{\today}

\begin{document}

% 封面页(无 header/footer)
\begin{frame}[plain,noframenumbering]
  \titlepage
\end{frame}

% 内容页示例
\setsection{Background}
\begin{frame}{第一页内容}
  \begin{itemize}
    \item 要点 1
    \item 要点 2
    \item 要点 3
  \end{itemize}
\end{frame}

% 致谢页(无 header/footer)
\begin{frame}[plain]
  \centering
  {\Huge Thank You!}
\end{frame}

\end{document}
\end{lstlisting}

\subsubsection{编译方法}

使用 XeLaTeX 编译器(推荐,支持中文):

\begin{lstlisting}[style=blockstyle,language=bash]
xelatex -interaction=nonstopmode your-presentation.tex
\end{lstlisting}

若需要生成含交叉引用的完整 PDF,需运行两次编译:

\begin{lstlisting}[style=blockstyle,language=bash]
xelatex -interaction=nonstopmode your-presentation.tex
xelatex -interaction=nonstopmode your-presentation.tex
\end{lstlisting}

\subsubsection{文档类选项说明}

表 \ref{tab:documentclass-options} 列出了常用的文档类选项及其含义。

\begin{table}[!h]
\centering
\caption{Beamer 文档类常用选项}
\label{tab:documentclass-options}
\begin{tabular}{p{3.5cm}p{9.5cm}}
\toprule
\textbf{选项} & \textbf{说明} \\
\midrule
\lstinline|aspectratio=169| & 设置幻灯片宽高比为 16:9(推荐用于现代显示器) \\
\lstinline|aspectratio=43| & 设置幻灯片宽高比为 4:3(传统比例) \\
\lstinline|11pt| & 设置基础字体大小为 11pt \\
\lstinline|12pt| & 设置基础字体大小为 12pt(默认) \\
\bottomrule
\end{tabular}
\end{table}

\subsubsection{中文支持配置}

模板使用 XeLaTeX + xeCJK 实现中文支持。在导言区添加以下代码:

\begin{lstlisting}[style=blockstyle,language=TeX]
\usepackage{xeCJK}
\setCJKmainfont{SimSun}    % Windows 系统
% 或
\setCJKmainfont{Noto Sans CJK SC}  % Linux 系统
\end{lstlisting}

若在 macOS 系统上使用,可选择以下字体:

\begin{lstlisting}[style=blockstyle,language=TeX]
\setCJKmainfont{STHeiti}   % macOS 默认字体
\end{lstlisting}

\subsection{页面结构控制}

IQB-JC Beamer 模板具有三种页面类型:封面页、标准内容页和特殊页面(如致谢页)。

\subsubsection{页面类型详解}

表 \ref{tab:frame-types} 总结了不同页面类型的特性及适用场景。

\begin{table}[!h]
\centering
\caption{IQB-JC 模板中的页面类型}
\label{tab:frame-types}
\begin{tabular}{p{2.5cm}p{3cm}p{3cm}p{4cm}}
\toprule
\textbf{页面类型} & \textbf{Header 显示} & \textbf{Footer 显示} & \textbf{应用场景} \\
\midrule
标准内容页 & 是 & 是 & 演示主要内容 \\
Plain 页面 & 否 & 否 & 封面、致谢、转场页 \\
\bottomrule
\end{tabular}
\end{table}

\subsubsection{标准内容页}

标准内容页同时显示 header 横幅和 footer 三段式设计。使用如下方式创建:

\begin{lstlisting}[style=blockstyle,language=TeX]
\setsection{Methods}  % 设置 footer 中间的 section 标识
\begin{frame}{页面标题}
  页面内容
\end{frame}
\end{lstlisting}

其中 \lstinline|\setsection{}| 命令设置 footer 中间部分显示的章节名称。该设置对后续所有页面有效,直到被重新设置。

\subsubsection{Plain 页面(无 Header/Footer)}

Plain 页面隐藏 header 和 footer,适用于封面、致谢等特殊页面。创建方式如下:

\begin{lstlisting}[style=blockstyle,language=TeX]
\begin{frame}[plain,noframenumbering]
  \titlepage
\end{frame}
\end{lstlisting}

其中:

\begin{itemize}
\item \lstinline|[plain]| - 隐藏 header、footer 和页码
\item \lstinline|[noframenumbering]| - 不计入总页数统计(适用于封面、致谢等)
\end{itemize}

\subsubsection{Footer Section 设置}

Footer 底部的三段式设计包含:左侧固定文字 "IQB Lab"、中间 section 标识、右侧页码显示。使用 \lstinline|\setsection{}| 命令管理中间部分:

\begin{lstlisting}[style=blockstyle,language=TeX]
\setsection{Background}
\begin{frame}{背景介绍}
  % footer 中间显示 "Background"
\end{frame}

\setsection{Methods}
\begin{frame}{研究方法}
  % footer 中间显示 "Methods"
\end{frame}
\end{lstlisting}

建议在每个主要章节的第一页调用 \lstinline|\setsection{}| 命令以保持 footer 信息的准确性。

\subsection{主题自定义}

\subsubsection{修改主题颜色}

IQB-JC 模板的主题颜色定义在 \lstinline|theme/beamerthemeiqb.sty| 文件中。主要颜色包括:

\begin{table}[!h]
\centering
\caption{IQB-JC 主题颜色定义}
\label{tab:theme-colors}
\begin{tabular}{p{3cm}p{4cm}p{4cm}}
\toprule
\textbf{颜色名称} & \textbf{RGB 值} & \textbf{十六进制} \\
\midrule
\lstinline|iqbblue| & 0, 51, 102 & \#003366 \\
\lstinline|iqbgray| & 85, 85, 85 & \#555555 \\
\lstinline|iqblightgray| & 240, 240, 240 & \#F0F0F0 \\
\bottomrule
\end{tabular}
\end{table}

若要修改主题颜色,可在导言区添加颜色重定义:

\begin{lstlisting}[style=blockstyle,language=TeX]
\definecolor{iqbblue}{RGB}{0, 100, 200}  % 自定义蓝色
\definecolor{iqbgray}{RGB}{100, 100, 100}  % 自定义灰色
\end{lstlisting}

\subsubsection{替换 Header 图片}

Header 横幅图片可通过重新定义 \lstinline|\iqbheaderimage| 命令更改:

\begin{lstlisting}[style=blockstyle,language=TeX]
\renewcommand{\iqbheaderimage}{path/to/your/header.png}
\end{lstlisting}

建议的 header 图片规格:

\begin{itemize}
\item 分辨率:1999×204 像素(保持原始 1999:204 比例)
\item 宽高比:约 9.8:1
\item 格式:PNG 或 PDF(支持透明度)
\end{itemize}

\subsubsection{调整字体大小}

模板统一采用以下字体层级(单位为 pt):

\begin{table}[!h]
\centering
\caption{IQB-JC 字体大小体系}
\label{tab:font-sizes}
\begin{tabular}{p{4cm}p{2cm}p{6cm}}
\toprule
\textbf{文本类型} & \textbf{字号} & \textbf{说明} \\
\midrule
封面标题(\lstinline|\title|) & 14.4pt & 使用 \lstinline|\Large| + \lstinline|\bfseries| \\
页面标题(\lstinline|\frametitle|) & 12pt & 使用 \lstinline|\large| + \lstinline|\bfseries| \\
正文(默认) & 8pt & 使用 \lstinline|\scriptsize| \\
脚注/Footer & 5pt & 使用 \lstinline|\tiny| \\
\bottomrule
\end{tabular}
\end{table}

若要修改全局字体大小,在 \lstinline|theme/beamerthemeiqb.sty| 中查找 \lstinline|\setbeamerfont| 命令进行调整。

\subsection{多列布局模块}

IQB-JC 模板通过 \lstinline|iqb-layouts.sty| 包提供了多种预设布局工具,满足不同的内容排版需求。

\subsubsection{双列布局(50-50)}

命令格式:

\begin{lstlisting}[style=blockstyle,language=TeX]
\iqblayouttwo{左列内容}{右列内容}
\end{lstlisting}

该命令将页面平均分为两列,各占 48\% 宽度,中间留 4\% 间距。适用于并行展示两个相关内容的场景。使用示例:

\begin{lstlisting}[style=blockstyle,language=TeX]
\begin{frame}{双列对比}
  \iqblayouttwo{
    \textbf{方法 A}
    \begin{itemize}
      \item 优点 1
      \item 优点 2
    \end{itemize}
  }{
    \textbf{方法 B}
    \begin{itemize}
      \item 优点 1
      \item 优点 2
    \end{itemize}
  }
\end{frame}
\end{lstlisting}

\subsubsection{不对称双列布局}

\textbf{1/3 + 2/3 布局}

命令格式:

\begin{lstlisting}[style=blockstyle,language=TeX]
\iqblayoutonethird{左列内容(占 31\%)}{右列内容(占 65\%)}
\end{lstlisting}

该布局将页面分为 31\% 和 65\% 两列,适合在右侧放置大图,左侧放置说明文字。使用示例:

\begin{lstlisting}[style=blockstyle,language=TeX]
\begin{frame}{竖版图片布局}
  \iqblayoutonethird{
    \textbf{说明文字}

    该图展示了关键结果
  }{
    \includegraphics[height=0.6\textheight,keepaspectratio]{image.png}
  }
\end{frame}
\end{lstlisting}

\textbf{2/3 + 1/3 布局}

命令格式:

\begin{lstlisting}[style=blockstyle,language=TeX]
\iqblayouttwothirds{左列内容(占 65\%)}{右列内容(占 31\%)}
\end{lstlisting}

该布局与上述布局对称,适合在左侧放置大图,右侧放置说明文字。

\subsubsection{三列布局}

命令格式:

\begin{lstlisting}[style=blockstyle,language=TeX]
\iqblayoutthree{左列内容}{中列内容}{右列内容}
\end{lstlisting}

将页面平均分为三列,各占 31\% 宽度。适合展示三个平行的内容或对比三种方法。使用示例:

\begin{lstlisting}[style=blockstyle,language=TeX]
\begin{frame}{三方法对比}
  \iqblayoutthree{
    \textbf{方法 A}

    优点与缺点分析
  }{
    \textbf{方法 B}

    优点与缺点分析
  }{
    \textbf{方法 C}

    优点与缺点分析
  }
\end{frame}
\end{lstlisting}

\subsection{网格与图表布局}

\subsubsection{2×2 网格布局}

命令格式:

\begin{lstlisting}[style=blockstyle,language=TeX]
\iqbgridtwobytwo{左上}{右上}{左下}{右下}
\end{lstlisting}

该布局适合展示四个相关的图片或内容块,自动排列成 2×2 网格。使用示例:

\begin{lstlisting}[style=blockstyle,language=TeX]
\begin{frame}{四个关键结果}
  \iqbgridtwobytwo{
    \includegraphics[height=0.35\textheight]{fig1.png}
  }{
    \includegraphics[height=0.35\textheight]{fig2.png}
  }{
    \includegraphics[height=0.35\textheight]{fig3.png}
  }{
    \includegraphics[height=0.35\textheight]{fig4.png}
  }
\end{frame}
\end{lstlisting}

\subsubsection{3×2 网格布局}

命令格式:

\begin{lstlisting}[style=blockstyle,language=TeX]
\iqbgridthreebytwo{左上}{中上}{右上}{左下}{中下}{右下}
\end{lstlisting}

该布局展示六个内容块,排列为 3 列 2 行。各列占 31\% 宽度,垂直间距为 0.5em。

\subsection{图文混排模块}

\subsubsection{图片左、文字右}

命令格式:

\begin{lstlisting}[style=blockstyle,language=TeX]
\iqbimagetext[图片选项]{图片路径}{右侧文字内容}
\end{lstlisting}

默认图片选项为 \lstinline|width=0.45\textwidth|。使用示例:

\begin{lstlisting}[style=blockstyle,language=TeX]
\begin{frame}{实验结果}
  \iqbimagetext[width=0.4\textwidth]{experiment.png}{
    \textbf{关键发现:}
    \begin{itemize}
      \item 发现 1
      \item 发现 2
      \item 发现 3
    \end{itemize}
  }
\end{frame}
\end{lstlisting}

\subsubsection{文字左、图片右}

命令格式:

\begin{lstlisting}[style=blockstyle,language=TeX]
\iqbtextimage[图片选项]{左侧文字内容}{图片路径}
\end{lstlisting}

这是上述布局的镜像版本,文字显示在左列,图片显示在右列。使用示例:

\begin{lstlisting}[style=blockstyle,language=TeX]
\begin{frame}{方法流程}
  \iqbtextimage[height=0.55\textheight]{
    \textbf{研究步骤:}
    \begin{enumerate}
      \item 数据收集
      \item 预处理
      \item 分析
    \end{enumerate}
  }{flowchart.png}
\end{frame}
\end{lstlisting}

\subsection{图表模块(自动编号)}

IQB-JC 模板提供了带自动编号和统一格式的图表模块。所有图表自动生成"图X:"格式的编号,caption 左对齐显示。

\subsubsection{单图模块}

命令格式:

\begin{lstlisting}[style=blockstyle,language=TeX]
\iqbfig[height=0.5\textheight]{图片路径}{图说文字}
\end{lstlisting}

参数说明见表 \ref{tab:iqbfig-params}。

\begin{table}[!h]
\centering
\caption{iqbfig 命令参数}
\label{tab:iqbfig-params}
\begin{tabular}{p{2.5cm}p{2.5cm}p{8cm}}
\toprule
\textbf{参数} & \textbf{默认值} & \textbf{说明} \\
\midrule
\lstinline|[选项]| & \lstinline|height=0.5\textheight| & 图片尺寸,支持 \lstinline|height| 或 \lstinline|width| \\
图片路径 & 必需 & 相对或绝对路径,支持 PNG/PDF \\
图说文字 & 必需 & 图片下方的说明文字,自动编号 \\
\bottomrule
\end{tabular}
\end{table}

使用示例:

\begin{lstlisting}[style=blockstyle,language=TeX]
\begin{frame}{单图展示}
  \iqbfig[height=0.55\textheight]{results.png}{
    结果展示图:实验数据对比分析
  }
\end{frame}
\end{lstlisting}

\subsubsection{双图模块}

命令格式:

\begin{lstlisting}[style=blockstyle,language=TeX]
\iqbtwofig[height=0.45\textheight]{img1}{caption1}{img2}{caption2}
\end{lstlisting}

该模块将两张图片并排显示,各占 48\% 宽度。使用示例:

\begin{lstlisting}[style=blockstyle,language=TeX]
\begin{frame}{双图对比}
  \iqbtwofig[height=0.5\textheight]{
    method_a.png
  }{
    方法 A 的结果展示
  }{
    method_b.png
  }{
    方法 B 的结果展示
  }
\end{frame}
\end{lstlisting}

\subsubsection{三图模块}

命令格式:

\begin{lstlisting}[style=blockstyle,language=TeX]
\iqbthreefig[height=0.35\textheight]{img1}{cap1}{img2}{cap2}{img3}{cap3}
\end{lstlisting}

三张图片均匀分布在一行,各占 31\% 宽度。该布局适合展示三个相关的分析结果。

\subsubsection{四图网格模块}

命令格式:

\begin{lstlisting}[style=blockstyle,language=TeX]
\iqbfourfig[height=0.35\textheight]{
  img1}{cap1}{img2}{cap2}{img3}{cap3}{img4}{cap4}
\end{lstlisting}

四张图片排列为 2×2 网格。该布局提供清晰的图表组织方式,适合展示多阶段或多方法的结果对比。

\subsection{作者信息模块}

IQB-JC 模板提供了专门的作者信息展示模块,适用于文献汇报时介绍论文作者的信息。

\subsubsection{双作者模块(无照片)}

命令格式:

\begin{lstlisting}[style=blockstyle,language=TeX]
\iqbauthorstwo{
  通讯作者姓名}{通讯作者单位}{通讯作者网址}{通讯作者研究方向}{
  第一作者姓名}{第一作者单位}{第一作者网址}{第一作者研究方向
}
\end{lstlisting}

该模块以 50-50 双列布局展示通讯作者和第一作者的信息。使用示例:

\begin{lstlisting}[style=blockstyle,language=TeX]
\begin{frame}{作者信息}
  \iqbauthorstwo{
    Prof. John Smith
  }{
    Department of Biology\\
    University of Example
  }{
    https://example.edu/smith
  }{
    Molecular dynamics, Protein folding
  }{
    Dr. Jane Doe
  }{
    Institute of Computational Science\\
    Example University
  }{
    https://example.edu/doe
  }{
    MD simulations, Free energy calculations
  }
\end{frame}
\end{lstlisting}

\subsubsection{双作者模块(带照片)}

命令格式:

\begin{lstlisting}[style=blockstyle,language=TeX]
\setauthorfirstfield{第一作者研究领域}
\iqbauthorstwophoto{
  通讯作者照片路径}{通讯作者姓名}{...}{...}{通讯作者研究方向}{
  第一作者照片路径}{第一作者姓名}{...}{...}
}
\end{lstlisting}

使用 \lstinline|\setauthorfirstfield{}| 命令单独设置第一作者的专业领域。使用示例:

\begin{lstlisting}[style=blockstyle,language=TeX]
\begin{frame}{论文作者}
  \setauthorfirstfield{计算生物物理、膜蛋白模拟、自由能计算}
  \iqbauthorstwophoto{
    author1.jpg
  }{
    Prof. Alice Brown
  }{
    College of Science\\
    Tech University
  }{
    https://example.org/alice
  }{
    Protein structure prediction, Force field development
  }{
    author2.jpg
  }{
    Dr. Bob White
  }{
    Center for Molecular Dynamics\\
    Research Institute
  }{
    https://example.org/bob
  }
\end{frame}
\end{lstlisting}

\subsubsection{单作者模块}

命令格式:

\begin{lstlisting}[style=blockstyle,language=TeX]
\iqbauthorone{作者姓名}{作者单位}{作者网址}{研究方向}
\end{lstlisting}

该模块适用于单作者论文或特邀评论文章。信息居中显示。

\subsection{内容模块}

\subsubsection{关键要点模块}

命令格式:

\begin{lstlisting}[style=blockstyle,language=TeX]
\iqbkeypoints{
  \begin{itemize}
    \item 要点 1
    \item 要点 2
  \end{itemize}
}
\end{lstlisting}

该模块将内容包装在突出显示的 block 环境中,背景色为 IQB 浅灰色,标题为"关键要点"。适合提炼每页的核心内容。使用示例:

\begin{lstlisting}[style=blockstyle,language=TeX]
\begin{frame}{研究成果}
  \iqbkeypoints{
    \begin{itemize}
      \item 发现了新的分子动力学计算方法
      \item 提高了计算精度 50\%
      \item 减少了计算时间 30\%
    \end{itemize}
  }
\end{frame}
\end{lstlisting}

\subsubsection{核心问题模块}

命令格式:

\begin{lstlisting}[style=blockstyle,language=TeX]
\iqbquestion{该项研究的核心问题是什么?}
\end{lstlisting}

该模块突出显示研究的核心科学问题,文字居中排列。适合在研究背景部分强调关键问题。

\subsubsection{结论模块}

命令格式:

\begin{lstlisting}[style=blockstyle,language=TeX]
\iqbconclusion{
  通过 XX 方法,我们发现了 YY 现象,为 ZZ 领域的发展奠定了基础。
}
\end{lstlisting}

该模块用于总结研究的主要结论,适合放在讨论或总结页面。文字自动居中显示。

\subsection{高级功能模块}

\subsubsection{公式与解释布局}

命令格式:

\begin{lstlisting}[style=blockstyle,language=TeX]
\iqbformulaexplain{公式内容}{公式右侧解释文字}
\end{lstlisting}

该模块采用 1/3 + 2/3 布局,左侧放置公式(居中),右侧放置解释说明。使用示例:

\begin{lstlisting}[style=blockstyle,language=TeX]
\begin{frame}{关键公式推导}
  \iqbformulaexplain{
    $\Delta G = \Delta H - T\Delta S$
  }{
    \textbf{含义:}
    \begin{itemize}
      \item $\Delta G$ 是自由能变化
      \item $\Delta H$ 是焓变
      \item $T\Delta S$ 是熵项贡献
    \end{itemize}
  }
\end{frame}
\end{lstlisting}

\subsubsection{时间线/流程图模块}

命令格式:

\begin{lstlisting}[style=blockstyle,language=TeX]
\iqbtimeline{步骤1标题}{步骤1内容}{步骤2标题}{步骤2内容}{
  步骤3标题}{步骤3内容}
\end{lstlisting}

该模块创建一个三步的流程图,使用 TikZ 绘制连接的方框。每个方框包含标题和内容,步骤间有箭头连接。使用示例:

\begin{lstlisting}[style=blockstyle,language=TeX]
\begin{frame}{研究方法流程}
  \iqbtimeline{
    数据采集}{收集 1000 个样本}{
    数据处理}{质量控制与预处理}{
    统计分析}{方差分析与回归
  }
\end{frame}
\end{lstlisting}

\subsubsection{双列对比模块}

命令格式:

\begin{lstlisting}[style=blockstyle,language=TeX]
\iqbcomparetwo{标题 A}{内容 A}{标题 B}{内容 B}
\end{lstlisting}

该模块在双列布局中展示两个方法或方案的对比。每列包含标题和内容,自动对齐。

\subsubsection{三列对比模块}

命令格式:

\begin{lstlisting}[style=blockstyle,language=TeX]
\iqbcomparethree{标题A}{内容A}{标题B}{内容B}{标题C}{内容C}
\end{lstlisting}

该模块采用三列布局展示三个方法或方案的对比分析。

\subsubsection{精确定位(高级)}

对于需要自定义精确布局的情况,模板提供了底层的定位工具。

\textbf{文本精确定位}

命令格式:

\begin{lstlisting}[style=blockstyle,language=TeX]
\iqbplace{x}{y}{宽度}{内容}
\end{lstlisting}

其中坐标 (0,0) 为页面左上角,x 和 y 单位为 cm。使用示例:

\begin{lstlisting}[style=blockstyle,language=TeX]
\begin{frame}{}
  \iqbplace{2}{3}{10cm}{
    自定义位置的文本内容
  }
\end{frame}
\end{lstlisting}

\textbf{图片精确定位}

命令格式:

\begin{lstlisting}[style=blockstyle,language=TeX]
\iqbplaceimage{x}{y}{宽度}{高度}{图片路径}
\end{lstlisting}

\subsection{实际应用指南}

\subsubsection{完整演示文稿示例}

基于以上模块,以下展示了一个包含多种布局的完整页面示例:

\begin{lstlisting}[style=blockstyle,language=TeX]
\documentclass[aspectratio=169,11pt]{beamer}
\usepackage{theme/beamerthemeiqb}
\usepackage{theme/iqb-layouts}
\usepackage{xeCJK}
\setCJKmainfont{SimSun}

\title{膜孔自由能与稳定性研究}
\author{研究人员}
\institute{IQB Lab}

\begin{document}

% 第 1 页:封面
\begin{frame}[plain,noframenumbering]
  \titlepage
\end{frame}

% 第 2 页:背景介绍
\setsection{Background}
\begin{frame}{研究背景}
  \iqbquestion{膜孔形成的分子机制是什么?}
\end{frame}

% 第 3 页:双列对比
\begin{frame}{两种研究方法对比}
  \iqbcomparetwo{
    方法 A:全原子 MD}{优点与局限}{
    方法 B:粗粒化 MD}{优点与局限
  }
\end{frame}

% 第 4 页:图表展示
\begin{frame}{主要结果}
  \iqbtwofig[height=0.5\textheight]{
    result_a.png}{结果 A 展示}{
    result_b.png}{结果 B 展示
  }
\end{frame}

% 第 5 页:作者信息
\setsection{Author}
\begin{frame}{论文作者}
  \setauthorfirstfield{计算生物物理、自由能计算}
  \iqbauthorstwophoto{
    author1.jpg}{Prof. Smith}{Dept of Bio\\University A}{
    https://example.edu}{Protein folding}{
    author2.jpg}{Dr. Doe}{Institute B}{
    https://example.edu
  }
\end{frame}

% 最后一页:致谢
\begin{frame}[plain]
  \centering
  {\Huge Thank You!}
\end{frame}

\end{document}
\end{lstlisting}

\subsubsection{布局选择建议}

表 \ref{tab:layout-selection} 提供了在不同内容类型下的布局选择建议。

\begin{table}[!h]
\centering
\caption{不同内容类型的布局选择}
\label{tab:layout-selection}
\begin{tabular}{p{3cm}p{3cm}p{6cm}}
\toprule
\textbf{内容类型} & \textbf{推荐布局} & \textbf{说明} \\
\midrule
并行对比 & 双列 50-50 & 两个等权内容 \\
图文结合 & 1/3+2/3 & 竖版图片 + 说明 \\
多个独立内容 & 三列或网格 & 3 个或 4 个内容 \\
方法流程 & 时间线 & 流程步骤展示 \\
关键信息 & 关键要点 & 突出核心发现 \\
\bottomrule
\end{tabular}
\end{table}

\subsubsection{常见问题与解决方案}

\textbf{Q1:页面内容溢出}

若页面内容过多导致溢出,建议:
\begin{itemize}
\item 减少列表项数量,提炼关键要点
\item 调整图片高度参数
\item 使用更紧凑的布局(如三列代替双列)
\item 将内容拆分为两页
\end{itemize}

\textbf{Q2:图片显示不完整}

检查以下几点:
\begin{itemize}
\item 图片路径是否正确(相对/绝对路径)
\item 图片格式是否支持(建议 PNG/PDF)
\item 高度/宽度参数是否过小
\item 尝试使用 \lstinline|keepaspectratio| 保持宽高比
\end{itemize}

\textbf{Q3:中文显示乱码}

确保:
\begin{itemize}
\item 使用 XeLaTeX 编译器
\item 字体设置正确(Windows 用 SimSun,Linux 用 Noto Sans CJK SC)
\item 文件编码为 UTF-8
\end{itemize}

\textbf{Q4:Footer 页码显示错误}

使用 \lstinline|[noframenumbering]| 选项排除不计页数的页面,确保页码准确。

\subsection{Phase 1: 高级快捷命令(Advanced Shortcuts)}

\subsubsection{概述}

为了进一步简化用户编写,模板在 \lstinline|iqb-layouts.sty| 中提供了一系列高级宏命令。这些命令将常见的Frame模式标准化和模板化,显著减少代码重复,提升编写效率。所有新命令与现有代码完全兼容,用户可以渐进式采用。

\subsubsection{Frame 快捷命令}

\textbf{标准 Frame}

命令格式:

\begin{lstlisting}[style=blockstyle,language=TeX]
\iqbframe{标题}{内容}
\end{lstlisting}

用于快速创建标准单栏 frame,相比原始写法减少 67\% 的代码。

\textbf{文字+图片 Frame(左文右图)}

命令格式:

\begin{lstlisting}[style=blockstyle,language=TeX]
\iqbframetextfig[高度]{标题}{文字内容}{图片路径}
\end{lstlisting}

自动应用 1/3 左 + 2/3 右的布局。可选参数 \lstinline|[高度]| 默认为 \lstinline|0.55\textheight|。

使用示例:

\begin{lstlisting}[style=blockstyle,language=TeX]
\iqbframetextfig{膜孔形成的重要性}{
  \textbf{应用领域}:
  \begin{itemize}
    \item 抗菌肽设计
    \item 药物递送系统
    \item 细胞通透性调控
  \end{itemize}
}{images/membrane-pore.png}
\end{lstlisting}

\textbf{图片+文字 Frame(左图右文)}

命令格式:

\begin{lstlisting}[style=blockstyle,language=TeX]
\iqbframefigttext[高度]{标题}{图片路径}{文字内容}
\end{lstlisting}

图片占 1/3 左,文字占 2/3 右。

\textbf{文字+双图 Frame}

命令格式:

\begin{lstlisting}[style=blockstyle,language=TeX]
\iqbframetexttwofig[高度]{标题}{文字}{img1}{cap1}{img2}{cap2}
\end{lstlisting}

文字占 1/3 左,两张并排的图片占 2/3 右。

\subsubsection{Section 管理快捷}

\textbf{自动 Section 分隔页}

命令格式:

\begin{lstlisting}[style=blockstyle,language=TeX]
\iqbsectionframe{SectionName}{中文标题}
\end{lstlisting}

自动生成 section 分隔页,并更新 footer 中的 section 进度标识,无需手动调用 \lstinline|\setsection|。

使用示例:

\begin{lstlisting}[style=blockstyle,language=TeX]
\iqbsectionframe{Methods}{方法}
% 后续该 section 下的 frame 会自动显示 "Methods" 在 footer 中

\iqbsectionframe{Results}{结果}
% section 切换,footer 自动更新
\end{lstlisting}

\subsubsection{公式+解释专用布局}

\textbf{公式 Frame}

命令格式:

\begin{lstlisting}[style=blockstyle,language=TeX]
\iqbformulaframe{标题}{公式内容}{右侧图文}
\end{lstlisting}

左侧 1/3 放置公式推导,右侧 2/3 放置图片或补充说明。代码量减少 70\%。

\textbf{公式块辅助命令}

命令格式:

\begin{lstlisting}[style=blockstyle,language=TeX]
\iqbformblock{标题}{公式}{解释}
\end{lstlisting}

自动管理公式块的间距和字体。使用示例:

\begin{lstlisting}[style=blockstyle,language=TeX]
\iqbformulaframe{Full-Path CV 原理}{
  \iqbformblock{成核部分 $\text{CV}_{\text{cyl}}$}{
    $$\text{CV}_{\text{cyl}} = 1 - d/\text{CV}_{\text{eq}}$$
  }{$d$: 圆柱内脂质尾部原子数}

  \iqbformblock{扩展部分 $\text{CV}_{\text{radius}}$}{
    $$\text{CV}_{\text{radius}} = r_{\text{min}}/r_{\text{unit}}$$
  }{$r_{\text{min}}$: 孔中心到最近脂质距离}
}{
  \iqbfig[height=0.55\textheight]{fig1.png}{集体变量示意图}
}
\end{lstlisting}

\subsubsection{对比展示快捷}

\textbf{三列对比}

命令格式:

\begin{lstlisting}[style=blockstyle,language=TeX]
\iqbthreecolcompare{title1}{img1}{items1}{title2}{img2}{items2}{title3}{img3}{content3}
\end{lstlisting}

三列均匀分布,每列可包含标题、图片、列表。代码量减少 83\%。

使用示例:

\begin{lstlisting}[style=blockstyle,language=TeX]
\begin{frame}{创新方法对比}
  \iqbthreecolcompare
    {Full-Path CV}{fig1u.png}{\item 成核+扩展统一 \item 无滞后}
    {Rapid CV}{fig1d.png}{\item 无限孔模拟 \item 效率高 10×}
    {开源实现}{plumed.png}{PLUMED 库\\GROMACS/LAMMPS}
\end{frame}
\end{lstlisting}

\textbf{双列对比}

命令格式:

\begin{lstlisting}[style=blockstyle,language=TeX]
\iqbtwocolcompare{title1}{img1}{items1}{title2}{img2}{items2}
\end{lstlisting}

\subsubsection{结果展示快捷}

\textbf{结果展示 Frame}

命令格式:

\begin{lstlisting}[style=blockstyle,language=TeX]
\iqbresultframe{标题}{左侧文字}{右侧图片}
\end{lstlisting}

左侧 1/3 放置关键发现和数据,右侧 2/3 放置结果图片。代码量减少 80\%。

使用示例:

\begin{lstlisting}[style=blockstyle,language=TeX]
\iqbresultframe{孔闭合过程分析}{
  \textbf{四阶段闭合}:
  \begin{enumerate}
    \item 平衡孔:初始稳定孔,连续水柱贯穿膜
    \item 半径缩小:孔边缘脂质重排,结构保持
    \item 水线程:闭合最后瞬间,仅剩连续水线
    \item 膜变薄:孔完全闭合,局部膜缺陷
  \end{enumerate}

  \textbf{关键发现}:脂质尾部密度与孔寿命强相关 (R²=0.82)
}{images/closure.png}
\end{lstlisting}

\textbf{对比结果 Frame}

命令格式:

\begin{lstlisting}[style=blockstyle,language=TeX]
\iqbcomparisonframe{标题}{左图}{左标题}{右图}{右标题}
\end{lstlisting}

并排展示两个对比结果,自动编号图片。代码量减少 90\%。

\subsubsection{块级快捷命令}

\textbf{快速创建 Block}

命令格式:

\begin{lstlisting}[style=blockstyle,language=TeX]
\iqbblock{标题}{内容}
\end{lstlisting}

替代冗长的 \lstinline|\begin{block}...\end{block}| 写法。

\textbf{快速创建列表}

命令格式:

\begin{lstlisting}[style=blockstyle,language=TeX]
\iqbitemize{\item 项1 \item 项2 \item 项3}

\iqbenumerate{\item 项1 \item 项2 \item 项3}
\end{lstlisting}

\subsubsection{VSCode Snippets 集成}

配置文件位置:\lstinline|.vscode/latex.code-snippets|

在 VSCode 中使用快捷方式,输入缩写后按 Tab 键自动展开完整命令模板:

\begin{table}[!h]
\centering
\caption{VSCode Snippets 快捷方式列表}
\label{tab:vscode-snippets}
\begin{tabular}{p{2cm}p{3cm}p{6cm}}
\toprule
\textbf{快捷方式} & \textbf{对应命令} & \textbf{功能说明} \\
\midrule
\lstinline|iqbf| & \lstinline|\iqbframe| & 快速创建标准 frame \\
\lstinline|iqbtf| & \lstinline|\iqbframetextfig| & 文字+图片 frame \\
\lstinline|iqbfit| & \lstinline|\iqbframefigttext| & 图片+文字 frame \\
\lstinline|iqbff| & \lstinline|\iqbformulaframe| & 公式+解释 frame \\
\lstinline|iqbsec| & \lstinline|\iqbsectionframe| & Section 分隔页 \\
\lstinline|iqb3c| & \lstinline|\iqbthreecolcompare| & 三列对比 \\
\lstinline|iqb2c| & \lstinline|\iqbtwocolcompare| & 双列对比 \\
\lstinline|iqbrf| & \lstinline|\iqbresultframe| & 结果展示 \\
\lstinline|iqbcf| & \lstinline|\iqbcomparisonframe| & 对比结果 \\
\lstinline|iqbitem| & \lstinline|\iqbitemize| & 列表 \\
\lstinline|iqbenum| & \lstinline|\iqbenumerate| & 编号列表 \\
\lstinline|iqbfb| & \lstinline|\iqbformblock| & 公式块 \\
\bottomrule
\end{tabular}
\end{table}

使用示例:在 VSCode 中输入 \lstinline|iqbtf| 并按 Tab,自动展开为:

\begin{lstlisting}[style=blockstyle,language=TeX]
\iqbframetextfig{标题}{
  文字内容...
}{images/xxx.png}
\end{lstlisting}

\subsubsection{使用建议与最佳实践}

\begin{itemize}
\item \textbf{优先使用快捷命令}:在日常编写中优先采用快捷命令,保持代码简洁明了。
\item \textbf{灵活使用可选参数}:大多数命令支持可选参数微调,如 \lstinline|[0.6\textheight]| 自定义图片高度。
\item \textbf{完全兼容旧代码}:新命令与现有代码完全兼容,现有幻灯片无需修改。
\item \textbf{IDE 集成}:配合 VSCode 和 Snippets 获得最佳编写体验。
\item \textbf{图注管理}:每个图片都应有图注,可以是详细描述或文字解读。当图片非常高时,可以在旁边文字栏中编写解释。
\end{itemize}

\subsubsection{代码改进效果统计}

表 \ref{tab:code-improvement} 展示了使用快捷命令后的代码行数改进。平均而言,使用这些快捷命令可将代码量减少约 79\%。

\begin{table}[!h]
\centering
\caption{快捷命令代码改进统计}
\label{tab:code-improvement}
\begin{tabular}{lcccr}
\toprule
\textbf{模板类型} & \textbf{改进前} & \textbf{改进后} & \textbf{减少} & \textbf{对应命令} \\
\midrule
标准 Frame & 3 行 & 1 行 & 67\% & \lstinline|\iqbframe| \\
文字+图片 & 6 行 & 1 行 & 83\% & \lstinline|\iqbframetextfig| \\
公式+解释 & 40 行 & 12 行 & 70\% & \lstinline|\iqbformulaframe| \\
三列对比 & 48 行 & 8 行 & 83\% & \lstinline|\iqbthreecolcompare| \\
结果展示 & 15 行 & 3 行 & 80\% & \lstinline|\iqbresultframe| \\
对比结果 & 10 行 & 1 行 & 90\% & \lstinline|\iqbcomparisonframe| \\
\midrule
\textbf{平均} & - & - & \textbf{79\%} & - \\
\bottomrule
\end{tabular}
\end{table}

对于一个典型的 14 页 Journal Club 演示,使用快捷命令可以将源代码从约 270 行减少至 80 行,显著提升维护效率和可读性。

\subsubsection{三线表快捷命令(Table Utilities)}

\textbf{概述}

Beamer 演示中常需要展示对比数据。本模板提供了三个预格式化的三线表命令,自动应用 \lstinline|booktabs| 风格的专业表格格式(顶部线、中间分割线、底部线),符合学术规范。

\textbf{标准三线表:\lstinline|\iqbthreelinetable|}

命令格式:

\begin{lstlisting}[style=blockstyle,language=TeX]
\iqbthreelinetable{表格标题}{表头行}{表体行}
\end{lstlisting}

使用示例:

\begin{lstlisting}[style=blockstyle,language=TeX]
\iqbthreelinetable{孔寿命/ns}{
  力场 & DMPC & DPPC & POPC & DOPC \\
}{
  C36 & 122 & 94 & 34 & 15 \\
  Slipids & 110 & 32 & 27 & 18 \\
}
\end{lstlisting}

\textbf{两列对比表:\lstinline|\iqbtwocoltable|}

用于展示条件对比数据(如离子效应、有/无处理组对比):

\begin{lstlisting}[style=blockstyle,language=TeX]
\iqbtwocoltable{脂质类型}{$\Delta\gamma$ 变化}{
  POPG/POPS & ↑30-50\% \\
  POPC/POPE & ≈0 \\
}
\end{lstlisting}

\textbf{图注样式复用:\lstinline|\iqbcaptiontext|}

\textbf{问题}:原文档中 7 处使用相同的图注格式 \lstinline|{\footnotesize\raggedright\setlength{\rightskip}{0pt plus 1cm} ... }|,代码重复。

\textbf{解决方案}:使用统一的 \lstinline|\iqbcaptiontext| 命令:

\begin{lstlisting}[style=blockstyle,language=TeX]
\iqbcaptiontext{\textbf{图1}:创新的集体变量设计说明文字...}
\end{lstlisting}

自动应用:
\begin{itemize}
\item \lstinline|\footnotesize|:合适的字体大小
\item \lstinline|\raggedright|:左对齐
\item \lstinline|\setlength{\rightskip}{0pt plus 1cm}|:灵活右边距,自动换行
\end{itemize}

\textbf{中文标题元数据配置:\lstinline|\papertitlechn|、\lstinline|\papertitlechnsub|}

在文档开头配置中文标题,这些标题会自动在全文使用:

\begin{lstlisting}[style=blockstyle,language=TeX]
% 在 \begin{document} 前定义
\papertitlechn{破解膜孔之谜:双CV联手揭示}
\papertitlechnsub{从成核到扩展的完整能量图景}
\end{lstlisting}

在封面页中使用:

\begin{lstlisting}[style=blockstyle,language=TeX]
{\LARGE\textcolor{iqbblue}{\textbf{破解膜孔之谜:双CV联手揭示}}}\\[0.3cm]
{\LARGE\textcolor{iqbblue}{\textbf{从成核到扩展的完整能量图景}}}
\end{lstlisting}

这样设计支持快速修改标题,而无需在多处更新。

\subsubsection{特殊图片布局命令(Special Image Layouts)}

处理特殊宽高比图片时,使用智能布局命令自动优化图文配置。

\textbf{高竖版图布局:\lstinline|\iqbframeonethirdfig|}

用于特别高的竖版图片(高 > 宽×1.2),图占 1/3 左侧,文字占 2/3 右侧。图注写在文字栏底部:

\begin{lstlisting}[style=blockstyle,language=TeX]
\iqbframeonethirdfig{标题}{图片路径}{右侧文字包含图注}
% 或指定高度
\iqbframeonethirdfig[0.65\textheight]{标题}{图片路径}{文字}
\end{lstlisting}

\textbf{宽图布局:\lstinline|\iqbframetwothirdsfig|}

用于宽图(宽 ≥ 高×1.5),文字占 1/3 左侧,图占 2/3 右侧。图注可写在图片下方:

\begin{lstlisting}[style=blockstyle,language=TeX]
\iqbframetwothirdsfig{标题}{左侧文字}{图片路径}
\end{lstlisting}

\textbf{单图单注:\lstinline|\iqbframefigcaption|}

用于像素图或需要 1:1 显示的图片,图占主体,图注在下方:

\begin{lstlisting}[style=blockstyle,language=TeX]
\iqbframefigcaption{标题}{图片路径}{图注文字}
\end{lstlisting}

\subsubsection{Phase 1.5: 高级模式快捷命令(P0+P1 Priority Commands)}

继Phase 1(基础快捷命令)之后,Phase 1.5进一步识别并抽象了8个常见的重复模式,提供6个新的快捷命令,可再减少30-50行代码。

\textbf{P0优先级(立即实现)}

\textbf{1. 封面页一行式模板:\lstinline|\iqbcoverframe|}

将20行的手动封面页代码简化为1行,自动从元数据生成:

\begin{lstlisting}[style=blockstyle,language=TeX]
% 文档开头定义元数据
\papertitlechn{破解膜孔之谜:双CV联手揭示}
\papertitlechnsub{从成核到扩展的完整能量图景}
\author{高旭帆}
\institute{IQB Lab}

% 文档中调用
\iqbcoverframe  % 1 line instead of 20!
\end{lstlisting}

\textbf{优势}:代码减少 95\%,修改标题只需改元数据一处

\textbf{2. 高亮发现块:\lstinline|\iqbhighlight|}

用于突出核心科学发现或临界结论:

\begin{lstlisting}[style=blockstyle,language=TeX]
\iqbhighlight{临界发现}{
  脂质尾部密度 $\leftrightarrow$ 孔寿命 $\tau$ (R²=0.82)
}{相关性极强!}
\end{lstlisting}

\textbf{优势}:代码减少 80\%,自动颜色和格式管理

\textbf{3. 分步骤列表:\lstinline|\iqbsteplist| 和 \lstinline|\step|}

清晰展示多阶段流程:

\begin{lstlisting}[style=blockstyle,language=TeX]
\iqbsteplist{四阶段闭合动力学}{
  \step{(A) 平衡孔}{初始稳定孔,连续水柱贯穿膜厚度}
  \step{(B) 半径缩小}{孔边缘脂质重排,整体结构保持}
  \step{(C) 水线程}{关键瞬间,孔只有纤细连续水线}
  \step{(D) 膜恢复}{孔完全闭合,局部膜厚度恢复}
}
\end{lstlisting}

\textbf{优势}:代码减少 50\%,自动编号和间距管理

---

\textbf{P1优先级(下一批)}

\textbf{4. 带色彩强调的列表:\lstinline|\iqbcolorlist|}

用于展示带颜色分类的列表(✓准确 / ~部分正确 / ✗失败):

\begin{lstlisting}[style=blockstyle,language=TeX]
\iqbcolorlist{力场表现}{
  \item {\color{green!60!black}\textbf{C36}}:准确
  \item {\color{orange}\textbf{Slipids}}:部分正确
  \item {\color{red}\textbf{M2.2/M3}}:失败
}
\end{lstlisting}

\textbf{优势}:代码减少 57\%

\textbf{5. Bullet列表快捷命令:\lstinline|\iqbbulletlist|}

简化bullet列表:

\begin{lstlisting}[style=blockstyle,language=TeX]
\iqbbulletlist{1. 线张力计算}{
  \item Full-Path CV: 成核+扩展全过程
  \item Rapid CV: 快速评估大孔极限
}
\end{lstlisting}

\textbf{优势}:代码减少 63\%

\textbf{6. 增强型公式参数块:\lstinline|\iqbformparam| 和 \lstinline|\param|}

将公式和参数表格化展示:

\begin{lstlisting}[style=blockstyle,language=TeX]
\iqbformparam{切换函数}{
  $$s = \frac{1}{1 + e^{\alpha(\text{CV} - \text{CV}_0)}}$$
}{
  \param{\alpha = 20}{陡峭}
  \param{\text{CV}_0 = 0.95}{切换点}
}
\end{lstlisting}

\textbf{优势}:代码减少 40\%,参数表格化易读

---

\textbf{Phase 1.5 改进统计}

\begin{table}[!h]
\centering
\caption{Phase 1.5 新命令代码改进统计}
\label{tab:p1.5-improvement}
\begin{tabular}{lcccr}
\toprule
\textbf{命令} & \textbf{原方式} & \textbf{新方式} & \textbf{减少} & \textbf{级别} \\
\midrule
\lstinline|\iqbcoverframe| & 20行 & 1行 & 95\% & P0 \\
\lstinline|\iqbhighlight| & 5行 & 1行 & 80\% & P0 \\
\lstinline|\iqbsteplist| & 8行 & 4行 & 50\% & P0 \\
\lstinline|\iqbcolorlist| & 7行 & 3行 & 57\% & P1 \\
\lstinline|\iqbbulletlist| & 8行 & 3行 & 63\% & P1 \\
\lstinline|\iqbformparam| & 10行 & 6行 & 40\% & P1 \\
\bottomrule
\end{tabular}
\end{table}

对于典型演示,Phase 1.5 可额外减少 \textbf{30-50 行代码}。整合Phase 1+Phase 1.5,平均代码改进达到 \textbf{82\%}(从270行→50行)。

\subsection{故障排除与常见问题}

本小节列举用户使用过程中的常见问题及解决方案。

\subsubsection{编译错误}

\textbf{问题 1:文件找不到 (File not found)}

症状:编译时出现 \lstinline|! I can't find file `../theme/beamerthemeiqb.sty'|

原因:相对路径不正确。该错误通常发生在文件结构与模板预期不符时。

解决方案:
\begin{enumerate}
\item 确认目录结构与模板匹配,如下所示:
\begin{lstlisting}[style=blockstyle]
my-presentation/
├── examples/
│   └── my-jc.tex
├── theme/
│   ├── beamerthemeiqb.sty
│   ├── iqb-layouts.sty
│   └── images/
└── ...
\end{lstlisting}

\item 或在导言区调整相对路径,例如:
\begin{lstlisting}[style=blockstyle,language=TeX]
% 若 .tex 文件在项目根目录
\usepackage{theme/beamerthemeiqb}

% 若 .tex 文件在 examples/ 目录
\usepackage{../theme/beamerthemeiqb}

% 绝对路径(不推荐)
\usepackage{/home/user/my-project/theme/beamerthemeiqb}
\end{lstlisting}
\end{enumerate}

\textbf{问题 2:中文乱码或无法显示}

症状:编译后 PDF 中中文显示为方框或乱码

原因:未使用 XeLaTeX 编译,或未正确配置中文字体

解决方案:
\begin{enumerate}
\item 确认使用 XeLaTeX 编译(非 pdflatex):
\begin{lstlisting}[style=blockstyle,language=bash]
xelatex -interaction=nonstopmode your-file.tex
\end{lstlisting}

\item 检查字体配置:
\begin{lstlisting}[style=blockstyle,language=TeX]
\usepackage{xeCJK}
% Windows 用户
\setCJKmainfont{SimSun}

% macOS 用户
\setCJKmainfont{Songti SC}

% Linux 用户(需先安装 fonts-noto-cjk)
\setCJKmainfont{Noto Serif CJK SC}
\end{lstlisting}

\item Linux 用户若字体缺失,安装字体包:
\begin{lstlisting}[style=blockstyle,language=bash]
sudo apt-get install fonts-noto-cjk
\end{lstlisting}
\end{enumerate}

\textbf{问题 3:化学公式显示错误}

症状:使用 \lstinline|\ce{}| 出现 \lstinline|Undefined control sequence| 错误

原因:未加载 \lstinline|mhchem| 包

解决方案:在导言区添加:
\begin{lstlisting}[style=blockstyle,language=TeX]
\usepackage{mhchem}
\end{lstlisting}

然后使用 \lstinline|\ce{}| 命令:
\begin{lstlisting}[style=blockstyle,language=TeX]
\ce{H2O}        % 水
\ce{Ca2+}       % 钙离子
\ce{Na-Cl}      % 离子键
\end{lstlisting}

\subsubsection{布局与排版问题}

\textbf{问题 4:内容超出页面 (Overfull hbox/vbox)}

症状:编译时出现警告 \lstinline|Overfull \hbox| 或 \lstinline|Overfull \vbox|

原因:某一行或某一页的内容超过了允许的宽度或高度

解决方案:
\begin{enumerate}
\item 减少文字内容,提炼关键点
\item 调整图片大小,例如:
\begin{lstlisting}[style=blockstyle,language=TeX]
% 超过高度,改用更小的图片
\iqbfig[height=0.4\textheight]{image.png}{图注}

% 或拆分为两页
\end{lstlisting}

\item 使用更紧凑的布局,如三列而不是两列:
\begin{lstlisting}[style=blockstyle,language=TeX]
% 改为三列(节省空间)
\iqblayoutthree{内容1}{内容2}{内容3}
\end{lstlisting}

\item 启用详细的编译报告以定位具体行数:
\begin{lstlisting}[style=blockstyle,language=bash]
xelatex --file-line-error your-file.tex
\end{lstlisting}
\end{enumerate}

\textbf{问题 5:页眉或页脚不显示}

症状:某些页面缺少 header/footer

原因:使用了 \lstinline|[plain]| 或 \lstinline|[plain,noframenumbering]| 选项

这是正常的——这两个选项用于隐藏 header/footer(用于封面和致谢页)。

若要显示 header/footer,移除 \lstinline|[plain]| 选项:
\begin{lstlisting}[style=blockstyle,language=TeX]
% 错误(隐藏 header/footer)
\begin{frame}[plain]{标题}
  内容
\end{frame}

% 正确(显示 header/footer)
\begin{frame}{标题}
  内容
\end{frame}
\end{lstlisting}

\textbf{问题 6:图片不居中或对齐错误}

症状:图片在列中向左对齐或大小不符预期

原因:布局命令的参数设置或图片大小参数不当

解决方案:
\begin{enumerate}
\item 为列中的内容添加 \lstinline|\centering|:
\begin{lstlisting}[style=blockstyle,language=TeX]
\iqblayouttwo{
  \centering
  \includegraphics[height=0.5\textheight]{img1.png}
}{
  \centering
  \includegraphics[height=0.5\textheight]{img2.png}
}
\end{lstlisting}

\item 确保使用 \lstinline|height| 而不是 \lstinline|width|(宽度会导致比例失调):
\begin{lstlisting}[style=blockstyle,language=TeX]
% 正确:指定高度,宽度自动计算
\includegraphics[height=0.5\textheight,keepaspectratio]{image.png}

% 错误:只指定宽度可能拉伸图片
\includegraphics[width=\textwidth]{image.png}
\end{lstlisting}
\end{enumerate}

\subsubsection{功能与命令问题}

\textbf{问题 7:图片计数器不递增(使用 \texttt{iqbfig} 系列命令)}

症状:多个图片都显示"图1"而不是递增编号

原因:图片计数器未正确初始化或重置

解决方案:若要重置计数器(例如在新 section 开始),使用:
\begin{lstlisting}[style=blockstyle,language=TeX]
% 在新 section 开始处
\setcounter{iqbfigure}{0}

% 然后继续使用 \iqbfig 命令
\iqbfig{...}{...}  % 会重新从 图1 开始计数
\end{lstlisting}

\textbf{问题 8:自定义颜色不生效}

症状:\lstinline|\textcolor{iqbblue}{...}| 无法改变文字颜色

原因:颜色未在主题中定义

解决方案:检查 \lstinline|beamerthemeiqb.sty| 中的颜色定义,或自定义新颜色:
\begin{lstlisting}[style=blockstyle,language=TeX]
% 已定义的颜色(可直接使用)
% iqbblue, iqbdarkblue, iqblightblue, iqbgray, iqborange, iqbgreen, iqbred

% 自定义新颜色
\usepackage{xcolor}
\definecolor{mycolor}{RGB}{255,128,0}

% 然后使用
\textcolor{mycolor}{文字}
\end{lstlisting}

\textbf{问题 9:间距命令不起作用}

症状:使用 \lstinline|\iqbsep| 等间距命令但间距不变

原因:命令使用位置不当(例如在 block 内部或列环境边界处)

解决方案:
\begin{enumerate}
\item 在段落之间使用(而非在环境开始/结束处):
\begin{lstlisting}[style=blockstyle,language=TeX]
% 正确
\begin{itemize}
  \item 第一点
\end{itemize}
\iqbsep
\begin{block}{标题}
  内容
\end{block}

% 错误
\iqbsep
\begin{block}{...}
  ...
\end{block}
\end{lstlisting}

\item 若要调整 block 内部间距,使用标准的 LaTeX 命令:
\begin{lstlisting}[style=blockstyle,language=TeX]
\begin{block}{标题}
  \begin{itemize}
    \item 项 1
    \itemsep=0.1em  % 项目间距
    \item 项 2
  \end{itemize}
\end{block}
\end{lstlisting}
\end{enumerate}

\subsubsection{性能与编译速度}

\textbf{问题 10:编译速度很慢}

症状:每次编译需要 10 秒以上

原因:
\begin{itemize}
\item 加载了过多的图片(尤其是高分辨率图片)
\item 使用了复杂的 TikZ 绘图
\item 多次编译以更新交叉引用
\end{itemize}

解决方案:
\begin{enumerate}
\item 优化图片大小(使用合适的分辨率而非超高分辨率)
\item 若大量使用 TikZ,考虑使用 \lstinline|externalize| 库缓存 TikZ 输出:
\begin{lstlisting}[style=blockstyle,language=TeX]
\usepackage{tikz}
\usetikzlibrary{external}
\tikzexternalize[prefix=tikz/]
\end{lstlisting}

\item 只在最终版本时进行双重编译;开发阶段单次编译即可
\end{enumerate}

\subsubsection{其他常见问题}

\textbf{问题 11:主题颜色或字体无法修改}

若需完全自定义主题颜色或字体,编辑 \lstinline|theme/beamerthemeiqb.sty|:

\begin{lstlisting}[style=blockstyle,language=TeX]
% 修改 IQB 主题色(原值 #003366)
\definecolor{iqbblue}{RGB}{0, 51, 102}  % 改为你的颜色

% 修改标题字体大小
\setbeamerfont{frametitle}{size=\large, series=\bfseries}
\end{lstlisting}

然后重新编译演示文稿。

\textbf{问题 12:无法使用 \texttt{iqbauthorstwophoto} 加载图片}

症状:编译时出现图片路径错误

原因:图片路径不正确

解决方案:确保相对路径正确,例如:
\begin{lstlisting}[style=blockstyle,language=TeX]
% 若 .tex 在 examples/ 目录,图片在 examples/images/ 目录
\iqbauthorstwophoto{images/author1.jpg}{...}{...}{...}{...}{...}{...}{...}{...}

% 或使用绝对路径
\iqbauthorstwophoto{/absolute/path/to/author1.jpg}{...}{...}{...}{...}{...}{...}{...}{...}
\end{lstlisting}

\subsubsection{获取帮助}

若上述解决方案未能解决问题,建议:

\begin{enumerate}
\item 查看完整示例:\lstinline|examples/membrane-pore-jc.tex|
\item 查看模板源代码中的注释:
\begin{itemize}
  \item \lstinline|theme/beamerthemeiqb.sty|:主题相关问题
  \item \lstinline|theme/iqb-layouts.sty|:布局相关问题
  \item \lstinline|template/jc-template.tex|:最小示例和配置
\end{itemize}
\item 参考 Beamer 官方文档:\url{https://ctan.org/pkg/beamer}
\item 参考 LaTeX Wikibooks 中文排版:\url{https://en.wikibooks.org/wiki/LaTeX}
\end{enumerate}

\subsubsection{Phase 2 后续计划}

未来版本计划添加:
\begin{itemize}
\item \textbf{数据可视化增强}:集成 \lstinline|pgfplots| 用于快速绘制科学图表
\item \textbf{交互式目录}:自动生成章节导航和进度指示
\item \textbf{多主题支持}:提供暗色、学术、企业等预设主题
\item \textbf{参考文献管理}:集成 BibTeX 和自动引文格式化
\item \textbf{媒体集成}:支持视频、音频和动画嵌入
\end{itemize}

